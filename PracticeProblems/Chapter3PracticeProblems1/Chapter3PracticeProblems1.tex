\documentclass[10pt,a4paper]{article}
\usepackage[utf8]{inputenc}
\usepackage[a4paper,%
            left=.75in,right=.75in,top=1in,bottom=1in]{geometry}
\setlength{\headsep}{0.25in}

\usepackage{amsthm}

\usepackage{graphicx}
\usepackage{pgfplots}
            
\usepackage[english]{babel}

\theoremstyle{theorem}
\newtheorem{theorem}{Theorem}
\newtheorem{lemma}{Lemma}
\newtheorem{corollary}{Corollary}
\newtheorem{case}{Case}

\usepackage{amsthm}
\usepackage{lipsum}

\makeatletter
\newcommand{\proofpart}[2]{%
  \par
  \addvspace{\medskipamount}%
  \noindent\emph{Part #1: #2}\par\nobreak
  \addvspace{\smallskipamount}%
  \@afterheading
}
\makeatother

\newcommand\restr[2]{{% we make the whole thing an ordinary symbol
  \left.\kern-\nulldelimiterspace % automatically resize the bar with \right
  #1 % the function
  \vphantom{\big|} % pretend it's a little taller at normal size
  \right|_{#2} % this is the delimiter
  }}

\theoremstyle{definition}
\newtheorem{definition}{Definition}
\newtheorem{remark}{Remark}

\usepackage{mathtools}
\DeclarePairedDelimiter\bra{\langle}{\rvert}
\DeclarePairedDelimiter\ket{\lvert}{\rangle}
\DeclarePairedDelimiterX\braket[2]{\langle}{\rangle}{#1 \delimsize\vert #2}

\usepackage{amsmath}
\usepackage{amsfonts}
\usepackage{amssymb}
\usepackage{fancyhdr}

\DeclareMathOperator{\interior}{int}

\newcommand{\Tau}{\mathcal{T}}

\newenvironment{amatrix}[1]{%
  \left(\begin{array}{@{}*{#1}{c}|c@{}}
}{%
  \end{array}\right)
}

\usepackage{calligra}
\DeclareMathAlphabet{\mathcalligra}{T1}{calligra}{m}{n}
\DeclareFontShape{T1}{calligra}{m}{n}{<->s*[2.2]callig15}{}

\newcommand{\scripty}[1]{\ensuremath{\mathcalligra{#1}}}

\pagestyle{fancy}
\author{Jeremiah Givens}
\newcommand{\subject}{Real Analysis I}
\newcommand{\Date}{9/2/2021} 
\makeatletter
\rhead{{\small\@author}}
\lhead{{\small\subject}}
\chead{{\large Chapter 3 Practice Problems 1}}
\cfoot{}
\rfoot{\thepage}
\lfoot{\today}

\renewcommand{\theequation}{\arabic{equation}}

\begin{document}
\section*{Problem 1}
Let $I = [a_1, b_1] \times [a_2, b_2]$, a closed interval in $\mathbb{R}^2$. Given $\epsilon >0$, construct an interval $I_\epsilon$ such that the following conditions hold:
\begin{enumerate}
\item $I \subset \mathring{I_\epsilon}$
\item $v(I_\epsilon) - v(I) < \epsilon$.
\end{enumerate}
\subsection*{Solution}
Define a new interval
\begin{align*}
I_\epsilon &= [a_1 - \delta, b_1 + \delta] \times [a_2 - \delta, b_2 + \delta]
\end{align*}
for some $\delta > 0$. This clearly satisfies condition (1). Now, we must find an upper bound on $\delta$ such that condition 2 holds. Thus, we must find the value of $\delta$ such that 
\begin{align}
v(I_\epsilon) &= v(I) + \epsilon.
\end{align}
By definition, we have
\begin{align*}
v(I) &= (b_1 - a_1)(b_2 - a_2),
\end{align*}
and 
\begin{align*}
v(I_\epsilon)&= ((b_1 + \delta) - (a_1 - \delta))((b_2 + \delta) - (a_2 - \delta))\\
&= (b_1 - a_1 + 2 \delta)(b_2 - a_2 + 2 \delta)\\
&= 4 \delta^2 + 2\delta(b_1 + b_2 - a_1 - a_2) + v(I).
\end{align*}
Combining this with equation (1), we have
\begin{align*}
4 \delta^2 + 2\delta(b_1 + b_2 - a_1 - a_2) + v(I) &= v(I) + \epsilon\\
4 \delta^2 + 2\delta(b_1 + b_2 - a_1 - a_2) - \epsilon = 0.
\end{align*}
For brevity, we define
\begin{align*}
B &= 2(b_1 + b_2 - a_1 - a_2),
\end{align*}
so that we have
\begin{align*}
4 \delta^2 + B\delta - \epsilon = 0.
\end{align*}
Using the quadratic formula, we have
\begin{align*}
\delta &= \frac{-B \pm \sqrt{B^2 + 16 \epsilon}}{8}.
\end{align*}
Thus, choosing $\delta$ such that 
\begin{align*}
\delta &< \frac{-B \pm \sqrt{B^2 + 16 \epsilon}}{8},
\end{align*}
we have satisfied condition 2.

\section*{Problem 2}
Use the result in Problem 1 to show that given a sequence $\{I_k\}_{k=1}^\infty$ in $\mathbb{R}^2$ and $\epsilon > 0$, there exists a sequence of intervals $\{I_k^\epsilon\}_{k=1}^\infty$ such that 
\begin{enumerate}
\item $I_k \subset \mathring{I_k^\epsilon}$, $k=1,2,...$.
\item $\sum_{k=1}^\infty v(I_k^\epsilon) < \sum_{k=1}^\infty v(I_k) + \epsilon$
\end{enumerate}
\subsection*{Solution}
From our solution to Problem 1, we can create an interval $I_k^\epsilon$ such that condition 1 holds and
\begin{align*}
v(I_k^\epsilon) < v(I_k) + 2^{-k} \epsilon
\end{align*}
for all $k \in \mathbb{N}$. Thus, summing over all $k$, we have
\begin{align*}
\sum_{k=1}^\infty v(I_k^\epsilon) &\leq \sum_{k=1}^\infty (v(I_k) +  2^{-k} \epsilon)\\
&= \sum_{k=1}^\infty v(I_k) +  \sum_{k=1}^\infty 2^{-k} \epsilon\\
&= \sum_{k=1}^\infty v(I_k) +  \epsilon \sum_{k=1}^\infty 2^{-k} \\
&= \sum_{k=1}^\infty v(I_k) + \epsilon,
\end{align*}
and we have shown that $\{I_k^\epsilon\}_{k=1}^\infty$ satisfies condition 2.

\section*{Problem 3}
Use the definition of the outer measure to show:
\begin{enumerate}
\item Let $E \subseteq \mathbb{R}^2$ be countable. Then
\begin{align*}
|E|_e = 0.
\end{align*}
\item Let E be the edge (4 boundaries) of  an interval $I \subseteq \mathbb{R}^2$. Then 
\begin{align*}
|E|_e = 0.
\end{align*}
\end{enumerate}
\subsection*{Solution}
\begin{proof}
\proofpart{1}{} Since $E$ is countable, it can be denoted as 
\begin{align*}
E = \{x_k: k \in \mathbb{N}\}.
\end{align*}
By definition of outer measure, we have $|E|_e \geq 0$. Let $\epsilon > 0$. Define a cover $\{I_k \}$ of $E$ such that $I_k$ is an interval centered at $x_k$ with $v(I_k) = \epsilon 2^{-k}$. Then, we have
\begin{align*}
|E|_e &\leq \sum_{k =1}^\infty v(I_k)\\
&= \sum_{k =1}^\infty \epsilon 2^{-k}\\
&= \epsilon.
\end{align*}
Thus, we have proven 1.
\proofpart{2}{} Let $I = [a_1, b_1] \times [a_2, b_2]$, and let $\epsilon > 0$. The boundary of $I$ is given by
\begin{align*}
\partial I &= ([a_1, b_1] \times \{a_2\}) \cup ([a_1, b_1] \times \{b_2\}) \cup (\{a_1\} \times [a_2, b_2]) \cup (\{b_1\} \times [a_2, b_2]).
\end{align*}
Define \begin{align*}
\delta_v = \frac{\epsilon}{4(b_1 - a_1)}\\
\delta_h = \frac{\epsilon}{4(b_2 - a_2)}.
\end{align*}
Then, we can define a finite cover for $I$:
\begin{align*}
C = \{[a_1, b_1] \times [a_2 - \frac{\delta_v}{2}, a_2 + \frac{\delta_v}{2}], [a_1, b_1] \times [b_2 - \frac{\delta_v}{2}, b_2 + \frac{\delta_v}{2}], [a_1 - \frac{\delta_h}{2}, a_1 + \frac{\delta_h}{2}] \times [a_2, b_2],\\ [b_1 - \frac{\delta_h}{2}, b_1 + \frac{\delta_h}{2}] \times [a_2, b_2]\}.
\end{align*}
By design, we have $v(i) = \frac{\epsilon}{4}$ for all $i \in C$. Thus, we have 
\begin{align*}
0 &\leq |\partial I|_e\\
&\leq \sum_{i \in C} v(i)\\
&= \epsilon.
\end{align*}
Since $\epsilon$ was arbitrary, our proof is complete.
\end{proof}

\end{document}