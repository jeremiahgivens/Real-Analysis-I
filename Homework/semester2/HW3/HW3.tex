\documentclass[10pt,a4paper]{article}
\usepackage[utf8]{inputenc}
\usepackage[a4paper,%
            left=.75in,right=.75in,top=1in,bottom=1in]{geometry}
\setlength{\headsep}{0.25in}

\usepackage{amsthm}

\usepackage{graphicx}
\usepackage{pgfplots}
            
\usepackage[english]{babel}

\theoremstyle{theorem}
\newtheorem{theorem}{Theorem}
\newtheorem{lemma}{Lemma}
\newtheorem{corollary}{Corollary}
\newtheorem{case}{Case}

\usepackage{amsthm}
\usepackage{lipsum}
\usepackage{tikz}

\makeatletter
\newcommand{\proofpart}[2]{%
  \par
  \addvspace{\medskipamount}%
  \noindent\emph{Part #1: #2}\par\nobreak
  \addvspace{\smallskipamount}%
  \@afterheading
}
\makeatother

\newcommand\restr[2]{{% we make the whole thing an ordinary symbol
  \left.\kern-\nulldelimiterspace % automatically resize the bar with \right
  #1 % the function
  \vphantom{\big|} % pretend it's a little taller at normal size
  \right|_{#2} % this is the delimiter
  }}

\theoremstyle{definition}
\newtheorem{definition}{Definition}
\newtheorem{remark}{Remark}

\usepackage{mathtools}
\DeclarePairedDelimiter\bra{\langle}{\rvert}
\DeclarePairedDelimiter\ket{\lvert}{\rangle}
\DeclarePairedDelimiterX\braket[2]{\langle}{\rangle}{#1 \delimsize\vert #2}

\usepackage{amsmath}
\usepackage{amsfonts}
\usepackage{amssymb}
\usepackage{fancyhdr}
\usepackage{tkz-euclide}

\DeclareMathOperator{\interior}{int}

\newcommand{\Tau}{\mathcal{T}}

\newenvironment{amatrix}[1]{%
  \left(\begin{array}{@{}*{#1}{c}|c@{}}
}{%
  \end{array}\right)
}

\usepackage{calligra}
\DeclareMathAlphabet{\mathcalligra}{T1}{calligra}{m}{n}
\DeclareFontShape{T1}{calligra}{m}{n}{<->s*[2.2]callig15}{}

\newcommand{\scripty}[1]{\ensuremath{\mathcalligra{#1}}}

\pagestyle{fancy}
\author{Jeremiah Givens}
\newcommand{\subject}{Real Analysis II}
\newcommand{\Date}{9/2/2021} 
\makeatletter
\rhead{{\small\@author}}
\lhead{{\small\subject}}
\chead{{\large Homework 1}}
\cfoot{}
\rfoot{\thepage}
\lfoot{\today}

\renewcommand{\theequation}{\arabic{equation}}

\begin{document}
\section*{Problem 1.6}
\begin{theorem}
Let $\alpha > 0$. Then $f(x) = x^\alpha$ is absolutely continuous on every subinterval $[a, b] \subseteq [0, \infty)$.
\end{theorem}

\subsection*{Solution}
\begin{proof}
We have that $f$ is differentiable on $(0, \infty)$ with derivative
\begin{align*}
f'(x) = \alpha x^{\alpha - 1}.
\end{align*}
Thus, $f$ is differentiable almost everywhere on $[0, \infty)$. Now, let $[a, b] \subseteq [0, \infty)$. Since $f'$ is continuous a.e. on $[a, b]$, $f'$ is integrable on $[a, b]$. Furthermore, we have for any $x \in [a, b]$,
\begin{align*}
\int_a^x f'(x) dx &= \int_a^x \alpha x^{\alpha - 1} dx\\
&= \left. \alpha x^{\alpha} \right|_a^x\\
&= b^\alpha - x^\alpha\\
&= f(x) - f(a).
\end{align*}
Thus, by theorem 7.29 in our textbook, we have that $f$ is absolutely continuous on any subinterval of $[0, \infty)$.
\end{proof}

\section*{Problem 1.7}
\begin{theorem}
A function $f$ is absolutely continuous on $[a, b]$ if and only if given $\epsilon > 0$, there exists $\delta > 0$ such that $|\sum [f(b_i) - f(a_i)]| < \epsilon$ for any finite collection $\{[a_i, b_i]\}$ of nonoverlapping subintervals of $[a, b]$ with $\sum (b_i - a_i) < \delta$.
\end{theorem}

\subsection*{Solution}
\begin{proof}
Suppose $f$ is absolutely continuous on $[a, b]$, and let $\epsilon > 0$. Since $f$ is absolutely continuous, there exists a $\delta > 0$ such that $\sum |[f(b_i) - f(a_i)]| < \epsilon$ for any finite collection $\{[a_i, b_i]\}$ of nonoverlapping subintervals of $[a, b]$ with $\sum (b_i - a_i) < \delta$. Thus, if we let $\{[a_i, b_i]\}$ be a set of nonoverlapping subintervals of $[a, b]$ with $\sum (b_i - a_i) < \delta$, we have
\begin{align*}
\epsilon &> \sum |[f(b_i) - f(a_i)]|\\
&\geq |\sum [f(b_i) - f(a_i)]|, &&\text{Basic property of absolute value}
\end{align*}
and we have proven the forward direction.

Now, suppose that if we are given $\epsilon > 0$, there exists $\delta > 0$ such that $|\sum [f(b_i) - f(a_i)]| < \epsilon$ for any finite collection $\{[a_i, b_i]\}$ of nonoverlapping subintervals of $[a, b]$ with $\sum (b_i - a_i) < \delta$. Let $\epsilon > 0$, and choose $\delta > 0$ such that $|\sum [f(b_i) - f(a_i)]| < \frac{\epsilon}{2}$ for any finite collection $\{[a_i, b_i]\}$ of nonoverlapping subintervals of $[a, b]$ with $\sum (b_i - a_i) < \delta$. Let $\{[a_i, b_i]\}$ be a set of nonoverlapping subintervals of $[a, b]$ with $\sum (b_i - a_i) < \delta$. We have
\begin{align*}
\sum_{i \in \{i: f(b_i) \geq f(a_i)\}} (b_i - a_i) < \delta,
\end{align*}
which means that 
\begin{align*}
\frac{\epsilon}{2} &> \left|\sum_ {i \in \{i: f(b_i) \geq f(a_i)\}}[f(b_i) - f(a_i)]\right|\\
&= \sum_{i \in \{i: f(b_i) \geq f(a_i)\}}|f(b_i) - f(a_i)|.
\end{align*}
Similarly, we have
\begin{align*}
\sum_{i \in \{i: f(b_i) < f(a_i)\}} (b_i - a_i) < \delta,
\end{align*}
which implies
\begin{align*}
\frac{\epsilon}{2} &> \left|\sum_{i \in \{i: f(b_i) < f(a_i)\}}[f(b_i) - f(a_i)]\right|\\
&= \sum_ {i \in \{i: f(b_i) < f(a_i)\}}|f(b_i) - f(a_i)|.
\end{align*}
Finally, we have
\begin{align*}
\sum_ {i}|f(b_i) - f(a_i)| &= \sum_ {i \in \{i: f(b_i) < f(a_i)\}}|f(b_i) - f(a_i)| + \sum_ {i \in \{i: f(b_i) \geq f(a_i)\}}|f(b_i) - f(a_i)|\\
&\leq \frac{\epsilon}{2} + \frac{\epsilon}{2}\\
&= \epsilon,
\end{align*}
and we have shown that $f$ is absolutely continuous. With this, our proof is complete.
\end{proof}

\section*{Problem 1.8}
\begin{theorem}
If $f$ is of bounded variation on $[a, b]$, and if the function $V(x) = V[a, x]$ is absolutely continuous on $[a, b]$, then $f$ is absolutely continuous on $[a, b]$.
\end{theorem}

\subsection*{Solution}
\begin{proof}
Let $\epsilon > 0$. Since $V(x)$ is absolutely continuous, we have that there exists $\delta > 0$ such that $\sum |V(b_i) - V(a_i)| < \epsilon$ for any finite collection $\{[a_i, b_i]\}$ of nonoverlapping subintervals of $[a, b]$ with $\sum (b_i - a_i) < \delta$. Let $\{[a_i, b_i]\}$ be a collection of nonoverlapping subintervals of $[a, b]$ with $\sum (b_i - a_i) < \delta$. From Theorem 2.2 (part i) in our textbook, since $f$ is of bounded variation, and $V(x)$ is finite for all $x \in [a, b]$. Furthermore, from theorem 2.2 (part ii), we have
\begin{align*}
V[a, b] &= V[a, a_i] + V[a_i, b]
&= V[a, b_i] + V[b_i, b].
\end{align*}
Then, 
\begin{align*}
V(b_i) - V(a_i) &= V[a, b_i] - V[a, a_i] \\
&= V[a_i, b] - V[b_i, b]\\
&\geq V[a_i, b]\\
&\geq V[a_i, b_i] &&\text{Theorem 2.2 part i}\\
&\geq |f(b_i) - f(a_i)|.
\end{align*}
Finally, we have
\begin{align*}
\epsilon &> \sum |V(b_i) - V(a_i)|\\
&\geq \sum |f(b_i) - f(a_i)|,
\end{align*}
and we have proven that $f$ is absolutely continuous.
\end{proof}

\end{document}