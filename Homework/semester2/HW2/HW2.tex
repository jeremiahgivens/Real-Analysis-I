\documentclass[10pt,a4paper]{article}
\usepackage[utf8]{inputenc}
\usepackage[a4paper,%
            left=.75in,right=.75in,top=1in,bottom=1in]{geometry}
\setlength{\headsep}{0.25in}

\usepackage{amsthm}

\usepackage{graphicx}
\usepackage{pgfplots}
            
\usepackage[english]{babel}

\theoremstyle{theorem}
\newtheorem{theorem}{Theorem}
\newtheorem{lemma}{Lemma}
\newtheorem{corollary}{Corollary}
\newtheorem{case}{Case}

\usepackage{amsthm}
\usepackage{lipsum}
\usepackage{tikz}

\makeatletter
\newcommand{\proofpart}[2]{%
  \par
  \addvspace{\medskipamount}%
  \noindent\emph{Part #1: #2}\par\nobreak
  \addvspace{\smallskipamount}%
  \@afterheading
}
\makeatother

\newcommand\restr[2]{{% we make the whole thing an ordinary symbol
  \left.\kern-\nulldelimiterspace % automatically resize the bar with \right
  #1 % the function
  \vphantom{\big|} % pretend it's a little taller at normal size
  \right|_{#2} % this is the delimiter
  }}

\theoremstyle{definition}
\newtheorem{definition}{Definition}
\newtheorem{remark}{Remark}

\usepackage{mathtools}
\DeclarePairedDelimiter\bra{\langle}{\rvert}
\DeclarePairedDelimiter\ket{\lvert}{\rangle}
\DeclarePairedDelimiterX\braket[2]{\langle}{\rangle}{#1 \delimsize\vert #2}

\usepackage{amsmath}
\usepackage{amsfonts}
\usepackage{amssymb}
\usepackage{fancyhdr}
\usepackage{tkz-euclide}

\DeclareMathOperator{\interior}{int}

\newcommand{\Tau}{\mathcal{T}}

\newenvironment{amatrix}[1]{%
  \left(\begin{array}{@{}*{#1}{c}|c@{}}
}{%
  \end{array}\right)
}

\usepackage{calligra}
\DeclareMathAlphabet{\mathcalligra}{T1}{calligra}{m}{n}
\DeclareFontShape{T1}{calligra}{m}{n}{<->s*[2.2]callig15}{}

\newcommand{\scripty}[1]{\ensuremath{\mathcalligra{#1}}}

\pagestyle{fancy}
\author{Jeremiah Givens}
\newcommand{\subject}{Real Analysis II}
\newcommand{\Date}{9/2/2021} 
\makeatletter
\rhead{{\small\@author}}
\lhead{{\small\subject}}
\chead{{\large Homework 1}}
\cfoot{}
\rfoot{\thepage}
\lfoot{\today}

\renewcommand{\theequation}{\arabic{equation}}

\begin{document}
\section*{Problem 1}
\begin{theorem}
Let $f:[a,b] \to \mathbb{R}$ and $x_0 \in [a, b]$. Then the set of all subsequential derivatives of $f$ at $x_0$ is a closed set of $\bar{\mathbb{R}} = \mathbb{R} \cup \{-\infty, \infty \}$.
\end{theorem}

\subsection*{Solution}
\begin{proof}
Let $\Lambda$ be the set of all subsequential derivatives of $f$ at $x_0$, and let $\lambda$ be a boundary point of $\Lambda$. Define, for any real $h \not = 0$ with $h + x_0 \in [a, b]$,
\begin{align*}
F(h) = \frac{f(x_0 + h) - f(x_0)}{h}.
\end{align*}
Since $\lambda$ is a boundary point of $\Lambda$, we have that for all $m \in \mathbb{N}$, there exist $\lambda_m \in \Lambda$ such that $|\lambda - \lambda_m| < \frac{1}{2m}$. By definition of the subsequential derivative, we have that there exist sequences $\{h_n^m \}$ such that $h_n^m \not = 0$, $h_n^m \to 0$, and 
\begin{align*}
\lim_{n \to \infty} F(h_n^m) = \lambda_m.
\end{align*}
By definition of a limit, we have that for each $m$, there exists $N_m \in \mathbb{N}$ such that for all $n \geq N_m$, we have
\begin{align*}
|F(h_n^m) - \lambda_m| < \frac{1}{2m}.
\end{align*}
For each $m$, define a new sequence $\{h_m \}$ such that 
\begin{align*}
h_m = h_{N_m}^m.
\end{align*}
We will show that $F(h_m) \to \lambda$.

Let $\epsilon > 0$. Choose $M \in \mathbb{N}$ large enough such that $\frac{1}{M} < \epsilon$. We have, for all $m \geq M$,
\begin{align*}
|F(h_m) - \lambda| &\leq |F(h_m) - \lambda_m| + |\lambda - \lambda_m| &&\text{By the triangle inequality}\\
&= |F(h_{N_m}^m) - \lambda_m| + |\lambda - \lambda_m|\\
&< \frac{1}{2m} + \frac{1}{2m}\\
&= \frac{1}{m}\\
&< \frac{1}{M}\\
&< \epsilon.
\end{align*}
Thus, we have shown that $\lambda \in \Lambda$, and we can conclude that $\Lambda$ is closed.
\end{proof}

\section*{Problem 2}
\begin{theorem}
Let $f:[a,b] \to \mathbb{R}$ be monotone increasing. If 
\begin{align*}
\int_a^b f'(t)dt = f(b) - f(a),
\end{align*}
then
\begin{align*}
\int_a^x f'(t)dt = f(x) - f(a) && \forall x \in [a, b].
\end{align*}
\end{theorem}

\subsection*{Solution}
\begin{proof}
We have
\begin{align*}
f(b) - f(a) &= \int_a^b f'(t)dt\\
&= \int_a^x f'(t)dt + \int_x^b f'(t)dt\\
\int_a^x f'(t)dt &= f(b) - f(a) - \int_x^b f'(t)dt &&\text{Since the integrals are finite}\\
&\geq f(b) - f(a) - (f(b) - f(x)) &&\text{By the weak version of FTC}\\
&= f(x) - f(a).
\end{align*}
Once again utilizing the weak version of the FTC that we proved in class, we have
\begin{align*}
\int_a^x f'(t)dt \leq f(x) - f(a).
\end{align*}
Thus, combining these two inequalities, we have the desired result.
\end{proof}

\section*{Problem 3}
Let $f:[a,b] \to \mathbb{R}$ and $E \subseteq [a, b]$. Assume that $f'(x)$ exists for all $x \in E$, and satisfies 
\begin{align*}
|f'(x)| \leq M,
\end{align*}
for all $x \in E$, and some $M > 0$. Then,
\begin{align*}
|f(E)|_e \leq M |E|_e.
\end{align*}

\subsection*{Solution}
\begin{proof}
Let $\epsilon > 0$. Then, there exists an open set $G$ such that $E \subseteq G$ and $|G| \leq |E|_e + \epsilon$. Let $x_0 \in E$. Then $\exists \{h_n\}$ with $h_n \searrow 0$ such that 
\begin{align*}
 \lim_{n \to \infty} \left|\frac{f(x_0 + h_n) - f(x_0)}{h_n} \right| \leq M.
\end{align*}
Let $M < \tilde{M} < M + \epsilon$. Then, we have that $\exists N \in \mathbb{N}$ such that for all $n \geq N$, we have
\begin{align}
\left|\frac{f(x_0 + h_n) - f(x_0)}{h_n} \right| \leq \tilde{M}.
\end{align}
Without loss of generality, we assume that for all $n$ and $x_0 \in E$,
\begin{align}
I_n(x_0) = [x_0 - h_n/2, x_0 + h_n/2] \subseteq G.
\end{align}
Clearly, $\{I_n(x_0): x_0 \in E, n \in \mathbb{N}\}$ is a Vitali cover of $E$.

Apply the Vitali covering lemma to obtain countably many disjoint intervals
\begin{align*}
\{I_{n_i}(x_i)\} \subseteq \{I_n(x_0): x_0 \in E, n \in \mathbb{N}\}
\end{align*}
such that
\begin{align*}
\left| E \backslash \bigcup _i^\infty I_{n_i}(x_i) \right| = 0.
\end{align*}
By (1), we have
\begin{align*}
f(I_{n_i}(x_i)) \subseteq [f(x_i) - (h_{n_i}/2) \tilde{M}, f(x_i) + (h_{n_i}/2) \tilde{M}],
\end{align*}
which implies
\begin{align*}
|f(I_{n_i}(x_i))| \leq h_{n_i} \tilde{M}.
\end{align*}
\begin{align*}
|f(E)| &\leq \left|f\left( E \backslash \bigcup _i^\infty I_{n_i}(x_i) \right) \right| + \left| f\left(\bigcup _i^\infty I_{n_i}(x_i) \right) \right|\\
&= \left| f\left(\bigcup _i^\infty I_{n_i}(x_i) \right) \right| &&\text{Explained in (*)}\\
&\leq \left| \bigcup _i^\infty f\left(I_{n_i}(x_i) \right) \right|\\
&\leq \sum_i^\infty |f\left(I_{n_i}(x_i) \right)|\\
&\leq \sum_i^\infty h_{n_i} \tilde{M}\\
&= \sum_i^\infty|I_{n_i}(x_i)|\tilde{M}\\
&= \left| \bigcup _i^\infty I_{n_i}(x_i) \right| \tilde{M}\\
&\leq |G| \tilde{M} &&\text{By (2)}\\
&= \tilde{M}|E|_e.
\end{align*}
(*) Recall that functions with bounded derivatives are Lipschitz transformations. From the proof of 3.33 in our book, we know that Lipschitz transformations map sets of measure zero to sets of measure zero.

With this, our proof is complete.
\end{proof}

\section*{Problem 4}
\begin{theorem}
Let $f:[a, b] \to \mathbb{R}$ be continuous and have a subsequential derivative $\lambda_x \geq 0$ at each $x \in [a,b]$. Then $f$ is increasing on $[a, b]$.
\end{theorem}

\subsection*{Solution}
\begin{proof}
Let $x \in [a, b]$. By definition of the subsequential derivative, we have that there exists a sequence $\{h_n\}$ with $h_n \to 0$, $h_n + x \in [a, b]$, and $h_n \not = 0$, such that
\begin{align*}
\lim_{n \to \infty} \frac{f(x + h_n) - f(x)}{h_n} = \lambda_x \geq 0.
\end{align*}
Let $\epsilon > 0$. Define a new function $g(x) = f(x) + \epsilon x$. Then, we have that $g$ is continuous, and that 
\begin{align*}
\lim_{n \to \infty} \frac{g(x + h_n) - g(x)}{h_n} &= \lim_{n \to \infty} \frac{f(x + h_n) + \epsilon(x+h_n) - f(x) - \epsilon x}{h_n}\\
&= \lim_{n \to \infty} \frac{f(x + h_n) - f(x) + \epsilon h_n}{h_n}\\
&= \lim_{n \to \infty} \frac{f(x + h_n) - f(x)}{h_n} + \epsilon\\
&= \lambda_x + \epsilon\\
&\geq \epsilon.\\
\end{align*}
By definition of a limit, there exists an $N \in \mathbb{N}$ such that for all $n \geq N$, we have
\begin{align*}
\left| \frac{g(x + h_n) - g(x)}{h_n} - (\lambda_x + \epsilon)\right| < \epsilon.
\end{align*}
This implies that for all $n \geq N$, 
\begin{align*}
\frac{g(x + h_n) - g(x)}{h_n} > 0.
\end{align*}

Now, let $n \geq N$. Without loss of generality, assume that $h_n > 0$. Then we have that $g(x + h_n) > g(x)$.
\end{proof}

\section*{Problem 5}
\begin{theorem}
If $f$ is continuous on $[a, b]$ and $|f|$ is of bounded variation on $[a, b]$, then $f$ is of bounded variation on $[a, b]$.
\end{theorem}

\subsection*{Solution}
Before we prove this theorem, we will prove a useful lemma:
\begin{lemma}
Let $f$ be a function on $[a, b]$, and let $T_1 = \{a = x_0 < ... < x_n = b\}$ be a partition of $[a, b]$. If $T_2$ is a refinement of $T_1$, then 
\begin{align*}
V(f, T_1) \leq V(f, T_2).
\end{align*}
\end{lemma}

\begin{proof}[Proof of Lemma 1]
Since $T_2$ is a refinement of $T_1$, we have $T_1 \subseteq T_2$, and for some of the $i \in \{0, ..., n-1\}$, there exists an $x'_i \in T_2$ such that $x_i < x'_i < x_{i+1}$. For those $i$ where $T_2$ does not have such an $x'_i$, define $x'_i = x_i$. Then
\begin{align*}
V(f, T_1) &= \sum_{i=1}^n |f(x_i) - f(x_{i - 1})|\\
&\leq \sum_{i=1}^n \left(|f(x_{i - 1}) - f(x'_i)| + |f(x'_i) - f(x_i)| \right) && \text{By the triangle inequality}\\
&= V(f, T_2),
\end{align*}
and our proof is complete.
\end{proof}

Now we are ready to begin the proof of the theorem.

\begin{proof}[Proof of Theorem 4]
Let $T_1 = \{a = x_0 < ... < x_n = b\}$ be a partition of $[a, b]$. For the cases where $f(x_i)f(x_{i+1}) < 0$ (the cases where $f$ changes sign), use the intermediate value theorem (since $f$ is continuous) to define $x_i < x'_i < x_{i + 1}$ such that $f(x'_i) = 0$. As we did in the proof of Lemma 1, we define a refinement $T_2$ of $T_1$. We have 
\begin{align*}
V(f, T_1) &= \sum_{i=1}^n |f(x_i) - f(x_{i - 1})|.
\end{align*}
For the cases where $f(x_i)f(x_{i+1}) < 0$, we have
\begin{align*}
|f(x_i) - f(x_{i - 1})| &= |f(x_i)| + |f(x_{i - 1})|\\
&= ||f(x_{i - 1})| - |f(x'_i)|| + ||f(x'_i)| - |f(x_i)||.
\end{align*}
Otherwise, we have
\begin{align*}
|f(x_i) - f(x_{i - 1})| &= ||f(x_i)| - |f(x_{i+1})||\\
&\leq ||f(x_{i - 1})| - |f(x'_i)|| + ||f(x'_i)| - |f(x_i)||.
\end{align*}
Thus, we see
\begin{align*}
V(f, T_1) &= \sum_{i=1}^n |f(x_i) - f(x_{i - 1})|\\
&\leq \sum_{i=1}^n \left( ||f(x_{i - 1})| - |f(x'_i)|| + ||f(x'_i)| - |f(x_i)|| \right)\\
&= V(|f|, T_2)\\
&\leq V(|f|)\\
&< \infty.
\end{align*}
Since this is true for any $T_1$, we have shown that $V(f) \leq V(|f|)$, and our proof is complete.
\end{proof}
*Note, I did not end up using Lemma 1 here like I thought I would need to, but I am leaving it here incase I need it for any future problems.

\section*{Problem 6}
\begin{theorem}
Let $BV[a, b]$ be the normed vector space of all functions of bounded variation on $[a, b]$ with norm $||f|| = V_{a}^{b}(f) + |f(a)|$. Then, $BV[a, b]$ is a Banach space.
\end{theorem}

\subsection*{Solution}
\begin{proof}
Let $\{f_m\}$ be a Cauchy sequence of functions in $BV[a, b]$, where the metric is induced by our norm. We will first show that this converges point wise to some function $f$ on $[a, b]$. Since this sequence is Cauchy, we have that there exists an $N \in \mathbb{N}$ such that for all $j, k \geq N$, 
\begin{align*}
\epsilon &> ||f_j - f_k||\\
&= V_{a}^{b}(f_j - f_k) + |f_j(a) - f_k(a)|\\
&\geq |f_j(a) - f_k(a)|.
\end{align*}
Thus, $\{f_m(a) \}$ is a cauchy sequence, which converges to what we will define as $f(a)$.

Now consider the two point partition $T = \{a=x_0 < x_1 = b \}$. Then, we have
\begin{align*}
||f_j - f_k|| &= V_{a}^{b}(f_j - f_k) + |f_j(a) - f_k(a)|\\
&\geq V_{a}^{b}(f_j - f_k, T) + |f_j(a) - f_k(a)|\\
&= |(f_j(x_1) - f_k(x_1)) - (f_j(x_0) - f_k(x_0))| + |f_j(a) - f_k(a)|\\
&= |(f_j(b) - f_k(b)) - (f_j(a) - f_k(a))| + |f_j(a) - f_k(a)|\\
&= |(f_j(b) - f_k(b)) - (f_j(a) - f_k(a))|.
\end{align*}
This, with the fact that $\{f_m(a)\}$ is Cauchy (which means $(f_j(a) - f_k(a)) \to 0$), allows us to conclude that $\{f_m(b)\}$ is Cauchy, and converges to some real number $f(b)$.

Now consider the three point partition $T = \{a=x_0 < x_1 = x < x_2 = b \}$. Proceeding as before, we see
\begin{align*}
||f_j - f_k|| &= V_{a}^{b}(f_j - f_k) + |f_j(a) - f_k(a)|\\
&\geq V_{a}^{b}(f_j - f_k, T) + |f_j(a) - f_k(a)|\\
&\geq V_{a}^{b}(f_j - f_k, T)\\
&= \sum_{i=1}^2 |(f_j(x_i) - f_k(x_i)) - (f_j(x_{i-1}) - f_k(x_{i-1}))|\\
&= |(f_j(x) - f_k(x)) - (f_j(a) - f_k(a))| + |(f_j(b) - f_k(b)) - (f_j(x) - f_k(x))|.
\end{align*}
As before, we use the fact that $\{f_m(a)\}$ and $\{f_m(b)\}$ are Cauchy, to conclude that $\{f_m(x)\}$ is Cauchy for all $x \in (a, b)$, and converges to to what we will define as $f(x)$.

Since we have proved that this sequence does converge to some function $f$, we must now show that $f$ is of bounded variation. Let $T = \{a = x_0 < ... < x_n = b\}$ be a partition of $[a, b]$. For some $\epsilon > 0$, choose $N \in \mathbb{N}$ large enough so that $|f_N(x_i) - f(x_i)| < \frac{\epsilon}{2n}$ for all $x_i \in T$. Since $\{f_n\}$ is Cauchy, the sequence is bounded, thus there exists an $M$ such that $V(f_n) < M$ for all $n$. Then, we have
\begin{align*}
V(f, T) &= \sum_{i=1}^n |f(x_i) - f(x_{i - 1})|\\
&\leq \sum_{i=1}^n |f(x_i) - f_N(x_i)| + \sum_{i=1}^n |f_N(x_i) - f(x_{i - 1})| \\
&\leq \sum_{i=1}^n |f(x_i) - f_N(x_i)|  + \sum_{i=1}^n |f_N(x_i) - f_N(x_{i - 1})| + \sum_{i=1}^n |f_N(x_{i-1}) - f(x_{i - 1})| \\
&\leq \sum_{i=1}^n \frac{\epsilon}{2n}  + \sum_{i=1}^n |f_N(x_i) - f_N(x_{i - 1})| + \sum_{i=1}^n \frac{\epsilon}{2n} \\
&= \frac{\epsilon}{2} + \sum_{i=1}^n |f_N(x_i) - f_N(x_{i - 1})| + \frac{\epsilon}{2}\\
&= \epsilon + \sum_{i=1}^n |f_N(x_i) - f_N(x_{i - 1})|\\
&= \epsilon + V(f_N, T)\\
&\leq \epsilon + V(f_N)\\
&< \epsilon + M.
\end{align*}
Since this is true for an parition $T$, we can conclude that $V(f) \leq M$, and we have shown that $f \in BV[a, b]$.
\end{proof}

\section*{Problem 7}
Let 
\[   f = \left\{
\begin{array}{ll}
      x \sin \left( \frac{\pi}{2x} \right), &  x \in (0, 1]\\
      0, & x = 0\\
\end{array} 
\right. \]
and recall that $f \not \in BV[0, 1]$. Construct a sequence $\{f_n\}$ such that $f_n \in BV[0, 1]$ and $f_n \to f$ uniformly on $[0, 1]$.

\subsection*{Solution}
By the Stone-Weierstrass theorem, we can construct a sequence of polynomials $\{f_n\}$ that converges uniformly to $f$ on [0, 1]. Since the derivative of a polynomial is another polynomial, we have that $f'_n$ is continuous for all $n$. Thus, since $[0, 1]$ is closed, and continuous functions on closed intervals are bounded, we have that there exists some $M_n \in \mathbb{R}$ such that $|f'_n(x)| \leq M_n$. We have
\begin{align*}
V(f_n) &= \int_0^1 |f'(x)| dx &&\text{By Corollary 2.10 in our textbook}\\
&\leq \int_0^1 M_n dx &&\text{By Theorem 5.5 (i) in our textbook}\\
&= M_n &&\text{By Corollary 5.4 in our textbook}\\
&< \infty.
\end{align*}
Thus, each $f_n$ is of bounded variation.
\end{document}