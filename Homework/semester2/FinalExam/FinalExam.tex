\documentclass[10pt,a4paper]{article}
\usepackage[utf8]{inputenc}
\usepackage[a4paper,%
            left=.75in,right=.75in,top=1in,bottom=1in]{geometry}
\setlength{\headsep}{0.25in}

\usepackage{amsthm}

\usepackage{graphicx}
\usepackage{pgfplots}
            
\usepackage[english]{babel}

\theoremstyle{theorem}
\newtheorem{theorem}{Theorem}
\newtheorem{lemma}{Lemma}
\newtheorem{corollary}{Corollary}
\newtheorem{case}{Case}

\usepackage{amsthm}
\usepackage{lipsum}
\usepackage{tikz}

\makeatletter
\newcommand{\proofpart}[2]{%
  \par
  \addvspace{\medskipamount}%
  \noindent\emph{Part #1: #2}\par\nobreak
  \addvspace{\smallskipamount}%
  \@afterheading
}
\makeatother

\newcommand\restr[2]{{% we make the whole thing an ordinary symbol
  \left.\kern-\nulldelimiterspace % automatically resize the bar with \right
  #1 % the function
  \vphantom{\big|} % pretend it's a little taller at normal size
  \right|_{#2} % this is the delimiter
  }}

\theoremstyle{definition}
\newtheorem{definition}{Definition}
\newtheorem{remark}{Remark}

\usepackage{mathtools}
\DeclarePairedDelimiter\bra{\langle}{\rvert}
\DeclarePairedDelimiter\ket{\lvert}{\rangle}
\DeclarePairedDelimiterX\braket[2]{\langle}{\rangle}{#1 \delimsize\vert #2}

\usepackage{amsmath}
\usepackage{amsfonts}
\usepackage{amssymb}
\usepackage{fancyhdr}
\usepackage{tkz-euclide}

\DeclareMathOperator{\interior}{int}

\newcommand{\Tau}{\mathcal{T}}

\newenvironment{amatrix}[1]{%
  \left(\begin{array}{@{}*{#1}{c}|c@{}}
}{%
  \end{array}\right)
}

\usepackage{calligra}
\DeclareMathAlphabet{\mathcalligra}{T1}{calligra}{m}{n}
\DeclareFontShape{T1}{calligra}{m}{n}{<->s*[2.2]callig15}{}

\newcommand{\scripty}[1]{\ensuremath{\mathcalligra{#1}}}

\pagestyle{fancy}
\author{Jeremiah Givens}
\newcommand{\subject}{Real Analysis II}
\newcommand{\Date}{9/2/2021} 
\makeatletter
\rhead{{\small\@author}}
\lhead{{\small\subject}}
\chead{{\large Final Exam}}
\cfoot{}
\rfoot{\thepage}
\lfoot{\today}

\renewcommand{\theequation}{\arabic{equation}}

\begin{document}
\section*{Problem 1}
\begin{theorem}
Let $f:[a, b] \to \mathbb{R}$ be absolutely continuous. Then $|f|$ is absolutely continuous on $[a, b]$, and 
\begin{align*}
\left| \frac{d}{dx}|f(x)| \right| \leq |f'(x)|
\end{align*}
for almost every $x \in [a, b]$.
\end{theorem}

\subsection*{Solution}
\begin{proof}
Let $\epsilon > 0$. Since $f$ is absolutely continuous, there exists a $\delta > 0$ such for any collection $\{[a_i, b_i]\}$ of nonoverlapping subintervals of $[a, b]$, we have
\begin{align*}
\sum |f(b_i) - f(a_i)| < \epsilon \text{ if } \sum (b_i - a_i) < \delta.
\end{align*}
Thus, let $\{[a_i, b_i]\}$ be a collection of nonoverlapping subintervals of $[a, b]$ with $\sum (b_i - a_i) < \delta$. Then, we have
\begin{align*}
\sum ||f(b_i)| - |f(a_i)|| &\leq \sum |f(b_i) - f(a_i)| &&\text{The reverse triangle inequality}\\
&< \epsilon.
\end{align*}
Thus, $|f|$ is absolutely continuous.

By theorem 7.29 in our textbook, the derivatives of $f$ and $|f|$ exists almost everywhere in $[a, b]$. Let $Z, Z' \subseteq [a, b]$ be the sets of measure zero where the derivatives of $f$ and $|f|$ respectively do not exist. Then, $Z \cup Z'$ has measure zero, and the derivatives of $|f|$ and $f$ exists on $[a, b] \backslash Z \cup Z'$. If we let $x \in [a, b] \backslash Z \cup Z'$, then we have
\begin{align*}
\left| \frac{d}{dx}|f(x)| \right| &= \left| \lim_{n \to 0} \frac{|f(x+n)| - |f(x)|}{n} \right|\\
&= \lim_{n \to 0} \frac{||f(x+n)| - |f(x)||}{|n|}\\
&\leq \lim_{n \to 0} \frac{|f(x+n)| - |f(x)|}{|n|} &&\text{Reverse triangle inequality}\\
&= \left| \lim_{n \to 0} \frac{f(x+n) - f(x)}{n} \right|\\
&= |f'(x)|,
\end{align*}
as desired.
\end{proof}

\section*{Problem 2}
Use Fubini's theorem and the relation 
\begin{align*}
\frac{1}{x} = \int_0^\infty e^{-xy} dy
\end{align*}
to prove that 
\begin{align*}
\lim_{A \to \infty} \int_0^A \frac{\sin x}{x}dx &= \frac{\pi}{2}.
\end{align*}

\subsection*{Solution}
\begin{proof}
Consider the function 
\begin{align*}
f(x, y) = \sin (x) e^{-xy}. 
\end{align*}
We have that $f$ is continuous, which implies that $f$ is initegrable on $(0, A) \times (0, \infty)$. Thus, by Fubini's theorem, we have
\begin{align*}
\int_0^A \frac{\sin x}{x}dx &= \int_0^\infty \left( \int_0^A\sin (x) e^{-xy}  dx \right) dy.
\end{align*}
By performing integration by parts twice, or by recognizing that 
\begin{align*}
\sin (x) e^{-xy} &= - \frac{1}{1 + y^2} \frac{d}{dx} (y \sin (x) e^{-xy} + \cos (x) e^{-xy}),
\end{align*}
we have
\begin{align*}
\int_0^A\sin (x) e^{-xy}  dx &= \int_0^A \left(- \frac{1}{1 + y^2} \frac{d}{dx} (y \sin (x) e^{-xy} + \cos (x) e^{-xy}) \right)dx\\
&= \frac{1}{1 + y^2} [1 - e^{-Ay}(y \sin A + \cos A)].
\end{align*}
Plugging this into our original integral, we have
\begin{align*}
\int_0^\infty \left( \int_0^A\sin (x) e^{-xy}  dx \right) dy &= \int_0^\infty \frac{1}{1 + y^2} [1 - e^{-Ay}(y \sin A + \cos A)] dy\\
&= \int_0^\infty \frac{1}{1 + y^2}dy  - \int_0^\infty \frac{e^{-Ay}(y \sin A + \cos A)}{1 + y^2} dy\\
&= \left. \tan^{-1} (y) \right|_0^\infty - \int_0^\infty \frac{e^{-Ay}(y \sin A + \cos A)}{1 + y^2} dy\\
&= \frac{\pi}{2} - \int_0^\infty \frac{e^{-Ay}(y \sin A + \cos A)}{1 + y^2} dy.
\end{align*}
Thus, we have
\begin{align*}
\lim_{A \to \infty} \int_0^A \frac{\sin x}{x}dx &= \frac{\pi}{2} - \lim_{A \to \infty}\int_0^\infty \frac{e^{-Ay}(y \sin A + \cos A)}{1 + y^2} dy.
\end{align*}
We can see that 
\begin{align*}
\left| \frac{e^{-Ay}(y \sin A + \cos A)}{1 + y^2} \right| &\leq e^{-Ay}(y + 1).
\end{align*}
Thus, since 
\begin{align*}
\int_0^\infty e^{-Ay}(y + 1) dy = \frac{1}{A} + \frac{1}{A^2},
\end{align*}
we have that $e^{-Ay}(y + 1)$ is integrable. By the Lebesgue dominated convergence theorem,
\begin{align*}
\lim_{A \to \infty}\int_0^\infty \frac{e^{-Ay}(y \sin A + \cos A)}{1 + y^2} dy &= \int_0^\infty \lim_{A \to \infty} \frac{e^{-Ay}(y \sin A + \cos A)}{1 + y^2} dy\\
&= \int_0^\infty (0) dy\\
&= 0.
\end{align*}
With this, our proof is complete.
\end{proof}

\section*{Problem 3}
\begin{theorem}
Let $E = [1, \infty)$ and $f \in L^2(E)$. Assume that $f \geq 0$ almost every where on $E$. Let 
\begin{align*}
g(x) = \int_E f(y) e^{-xy} dy
\end{align*}
for all $x \in E$. Then $g \in L^1(E)$ and 
\begin{align*}
||g||_1 \leq c ||f||_2
\end{align*}
for some $c < 1$. (Also Estimate $c$).
\end{theorem}

\subsection*{Solution}
\begin{proof}

\end{proof}
\end{document}