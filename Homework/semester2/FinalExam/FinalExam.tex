\documentclass[10pt,a4paper]{article}
\usepackage[utf8]{inputenc}
\usepackage[a4paper,%
            left=.75in,right=.75in,top=1in,bottom=1in]{geometry}
\setlength{\headsep}{0.25in}

\usepackage{amsthm}

\usepackage{graphicx}
\usepackage{pgfplots}
            
\usepackage[english]{babel}

\theoremstyle{theorem}
\newtheorem{theorem}{Theorem}
\newtheorem{lemma}{Lemma}
\newtheorem{corollary}{Corollary}
\newtheorem{case}{Case}

\usepackage{amsthm}
\usepackage{lipsum}
\usepackage{tikz}

\makeatletter
\newcommand{\proofpart}[2]{%
  \par
  \addvspace{\medskipamount}%
  \noindent\emph{Part #1: #2}\par\nobreak
  \addvspace{\smallskipamount}%
  \@afterheading
}
\makeatother

\newcommand\restr[2]{{% we make the whole thing an ordinary symbol
  \left.\kern-\nulldelimiterspace % automatically resize the bar with \right
  #1 % the function
  \vphantom{\big|} % pretend it's a little taller at normal size
  \right|_{#2} % this is the delimiter
  }}

\theoremstyle{definition}
\newtheorem{definition}{Definition}
\newtheorem{remark}{Remark}

\usepackage{mathtools}
\DeclarePairedDelimiter\bra{\langle}{\rvert}
\DeclarePairedDelimiter\ket{\lvert}{\rangle}
\DeclarePairedDelimiterX\braket[2]{\langle}{\rangle}{#1 \delimsize\vert #2}

\usepackage{amsmath}
\usepackage{amsfonts}
\usepackage{amssymb}
\usepackage{fancyhdr}
\usepackage{tkz-euclide}

\DeclareMathOperator{\interior}{int}

\newcommand{\Tau}{\mathcal{T}}

\newenvironment{amatrix}[1]{%
  \left(\begin{array}{@{}*{#1}{c}|c@{}}
}{%
  \end{array}\right)
}

\usepackage{calligra}
\DeclareMathAlphabet{\mathcalligra}{T1}{calligra}{m}{n}
\DeclareFontShape{T1}{calligra}{m}{n}{<->s*[2.2]callig15}{}

\newcommand{\scripty}[1]{\ensuremath{\mathcalligra{#1}}}

\pagestyle{fancy}
\author{Jeremiah Givens}
\newcommand{\subject}{Real Analysis II}
\newcommand{\Date}{9/2/2021} 
\makeatletter
\rhead{{\small\@author}}
\lhead{{\small\subject}}
\chead{{\large Final Exam}}
\cfoot{}
\rfoot{\thepage}
\lfoot{\today}

\renewcommand{\theequation}{\arabic{equation}}

\begin{document}
\section*{Problem 1}
\begin{theorem}
Let $f:[a, b] \to \mathbb{R}$ be absolutely continuous. Then $|f|$ is absolutely continuous on $[a, b]$, and 
\begin{align*}
\left| \frac{d}{dx}|f(x)| \right| \leq |f'(x)|
\end{align*}
for almost every $x \in [a, b]$.
\end{theorem}

\subsection*{Solution}
\begin{proof}
Let $\epsilon > 0$. Since $f$ is absolutely continuous, there exists a $\delta > 0$ such for any collection $\{[a_i, b_i]\}$ of nonoverlapping subintervals of $[a, b]$, we have
\begin{align*}
\sum |f(b_i) - f(a_i)| < \epsilon \text{ if } \sum (b_i - a_i) < \delta.
\end{align*}
Thus, let $\{[a_i, b_i]\}$ be a collection of nonoverlapping subintervals of $[a, b]$ with $\sum (b_i - a_i) < \delta$. Then, we have
\begin{align*}
\sum ||f(b_i)| - |f(a_i)|| &\leq \sum |f(b_i) - f(a_i)| &&\text{The reverse triangle inequality}\\
&< \epsilon.
\end{align*}
Thus, $|f|$ is absolutely continuous.

By theorem 7.29 in our textbook, the derivatives of $f$ and $|f|$ exists almost everywhere in $[a, b]$. Let $Z, Z' \subseteq [a, b]$ be the sets of measure zero where the derivatives of $f$ and $|f|$ respectively do not exist. Then, $Z \cup Z'$ has measure zero, and the derivatives of $|f|$ and $f$ exists on $[a, b] \backslash Z \cup Z'$. If we let $x \in [a, b] \backslash Z \cup Z'$, then we have
\begin{align*}
\left| \frac{d}{dx}|f(x)| \right| &= \left| \lim_{n \to 0} \frac{|f(x+n)| - |f(x)|}{n} \right|\\
&= \lim_{n \to 0} \frac{||f(x+n)| - |f(x)||}{|n|}\\
&\leq \lim_{n \to 0} \frac{|f(x+n)| - |f(x)|}{|n|} &&\text{Reverse triangle inequality}\\
&= \left| \lim_{n \to 0} \frac{f(x+n) - f(x)}{n} \right|\\
&= |f'(x)|,
\end{align*}
as desired.
\end{proof}

\section*{Problem 2}
Use Fubini's theorem and the relation 
\begin{align*}
\frac{1}{x} = \int_0^\infty e^{-xy} dy
\end{align*}
to prove that 
\begin{align*}
\lim_{A \to \infty} \int_0^A \frac{\sin x}{x}dx &= \frac{\pi}{2}.
\end{align*}

\subsection*{Solution}
\begin{proof}
Consider the function 
\begin{align*}
f(x, y) = \sin (x) e^{-xy}. 
\end{align*}
We have that $f$ is continuous, which implies that $f$ is initegrable on $(0, A) \times (0, \infty)$. Thus, by Fubini's theorem, we have
\begin{align*}
\int_0^A \frac{\sin x}{x}dx &= \int_0^\infty \left( \int_0^A\sin (x) e^{-xy}  dx \right) dy.
\end{align*}
By performing integration by parts twice, or by recognizing that 
\begin{align*}
\sin (x) e^{-xy} &= - \frac{1}{1 + y^2} \frac{d}{dx} (y \sin (x) e^{-xy} + \cos (x) e^{-xy}),
\end{align*}
we have
\begin{align*}
\int_0^A\sin (x) e^{-xy}  dx &= \int_0^A \left(- \frac{1}{1 + y^2} \frac{d}{dx} (y \sin (x) e^{-xy} + \cos (x) e^{-xy}) \right)dx\\
&= \frac{1}{1 + y^2} [1 - e^{-Ay}(y \sin A + \cos A)].
\end{align*}
Plugging this into our original integral, we have
\begin{align*}
\int_0^\infty \left( \int_0^A\sin (x) e^{-xy}  dx \right) dy &= \int_0^\infty \frac{1}{1 + y^2} [1 - e^{-Ay}(y \sin A + \cos A)] dy\\
&= \int_0^\infty \frac{1}{1 + y^2}dy  - \int_0^\infty \frac{e^{-Ay}(y \sin A + \cos A)}{1 + y^2} dy\\
&= \left. \tan^{-1} (y) \right|_0^\infty - \int_0^\infty \frac{e^{-Ay}(y \sin A + \cos A)}{1 + y^2} dy\\
&= \frac{\pi}{2} - \int_0^\infty \frac{e^{-Ay}(y \sin A + \cos A)}{1 + y^2} dy.
\end{align*}
Thus, we have
\begin{align*}
\lim_{A \to \infty} \int_0^A \frac{\sin x}{x}dx &= \frac{\pi}{2} - \lim_{A \to \infty}\int_0^\infty \frac{e^{-Ay}(y \sin A + \cos A)}{1 + y^2} dy.
\end{align*}
We can see that 
\begin{align*}
\left| \frac{e^{-Ay}(y \sin A + \cos A)}{1 + y^2} \right| &\leq e^{-Ay}(y + 1).
\end{align*}
Thus, since 
\begin{align*}
\int_0^\infty e^{-Ay}(y + 1) dy = \frac{1}{A} + \frac{1}{A^2},
\end{align*}
we have that $e^{-Ay}(y + 1)$ is integrable. By the Lebesgue dominated convergence theorem,
\begin{align*}
\lim_{A \to \infty}\int_0^\infty \frac{e^{-Ay}(y \sin A + \cos A)}{1 + y^2} dy &= \int_0^\infty \lim_{A \to \infty} \frac{e^{-Ay}(y \sin A + \cos A)}{1 + y^2} dy\\
&= \int_0^\infty (0) dy\\
&= 0.
\end{align*}
With this, our proof is complete.
\end{proof}

\section*{Problem 3}
\begin{theorem}
Let $E = [1, \infty)$ and $f \in L^2(E)$. Assume that $f \geq 0$ almost every where on $E$. Let 
\begin{align*}
g(x) = \int_E f(y) e^{-xy} dy
\end{align*}
for all $x \in E$. Then $g \in L^1(E)$ and 
\begin{align*}
||g||_1 \leq c ||f||_2
\end{align*}
for some $c < 1$. (Also Estimate $c$).
\end{theorem}

\subsection*{Solution}
\begin{proof}
We have
\begin{align*}
||g||_1 &= \int_E |g| dx\\
&= \int_1^\infty \left|\int_E f(y) e^{-xy} dy \right| dx\\
&= \int_1^\infty \int_1^\infty f(y) e^{-xy} dy dx\\
&= \int_1^\infty f(y) \left( \int_0^\infty e^{-xy} dx - \int_0^1 e^{-xy} dx\right) dy\\
&= \int_1^\infty f(y) \left(\frac{1}{y} - \int_0^1 e^{-xy} dx\right) dy\\
&= \int_1^\infty f(y)\frac{1}{y} dy - \int_1^\infty \left( \int_0^1 f(y)e^{-xy} dx \right) dy\\
&\leq ||f||_2 \cdot \sqrt{\int_1^\infty y^{-2} dy} - \int_1^\infty \left( \int_0^1 f(y)e^{-xy} dx \right) dy &&\text{Cauchy-Schwarz's}\\
&= ||f||_2 - \int_1^\infty \left( \int_0^1 f(y)e^{-xy} dx \right)\\
&= ||f||_2 \left( 1 - \frac{1}{||f||_2}\int_1^\infty \left( \int_0^1 f(y)e^{-xy} dx \right) dy \right).
\end{align*}
Thus, we have shown that there is some $c < 1$ such that $|g||_1 \leq c ||f||_2$.

To estimate $c$, let's find a lower bound on $c$. Finding a lower bound on $c$ can be achieved by finding an upper bound on the integral in that last equation. We have
\begin{align*}
\int_1^\infty \left( \int_0^1 f(y)e^{-xy} dx \right) dy &= \int_1^\infty f(y)\frac{1}{y}(1 - e^{-y}) dy\\
&\leq ||f||_2 \cdot \sqrt{\int_1^\infty \left(\frac{1 - e^{-y}}{y}\right)^2} dy &&\text{Holder's Inequality}\\
&< ||f||_2 \cdot \sqrt{\frac{3}{4}} &&\text{Wolfram alpha for computing integral}\\
&= ||f||_2 \cdot \frac{\sqrt{3}}{2}.
\end{align*}
Thus, we have
\begin{align*}
||g||_1 < \left(1 - \frac{\sqrt{3}}{2}\right) \cdot ||f||_2.
\end{align*}
\end{proof}

\section*{Problem 4}
\begin{theorem}
Let $f \in L^1(\mathbb{R})$ and 
\begin{align*}
g(x) = \int_\mathbb{R} f(y) e^{-(x - y)^2} dy
\end{align*}
for $x \in \mathbb{R}$. Then $g \in L^p(\mathbb{R})$ for any $1 \leq p \leq \infty$, and
\begin{align*}
||g||_p \leq \left( \frac{\pi}{p} \right)^{1/p} \cdot ||f||_1
\end{align*}
for $1 \leq p < \infty$.
\end{theorem}

\subsection*{Solution}
\begin{proof}
From theorem 9.1 in our textbook, we have $g \in L^p(\mathbb{R})$, and
\begin{align*}
||g||_p &\leq ||f||_1 \cdot ||e^{-(x - y)^2}||_p\\
&= ||f||_1 \cdot \left( \int_{\mathbb{R}^2} e^{-p(x - y)^2} \right)^{1/p}\\
&= ||f||_1 \cdot \left( \int_{\mathbb{R}^2} e^{-px^2} e^{-py^2} \right)^{1/p}\\
&= ||f||_1 \cdot \left( \int_\mathbb{R} e^{-px^2} dx  \int_\mathbb{R} e^{-py^2} dy \right)^{1/p}\\
&= ||f||_1 \cdot \left( \sqrt{\frac{\pi}{p}}  \sqrt{\frac{\pi}{p}}  \right)^{1/p}\\
&= \left( \frac{\pi}{p} \right)^{1/p} \cdot ||f||_1.
\end{align*}
\end{proof}

\section*{Problem 5}
\begin{theorem}
Let $f:[a, b] \to \mathbb{R}$ be continuous and satisfy the midpoint convexity
\begin{align*}
f \left(\frac{x_1 + x_2}{2} \right) \leq \frac{f(x_1) + f(x_2)}{2}.
\end{align*}
for any $x_1, x_2 \in [a, b]$. Then $f$ is convex on $[a, b]$.
\end{theorem}

\subsection*{Solution}
\begin{proof}
We will first prove, that for any $n \in \mathbb{N}$, if $x_1,...,x_{2^n} \in [a, b]$, then
\begin{align*}
f\left( \frac{x_1 + ... + x_{2^n}}{2^n} \right) \leq \frac{f(x_1) + ... + f(x_{2^n})}{2^n}.
\end{align*}
To prove this, we note that the base case ($n=1$) is true by midpoint convexity of $f$. Now, suppose for some $n \in \mathbb{N}$ that if $x_1,...,x_{2^n} \in [a, b]$, then
\begin{align*}
f\left( \frac{x_1 + ... + x_{2^n}}{2^n} \right) \leq \frac{f(x_1) + ... + f(x_{2^n})}{2^n}.
\end{align*}
Let $x_1,...,x_{2^{n+1}} \in [a, b]$. Then, we have
\begin{align*}
f\left( \frac{x_1 + ... + x_{2^{n+1}}}{2^{n+1}} \right) &= f\left( \frac{\frac{x_1 + ... + x_{2^n}}{2^n} + \frac{x_{2^n + 1} + ... + x_{2^{n+1}}}{2^n}}{2} \right)\\
&\leq \frac{f\left( \frac{x_1 + ... + x_{2^n}}{2^n}\right) + f \left( \frac{x_{2^n + 1} + ... + x_{2^{n+1}}}{2^n} \right)}{2}\\
&= \frac{f(x_1) + ... + f(x_{2^{n+1}})}{2^{n+1}},
\end{align*}
and our induction is complete.

From elementary analysis, we have that rational numbers of the form $\frac{m}{2^n}$ with $1 \leq m \leq 2^n$ are dense in $[0, 1]$. Let $[a_1, b_2] \subseteq [a, b]$. Fix some $n \in \mathbb{N}$ and some $1 \leq m \leq 2^n$. Setting $x_i = a_i$ for $1 \leq i \leq m$, and $x_i = b_i$ for $m+1 \leq i \leq 2^n$, we see from the above argument that
\begin{align*}
f\left( \frac{x_1 + ... + x_{2^n}}{2^n} \right) &\leq \frac{f(x_1) + ... + f(x_{2^n})}{2^n}\\
&= \frac{mf(a_i) + (2^n - m)f(b_i)}{2^n}\\
&= \frac{m}{2^n}f(a_i) + (1 - \frac{m}{2^n})f(b_i).
\end{align*}
Finally, let $\theta \in [0, 1]$ be a real number. For each $n \in \mathbb{N}$, define $1 \leq m_n \leq 2^n$ to be the largest number such that $\frac{m_n}{2^n} \geq \theta$. Then, $\frac{m_n}{2^n} \to \theta$, and it follows from the continuity of $f$ that 
\begin{align*}
f(\theta a_i + (1 - \theta)b_i) \leq \theta f(a_i) + (1 - \theta)f(b_i),
\end{align*}
and we have shown that $f$ is convex.
\end{proof}


\section*{Problem 6}
\begin{theorem}
If $f_k \to f$ in $L^p(\mathbb{R}^n), 1 < p < \infty$, $g_k \to g$ pointwise, and $||g_k||_\infty \leq M$ for all $k$, then 
\begin{align*}
f_k g_k \to fg
\end{align*}
in $L^p (\mathbb{R}^n)$.
\end{theorem}

\subsection*{Solution}
\begin{proof}
We have
\begin{align*}
||fg - f_kg_k||_p &= ||fg - g_kf + g_k f - f_k g_k||_p\\
&\leq ||fg - g_kf||_p + ||g_k f - f_k g_k||_p\\
&\leq ||fg - g_kf||_p + ||g_k (f - f_k)||_p\\
&\leq ||fg - g_kf||_p + ||M (f - f_k)||_p\\
&= ||fg - g_kf||_p + M||f - f_k||_p.
\end{align*}
Since $f_k \to f$ in $L^p(\mathbb{R}^n)$, the second term in the last equation will vanish as $k \to \infty$. Thus, we must show that $||fg - g_kf||_p \to 0$. We have
\begin{align*}
||fg - g_kf||_p^p &= ||f(g - g_k)||_p^p\\
&= \int_\mathbb{R} |f(g - g_k)|^p\\
&= \int_\mathbb{R} |g - g_k|^p|f|^p.
\end{align*}
Now, $|g - g_k|^p|f|^p \leq (2M)^p|f|^p \in L(\mathbb{R}^n)$, thus, by the lebesgue dominated convergence theorem, we have
\begin{align*}
\lim_{k \to \infty} \int_{\mathbb{R}^n} |g - g_k|^p|f|^p &=  \int_{\mathbb{R}^n} \lim_{k \to \infty} |g - g_k|^p|f|^p\\
&= 0,
\end{align*}
and our proof is complete.
\end{proof}

\section*{Problem 7}
\begin{theorem}
Suppose $f_k \to f$ a.e. on $\mathbb{R}^n$ and that $f_k, f \in L^p(\mathbb{R}^n)$, for $1 < p < \infty$. If $||f_k||_p \leq M < \infty$, then 
\begin{align*}
f_kg \to fg 
\end{align*}
in $L^1(\mathbb{R}^n)$ for all $g \in L^{p'}(\mathbb{R}^n)$, with 
\begin{align*}
\frac{1}{p} + \frac{1}{p'} = 1.
\end{align*}
\end{theorem}

\subsection*{Solution}
\begin{proof}
We have
\begin{align*}
||f_k g - f g ||_1 &= ||g(f_k - f)||_p\\
&= ||g||_{p'} \cdot ||f_k - f||_p &&\text{Holder's inequality.}
\end{align*}
Thus, it will suffice to show that $||f_k - f||_p \to 0$. We have
\begin{align*}
|f_k - f|^p &\leq (2 \max\{|f|, |f_k|\})^p\\
&= 2^p \max\{|f|^p, |f_k|^p\}\\
&\leq 2^p (|f|^p + |f_k|^p).
\end{align*}
Now, 
\begin{align*}
\int_{\mathbb{R}^n} 2^p (|f|^p + |f_k|^p) &= 2^p \int_{\mathbb{R}^n}|f|^p + 2^p \int_{\mathbb{R}^n}|f_k|^p\\
&= 2^p ||f||_p^p + 2^p ||f_k||_p^p\\
&\leq 2^p ||f||_p^p + 2^p M^p.
\end{align*}
Thus, we can use the Lebesgue dominated convergence theorem to yield
\begin{align*}
\lim_{k \to \infty} \int_{\mathbb{R}^n} |f_k - f|^p &= \int_{\mathbb{R}^n} \lim_{k \to \infty} |f_k - f|^p\\
&= 0,
\end{align*}
and our proof is complete.
\end{proof}

\section*{Problem 8}
\begin{theorem}
Let $f \in L^2 (\mathbb{R})$, and $g(x) = x f(x) \in L^2 (\mathbb{R})$. Then $f \in L^1 (\mathbb{R})$ and 
\begin{align*}
||f||_1 \leq \sqrt{2}(||f||_2 + ||g||_2).
\end{align*}
\end{theorem}

\subsection*{Solution}
\begin{proof}
We have
\begin{align*}
||f||_1 &= \int_\mathbb{R} |f|\\
&= \int_{|x| < 1} |f| + \int_{|x| \geq 1} |f|\\
&= \int_{|x| < 1} |(1)f| + \int_{|x| \geq 1} \left|\frac{1}{x}g \right|\\
&\leq \left( \int_{|x| < 1} 1 \right)^{1/2} \cdot \left( \int_{|x| < 1} |f|^2 \right)^{1/2} + \left( \int_{|x| \geq 1} \left| \frac{1}{x} \right|^2 \right)^{1/2} \cdot \left( \int_{|x| \geq 1} |g|^2 \right)^{1/2} &&\text{Cauchy Schwarz's}\\
&= \sqrt{2} \left( \int_{|x| < 1} |f|^2 \right)^{1/2} + \sqrt{2} \left( \int_{|x| \geq 1} |g|^2 \right)^{1/2}\\
&= \sqrt{2} \left( \int_{\mathbb{R}} |f|^2 \right)^{1/2} + \sqrt{2} \left( \int_{\mathbb{R}} |g|^2 \right)^{1/2}\\
&= \sqrt{2}(||f||_2 + ||g||_2),
\end{align*}
as desired.
\end{proof}

\section*{Problem 9}
\begin{theorem}
Let 
\begin{align*}
f(x) = \begin{cases} 
      x^{3/2} \sin\frac{1}{x} & x \in (0, 1] \\
     0 & \text{if } x=0 \\
\end{cases}
\end{align*}
Then $f$ is absolutely continuous on $[0, 1]$.
\end{theorem}

\subsection*{Solution}
\begin{proof}
For $x \in (0, 1]$, we have
\begin{align*}
f'(x) = \frac{3}{2}x^{1/2} \sin \frac{1}{x} + x^{1/2} \cos \frac{1}{x},
\end{align*}
and $f'(0) = 0$. Thus, we have for all $x \in [0, 1]$, 
\begin{align*}
|f'(x)| \leq \frac{5}{2}.
\end{align*}
Since every function on a compact interval with a bounded derivative is Lipschitz continuous, we can conclude that $f$ is Lipschitz continuous. Since Lipschitz continuity implies absolute continuity, our proof is complete.
\end{proof}
\end{document}