\documentclass[10pt,a4paper]{article}
\usepackage[utf8]{inputenc}
\usepackage[a4paper,%
            left=.75in,right=.75in,top=1in,bottom=1in]{geometry}
\setlength{\headsep}{0.25in}

\usepackage{amsthm}

\usepackage{graphicx}
\usepackage{pgfplots}
            
\usepackage[english]{babel}

\theoremstyle{theorem}
\newtheorem{theorem}{Theorem}
\newtheorem{lemma}{Lemma}
\newtheorem{corollary}{Corollary}
\newtheorem{case}{Case}

\usepackage{amsthm}
\usepackage{lipsum}
\usepackage{tikz}

\makeatletter
\newcommand{\proofpart}[2]{%
  \par
  \addvspace{\medskipamount}%
  \noindent\emph{Part #1: #2}\par\nobreak
  \addvspace{\smallskipamount}%
  \@afterheading
}
\makeatother

\newcommand\restr[2]{{% we make the whole thing an ordinary symbol
  \left.\kern-\nulldelimiterspace % automatically resize the bar with \right
  #1 % the function
  \vphantom{\big|} % pretend it's a little taller at normal size
  \right|_{#2} % this is the delimiter
  }}

\theoremstyle{definition}
\newtheorem{definition}{Definition}
\newtheorem{remark}{Remark}

\usepackage{mathtools}
\DeclarePairedDelimiter\bra{\langle}{\rvert}
\DeclarePairedDelimiter\ket{\lvert}{\rangle}
\DeclarePairedDelimiterX\braket[2]{\langle}{\rangle}{#1 \delimsize\vert #2}

\usepackage{amsmath}
\usepackage{amsfonts}
\usepackage{amssymb}
\usepackage{fancyhdr}
\usepackage{tkz-euclide}

\DeclareMathOperator{\interior}{int}

\newcommand{\Tau}{\mathcal{T}}

\newenvironment{amatrix}[1]{%
  \left(\begin{array}{@{}*{#1}{c}|c@{}}
}{%
  \end{array}\right)
}

\usepackage{calligra}
\DeclareMathAlphabet{\mathcalligra}{T1}{calligra}{m}{n}
\DeclareFontShape{T1}{calligra}{m}{n}{<->s*[2.2]callig15}{}

\newcommand{\scripty}[1]{\ensuremath{\mathcalligra{#1}}}

\pagestyle{fancy}
\author{Jeremiah Givens}
\newcommand{\subject}{Real Analysis II}
\newcommand{\Date}{9/2/2021} 
\makeatletter
\rhead{{\small\@author}}
\lhead{{\small\subject}}
\chead{{\large Homework 5}}
\cfoot{}
\rfoot{\thepage}
\lfoot{\today}

\renewcommand{\theequation}{\arabic{equation}}

\begin{document}
\section*{Problem 1}
Use Holder's inequality to show 
\begin{align*}
\int_0^1 \sqrt{x} (1 - x)^{-1/3} dx \leq \frac{2}{5^{1/3}}.
\end{align*}

\subsection*{Solution}
Using Holder's inequality, let $p = 3$, and let $p' = 3/2$. Then, we have
\begin{align*}
\int_0^1 \sqrt{x} (1 - x)^{-1/3} dx &\leq \left( \int_0^1 \sqrt{x}^{3} dx \right)^{1/3} \left( \int_0^1 ((1 - x)^{-1/3})^{3/2} dx \right)^{2/3}\\
&= \left( \int_0^1 x^{3/2} dx \right)^{1/3} \left( \int_0^1 u^{-1/2} du \right)^{2/3}\\
&= \left( \frac{2}{5} \right)^{1/3}  2^{2/3}\\
&=  \frac{2}{5^{1/3}},
\end{align*}
as desired.

\section*{Problem 2}
Let $E \subseteq \mathbb{R}^n$ be measurable with $|E| = 1$. Let $h \geq 0$ be measurable on $E$. Let $A = \int_E h dx$. Show that 
\begin{align*}
\sqrt{1 + A^2} \leq \int_E \sqrt{1 + h^2} dx \leq 1 + A.
\end{align*}

\subsection*{Solution}
(I've not yet figured out the proof of the first inequality, which would go here.)

For the second inequality, we have
\begin{align*}
\int_E \sqrt{1 + h^2} dx &\leq \int_E \sqrt{1 + h^2 + 2h} dx\\
&= \int_E \sqrt{(1 + h)^2} dx\\
&= \int_E 1 + h dx\\
&= 1 + A.
\end{align*}

\section*{Problem 3}
Find all nonnegative functions $g \in L^3(0, 1)$ such that 
\begin{align*}
\left( \int_0^1 x g(x)dx \right)^3 = \frac{4}{25} \int_0^1 g^3(x)dx
\end{align*}

\subsection*{Solution}
Using Holders inequality, we can see that if $p = \frac{3}{2}$, and $p' = 3$, then
\begin{align*}
\int_0^1 x g(x)dx &\leq \left(\int_0^1 x^{3/2} dx \right)^{2/3} \left(\int_0^1  g^3(x)dx \right)^{1/3}\\
&= \left( \frac{2}{5} \right)^{2/3} \left(\int_0^1  g^3(x)dx \right)^{1/3}\\
&= \left( \frac{4}{25} \int_0^1  g^3(x)dx \right)^{1/3}\\
\left( \int_0^1 x g(x)dx \right)^3 &= \frac{4}{25} \int_0^1 g^3(x)dx.
\end{align*}
Now, as we proved in class, the equality holds if and only if $\alpha x^{3/2} = g^3(x)$ almost everywhere for some real $\alpha$. Thus, we have
\begin{align*}
g(x) = \alpha x^\frac{1}{2}
\end{align*}
for some nonnegative real number $\alpha$ and for almost every $x$.

\section*{Problem 4}
Let $f \in L^\infty (0, 1)$ and $||f||_\infty \leq 1$. Show that 
\begin{align*}
\int_0^1 \sqrt{1 - f^2(x)} dx \leq \sqrt{1 - \left( \int_0^1 f(x)dx \right)^2},
\end{align*}
and describe the class of functions $f$ for which equality takes place.

\subsection*{Solution}
We have
\begin{align*}
\int_0^1 \sqrt{1 - f^2(x)} dx &= \int_0^1 \sqrt{(1 - f)(1 + f)} dx\\
&= \int_0^1 \sqrt{1 - f} \sqrt{1 + f} dx\\
&\leq \sqrt{\int_0^1 (1 - f) dx}\sqrt{\int_0^1 (1 + f) dx} && \text{Cauchy-Schwarz's inequality}\\
&= \sqrt{1 - \int_0^1 f dx}\sqrt{1 + \int_0^1 f dx}\\
&= \sqrt{1 - \left( \int_0^1 f(x)dx \right)^2}.
\end{align*}
From the proof of Holder's Inequality, we know that the equality holds iff there exists some $\alpha \in \mathbb{R}$ such that $1 - f = \alpha(1 + f)$. Thus, 
\begin{align*}
1 - f = \alpha(1 + f) &\implies 1 - f = \alpha + \alpha f\\
&\implies 1 - \alpha = (1 + \alpha)f\\
&\implies f = \frac{1 - \alpha}{1 + \alpha}.
\end{align*}

\section*{Problem 5}
\begin{theorem}
Prove that 
\begin{align*}
\int_0^\infty e^{-x} \sqrt{x^4 + 3x^2 + 2} dx &\leq \sqrt{12},
\end{align*}
and that the equality does not hold.
\end{theorem}

\subsection*{Solution}
\begin{proof}
We have
\begin{align*}
\int_0^\infty e^{-x} \sqrt{x^4 + 3x^2 + 2} dx &= \int_0^\infty \sqrt{e^{-2x}(x^4 + 3x^2 + 2)} dx\\
&= \int_0^\infty \sqrt{e^{-x}(x^2 + 2)e^{-x}(x^2 + 1)} dx\\
&= \int_0^\infty \sqrt{e^{-x}(x^2 + 2)} \sqrt{e^{-x}(x^2 + 1)} dx\\
&\leq \sqrt{ \int_0^\infty e^{-x}(x^2 + 2)dx} \sqrt{\int_0^\infty e^{-x}(x^2 + 1)dx} dx && \text{Cauchy-Schwarz's inequality}\\
&= \sqrt{ 2 + 2 } \sqrt{2 + 1} && \text{Apply integration by parts twice}\\
&= \sqrt{12}.
\end{align*}
Assume that the equality holds. Then, there exists some $\alpha \in \mathbb{R}$ such that for almost every $x \in [0, \infty)$, we have
\begin{align*}
e^{-x}(x^2 + 2) &= \alpha e^{-x}(x^2 + 1).
\end{align*}
With this, we see
\begin{align*}
x^2 + 2 = \alpha x^2 + \alpha &\implies x^2(1 - \alpha) = 2 + \alpha\\
&\implies x^2 = \frac{1 - \alpha}{2 + \alpha}.
\end{align*}
Thus, for any given $\alpha$, this can only be true for at most two points in $[0, \infty)$, which contradicts our assumption that this is true almost everywhere. Therefore, we can conclude that the equality does not hold, as desired.
\end{proof}

\section*{Problem 6}
Let $f: \mathbb{R}^n \to \mathbb{R}$ be continuous and bounded. Show that 
\begin{align*}
||f||_\infty = \sup \{|f(x)| : x \in \mathbb{R}^n \}.
\end{align*}

\subsection*{Solution}
Let $M = \sup \{|f(x)| : x \in \mathbb{R}^n \}$, and let $\alpha < M$. Then, by definition of supremum, we have
\begin{align*}
\{x \in \mathbb{R}^n : |f(x)| > M \} = \emptyset &\implies |\{x \in \mathbb{R}^n : |f(x)| > M \}| = 0\\
&\implies M \geq ||f||_\infty.
\end{align*}
Thus, we just need to show that $M \leq ||f||_\infty$, and our proof will be complete. To do this, let's suppose for sake of contradiction that $M > ||f||_\infty$. Then, there exists some nonnegative $\alpha < M$ such that 
\begin{align*}
|\{x \in \mathbb{R}^n| |f(x)| > \alpha \}| = 0.
\end{align*}
We have
\begin{align*}
\{x \in \mathbb{R}^n| |f(x)| > \alpha \} &= |f|^{-1}((\alpha, \infty)).
\end{align*}
Since $f$ is continuous, $|f|$ is continuous. Thus, since $(\alpha, \infty)$ is open, we can conclude that $|f|^{-1}((\alpha, \infty))$ is open. Now, by definition of supremum, we have that there exists some $x \in \mathbb{R}^n$ such that $\alpha < |f(x)| \leq M$. Thus, $|f|^{-1}((\alpha, \infty))$ is nonempty. Since nonempty open subsets of $\mathbb{R}^n$ have positive measure, we can conclude that
\begin{align*}
|\{x \in \mathbb{R}^n| |f(x)| > \alpha \}| > 0
\end{align*}
and we have reached a contradiction. Thus, $M \leq ||f||_\infty$, and our proof is complete.

\section*{Problem 7}
\begin{theorem}
Let $\{f_k\} \subset L^p(E)$ for some $1 \leq p < \infty$. Assume that $|E| < \infty$, $||f_k||_p \leq A$ for each natural $k$, and that 
\begin{align*}
\lim_{k \to \infty} f_k(x) = f(x)
\end{align*}
for almost every $x \in E$. Then $f_k \to f$ in $L^1$.
\end{theorem}

\subsection*{Solution}
To prove this, we will start by proving an intuitive lemma.
\begin{lemma}
Let $E \subseteq \mathbb{R}^n$, and let $f \in L(E)$. Then for any $\epsilon > 0$, there exists a $\delta > 0$ such that if $A \subseteq E$ with $|A| < \delta$, then 
\begin{align*}
\int_A |f| < \epsilon.
\end{align*}
Informally, this is known as the fact that integrals over small sets are small.
\end{lemma}
\begin{proof}
Let $\epsilon > 0$, and let $A \subseteq E$. Let $g$ be a simple function with $0 \leq g \leq |f|$. Then
\begin{align*}
\int_A (|f| - g) \leq \int_E (|f| - g) &\implies \int_A |f| \leq \int_A g + \int_E |f| - \int_E g.
\end{align*}
By theorem 4.13 in our textbook, and the monotone convergence theorem, we can choose $g$ such that 
\begin{align*}
\int_E |f| - \int_E g < \frac{\epsilon}{2}.
\end{align*}
If we define $M = \max g$, then 
\begin{align*}
\int_A g &\leq |A|M.
\end{align*}
Thus, if we let $\delta = \frac{\epsilon}{2M}$, and $A$ is such that $|A| < \delta$, then we have 
\begin{align*}
\int_A |f| &\leq \int_A g + \int_E |f| - \int_E g\\
&< \frac{\epsilon}{2} + \frac{\epsilon}{2}\\
&= \epsilon,
\end{align*}
and our proof is complete.
\end{proof}
Now we are ready for the proof of the main theorem.
\begin{proof}
Let $q$ be the conjugate exponent of $p$, and let $F \subseteq E$ be measurable. If $p > 1$, we have
\begin{align*}
\int_F |f_k| &\leq \left( \int_F 1 \right)^{1/q} \left( \int_F |f|^p \right)^{1/p} &&\text{By Holder's inequality}\\
&\leq \left( \int_F 1 \right)^{1/q} \left( \int_E |f|^p \right)^{1/p}\\
&= |F|^{1/q} ||f_k||_p\\
&\leq |F|^{1/q} A.
\end{align*}
Let $\epsilon > 0$. By Ergorov's theorem, there exists a closed set $F \subseteq E$ such that $f_k$ converges uniformly to $f$ on $F$, and $|E \backslash F| < \epsilon$. We have
\begin{align*}
\lim_{k \to \infty} \int_E |f_k - f| &= \lim_{k \to \infty} \int_{F} |f_k - f| + \int_{E \backslash F} |f_k - f|\\
&= 0 + \lim_{k \to \infty} \int_{E \backslash F} |f_k - f| &&\text{By the uniform convergence theorem}\\
&\leq \int_{E \backslash F} |f_k| + \int_{E \backslash F} |f|\\
&\leq |E \backslash F|^{1/q} A + \int_{E \backslash F} \liminf |f_k|\\
&\leq |E \backslash F|^{1/q} A + \liminf \int_{E \backslash F} |f_k| &&\text{By Fatou's Lemma}\\
&\leq 2 |E \backslash F|^{1/q} A\\
&\leq 2|\epsilon|^{1/q} A.
\end{align*}
Thus, taking the limit as $\epsilon \to 0$, we have proven that $f_k$ converges to $f$ in $L^1$ if $p > 1$.

Now consider the case of $p = 1$. Then, we have
\begin{align*}
\lim_{k \to \infty} \int_E |f_k - f| &= \lim_{k \to \infty} \int_{F} |f_k - f| + \int_{E \backslash F} |f_k - f|\\
&= 0 + \lim_{k \to \infty} \int_{E \backslash F} |f_k - f| &&\text{By the uniform convergence theorem}\\
&\leq \int_{E \backslash F} |f_k| + \int_{E \backslash F} |f|\\
&\leq \int_{E \backslash F} |f_k| + \int_{E \backslash F} \liminf |f_k|\\
&\leq \int_{E \backslash F} |f_k| + \liminf \int_{E \backslash F} |f_k| &&\text{By Fatou's Lemma}.
\end{align*}
Thus, taking the limit as $\epsilon \to 0$, the above lemma allows us to conclude that $f_k$ converges to $f$ in $L^1$, and our proof is complete.
\end{proof}

\section*{Problem 8}
\begin{theorem}
Let $1 \leq p < \infty$ and assume that $f \in L^p(\mathbb{R})$. Then
\begin{align*}
\lim_{t \to \infty} \int_t^{t + 1} f(x) dx = 0.
\end{align*}
\end{theorem}

\subsection*{Solution}
\begin{proof}
We have
\begin{align*}
\lim_{t \to \infty} \int_t^{t + 1} |f(x)| dx = 0 &\implies  \lim_{t \to \infty} \left| \int_t^{t + 1} f(x) dx \right| = 0\\
&\implies \lim_{t \to \infty} \int_t^{t + 1} f(x) dx = 0,
\end{align*}
so it will suffice to show that $\lim_{t \to \infty} \int_t^{t + 1} f(x) dx = 0$.

Suppose, for sake of contradiction, that $\lim_{t \to \infty} \int_t^{t + 1} f(x) dx \not = 0$. Then, there exists some $\epsilon > 0$ such that for any natural $N$, there exists an $t \geq N$ such that 
\begin{align*}
\int_t^{t + 1} f(x) dx > \epsilon.
\end{align*}
Let $t_0$ be any natural number such that 
\begin{align*}
\int_{t_0}^{t_0 + 1} f(x) dx > \epsilon,
\end{align*}
and define a sequence $\{t_n \}_{n=1}^\infty$ such that $t_n \geq t_{n-1} + 1$, and 
\begin{align*}
\int_{t_n}^{t_n + 1} f(x) dx > \epsilon.
\end{align*}
Then, the intervals $[t_{n}, t_{n} + 1]$ are non overlapping, and we have
\begin{align*}
\int_{\mathbb{R}} |f(x)|dx &\geq \sum_{n = 0}^\infty \int_{t_n}^{t_n + 1} f(x) dx \\
&> \sum_{n = 0}^\infty \epsilon\\
&= \infty.
\end{align*}
Thus, $f \not \in L^1$ and by theorem 8.2 $f \not \in L^p$, which is a contradiction. Thus, we have proven the theorem.
\end{proof}

\section*{Problem 9}
\begin{theorem}
Let $p,q,r \in [1, \infty)$ such that
\begin{align*}
\frac{1}{r} = \frac{1}{p} + \frac{1}{q}.
\end{align*}
If $f \in L^p(\mathbb{R}^n)$ and $g \in L^p(\mathbb{R}^n)$, then
\begin{align*}
||fg||_r \leq ||f||_p \cdot ||g||_q.
\end{align*}
\end{theorem}

\subsection*{Solution}
\begin{proof}
Define $p' = \frac{p}{r}$ and $q' = \frac{q}{r}$. Then $p'$ and $q'$ are conjugate exponents. We have
\begin{align*}
||fg||_r &= \left(\int_{\mathbb{R}^n} |fg|^r \right)^{1/r}\\
&= \left(\int_{\mathbb{R}^n} |f^r g^r| \right)^{1/r}\\
&\leq \left(\left(\int_{\mathbb{R}^n} |f^r|^{p'} \right)^{1/p'} \left(\int_{\mathbb{R}^n} |g^r|^{q'} \right)^{1/q'} \right)^{1/r} &&\text{By Holder's inequality}\\
&= \left(\int_{\mathbb{R}^n} |f|^{rp'} \right)^{1/rp'} \left(\int_{\mathbb{R}^n} |g|^{rq'} \right)^{1/rq'}\\
&= \left(\int_{\mathbb{R}^n} |f|^{p} \right)^{1/p} \left(\int_{\mathbb{R}^n} |g|^{q} \right)^{1/q}\\
&= ||f||_p \cdot ||g||_q,
\end{align*}
and our proof is complete.
\end{proof}

\section*{Problem 10}
\begin{theorem}
Let $1 \leq p < r < q < + \infty$ and define $\theta \in (0, 1)$ by 
\begin{align*}
\frac{1}{r} = \frac{\theta}{p} + \frac{1 - \theta}{q}.
\end{align*}
Let $f \in L^p \cap L^q$. Prove that 
\begin{align*}
||f||_r \leq ||f||_p^\theta \cdot ||f||_q^{1 - \theta}.
\end{align*}
\end{theorem}

\subsection*{Solution}
\begin{proof}
Define $p' = \frac{p}{\theta}$ and $q' = \frac{q}{1 - \theta}$. Then, we have
\begin{align*}
\frac{1}{r} = \frac{1}{p'} + \frac{1}{q'}.
\end{align*}
Utilizing the results of the previous problem, we have
\begin{align*}
||f||_r &= ||f^\theta f^{1 - \theta}||_r\\
&\leq ||f^\theta||_{p'} \cdot ||f^{1- \theta}||_{q'}\\
&= \left(\int_{\mathbb{R}^n} |f^\theta|^{p'} \right)^{1/p'} \left(\int_{\mathbb{R}^n} |f^{1 - \theta}|^{q'} \right)^{1/q'}\\
&= \left(\int_{\mathbb{R}^n} |f|^{\theta p'} \right)^{1/p'} \left(\int_{\mathbb{R}^n} |f|^{(1 - \theta)q'} \right)^{1/q'}\\
&= \left(\int_{\mathbb{R}^n} |f|^{p} \right)^{\theta/p} \left(\int_{\mathbb{R}^n} |f|^{q} \right)^{(1-\theta)/q}\\
&= ||f||_p^\theta \cdot ||f||_q^{1 - \theta},
\end{align*}
and our proof is complete.
\end{proof}

\section*{Problem 11}
\begin{theorem}
Let $0 < r < \infty$ and $f \in L^r(\mathbb{R}^n) \cap L^\infty(\mathbb{R}^n)$. Then the following statements are true:
\proofpart{(i)}{} For all $p \in (r, \infty)$, 
\begin{align*}
||f||_p \leq ||f||_r^{r/p} \cdot ||f||_\infty^{1 - r/p}.
\end{align*}

\proofpart{(ii)}{}
\begin{align*}
\lim_{p \to \infty} ||f||_p = ||f||_\infty.
\end{align*}
\end{theorem}

\subsection*{Solution}
\begin{proof}
\proofpart{(i)}{} We have
\begin{align*}
||f||_p &= ||f^{r/p} f^{1 - r/p}||_p\\
&= \left( \int_{\mathbb{R}^n} |f^{r/p} f^{1 - r/p}|^p dx \right)^{1/p}\\
&= \left( \int_{\mathbb{R}^n} |f^{r} f^{p - r}| dx \right)^{1/p}\\
&\leq \left(||f^{p - r}||_\infty \cdot \int_{\mathbb{R}^n} |f^{r} | dx \right)^{1/p} &&\text{Holder's Inequality}\\
&= \left(||f||_\infty^{p - r} \cdot \int_{\mathbb{R}^n} |f|^{r} dx \right)^{1/p}\\
&= \left(||f||_\infty^{p - r} \cdot ||f||_r^r \right)^{1/p}\\
&= ||f||_r^{r/p} \cdot ||f||_\infty^{1 - r/p},
\end{align*}
as desired.

\proofpart{(ii)}{}
Using the results from the previous part, we have
\begin{align*}
\lim_{p \to \infty} ||f||_p &\leq \lim_{p \to \infty} ||f||_r^{r/p} \cdot ||f||_\infty^{1 - r/p}\\
&= ||f||_r^{0} \cdot ||f||_\infty^{1 - 0}\\
&= ||f||_\infty.
\end{align*}
Thus, all we must do to complete the proof is show that $\lim_{p \to \infty} ||f||_p \geq ||f||_\infty$ (but I've not figured this part out yet).
\end{proof}

\end{document}