\documentclass[10pt,a4paper]{article}
\usepackage[utf8]{inputenc}
\usepackage[a4paper,%
            left=.75in,right=.75in,top=1in,bottom=1in]{geometry}
\setlength{\headsep}{0.25in}

\usepackage{amsthm}

\usepackage{graphicx}
\usepackage{pgfplots}
            
\usepackage[english]{babel}

\theoremstyle{theorem}
\newtheorem{theorem}{Theorem}
\newtheorem{lemma}{Lemma}
\newtheorem{corollary}{Corollary}
\newtheorem{case}{Case}

\usepackage{amsthm}
\usepackage{lipsum}
\usepackage{tikz}

\makeatletter
\newcommand{\proofpart}[2]{%
  \par
  \addvspace{\medskipamount}%
  \noindent\emph{Part #1: #2}\par\nobreak
  \addvspace{\smallskipamount}%
  \@afterheading
}
\makeatother

\newcommand\restr[2]{{% we make the whole thing an ordinary symbol
  \left.\kern-\nulldelimiterspace % automatically resize the bar with \right
  #1 % the function
  \vphantom{\big|} % pretend it's a little taller at normal size
  \right|_{#2} % this is the delimiter
  }}

\theoremstyle{definition}
\newtheorem{definition}{Definition}
\newtheorem{remark}{Remark}

\usepackage{mathtools}
\DeclarePairedDelimiter\bra{\langle}{\rvert}
\DeclarePairedDelimiter\ket{\lvert}{\rangle}
\DeclarePairedDelimiterX\braket[2]{\langle}{\rangle}{#1 \delimsize\vert #2}

\usepackage{amsmath}
\usepackage{amsfonts}
\usepackage{amssymb}
\usepackage{fancyhdr}
\usepackage{tkz-euclide}

\DeclareMathOperator{\interior}{int}

\newcommand{\Tau}{\mathcal{T}}

\newenvironment{amatrix}[1]{%
  \left(\begin{array}{@{}*{#1}{c}|c@{}}
}{%
  \end{array}\right)
}

\usepackage{calligra}
\DeclareMathAlphabet{\mathcalligra}{T1}{calligra}{m}{n}
\DeclareFontShape{T1}{calligra}{m}{n}{<->s*[2.2]callig15}{}

\newcommand{\scripty}[1]{\ensuremath{\mathcalligra{#1}}}

\pagestyle{fancy}
\author{Jeremiah Givens}
\newcommand{\subject}{Real Analysis II}
\newcommand{\Date}{9/2/2021} 
\makeatletter
\rhead{{\small\@author}}
\lhead{{\small\subject}}
\chead{{\large Homework 5}}
\cfoot{}
\rfoot{\thepage}
\lfoot{\today}

\renewcommand{\theequation}{\arabic{equation}}

\begin{document}
\section*{Problem 1}
Use Holder's inequality to show 
\begin{align*}
\int_0^1 \sqrt{x} (1 - x)^{-1/3} dx \leq \frac{2}{5^{1/3}}.
\end{align*}

\subsection*{Solution}
Using Holder's inequality, let $p = 3$, and let $p' = 3/2$. Then, we have
\begin{align*}
\int_0^1 \sqrt{x} (1 - x)^{-1/3} dx &\leq \left( \int_0^1 \sqrt{x}^{3} dx \right)^{1/3} \left( \int_0^1 ((1 - x)^{-1/3})^{3/2} dx \right)^{2/3}\\
&= \left( \int_0^1 x^{3/2} dx \right)^{1/3} \left( \int_0^1 u^{-1/2} du \right)^{2/3}\\
&= \left( \frac{2}{5} \right)^{1/3}  2^{2/3}\\
&=  \frac{2}{5^{1/3}},
\end{align*}
as desired.

\section*{Problem 2}
Let $E \subseteq \mathbb{R}^n$ be measurable with $|E| = 1$. Let $h \geq 0$ be measurable on $E$. Let $A = \int_E h dx$. Show that 
\begin{align*}
\sqrt{1 + A^2} \leq \int_E \sqrt{1 + h^2} dx \leq 1 + A.
\end{align*}

\subsection*{Solution}
Show the first inequality here!!!!

Now, we have
\begin{align*}
\int_E \sqrt{1 + h^2} dx &\leq \int_E \sqrt{1 + h^2 + 2h} dx\\
&= \int_E \sqrt{(1 + h)^2} dx\\
&= \int_E 1 + h dx\\
&= 1 + A.
\end{align*}

\section*{Problem 3}
Find all nonnegative functions $g \in L^3(0, 1)$ such that 
\begin{align*}
\left( \int_0^1 x g(x)dx \right)^3 = \frac{4}{25} \int_0^1 g^3(x)dx
\end{align*}

\subsection*{Solution}
Using Holders inequality, we can see that if $p = \frac{3}{2}$, and $p' = 3$, then
\begin{align*}
\int_0^1 x g(x)dx &\leq \left(\int_0^1 x^{3/2} dx \right)^{2/3} \left(\int_0^1  g^3(x)dx \right)^{1/3}\\
&= \left( \frac{2}{5} \right)^{2/3} \left(\int_0^1  g^3(x)dx \right)^{1/3}\\
&= \left( \frac{4}{25} \int_0^1  g^3(x)dx \right)^{1/3}\\
\left( \int_0^1 x g(x)dx \right)^3 &= \frac{4}{25} \int_0^1 g^3(x)dx.
\end{align*}
Now, as we proved in class, the equality holds if and only if $\alpha x^{3/2} = g^3(x)$ almost everywhere for some real $\alpha$. Thus, we have
\begin{align*}
g(x) = \alpha x^\frac{1}{2}
\end{align*}
for some nonnegative real number $\alpha$ and for almost every $x$.
\end{document}