\documentclass[10pt,a4paper]{article}
\usepackage[utf8]{inputenc}
\usepackage[a4paper,%
            left=.75in,right=.75in,top=1in,bottom=1in]{geometry}
\setlength{\headsep}{0.25in}

\usepackage{amsthm}

\usepackage{graphicx}
\usepackage{pgfplots}
            
\usepackage[english]{babel}

\theoremstyle{theorem}
\newtheorem{theorem}{Theorem}
\newtheorem{lemma}{Lemma}
\newtheorem{corollary}{Corollary}
\newtheorem{case}{Case}

\usepackage{amsthm}
\usepackage{lipsum}
\usepackage{tikz}

\makeatletter
\newcommand{\proofpart}[2]{%
  \par
  \addvspace{\medskipamount}%
  \noindent\emph{Part #1: #2}\par\nobreak
  \addvspace{\smallskipamount}%
  \@afterheading
}
\makeatother

\newcommand\restr[2]{{% we make the whole thing an ordinary symbol
  \left.\kern-\nulldelimiterspace % automatically resize the bar with \right
  #1 % the function
  \vphantom{\big|} % pretend it's a little taller at normal size
  \right|_{#2} % this is the delimiter
  }}

\theoremstyle{definition}
\newtheorem{definition}{Definition}
\newtheorem{remark}{Remark}

\usepackage{mathtools}
\DeclarePairedDelimiter\bra{\langle}{\rvert}
\DeclarePairedDelimiter\ket{\lvert}{\rangle}
\DeclarePairedDelimiterX\braket[2]{\langle}{\rangle}{#1 \delimsize\vert #2}

\usepackage{amsmath}
\usepackage{amsfonts}
\usepackage{amssymb}
\usepackage{fancyhdr}
\usepackage{tkz-euclide}

\DeclareMathOperator{\interior}{int}

\newcommand{\Tau}{\mathcal{T}}

\newenvironment{amatrix}[1]{%
  \left(\begin{array}{@{}*{#1}{c}|c@{}}
}{%
  \end{array}\right)
}

\usepackage{calligra}
\DeclareMathAlphabet{\mathcalligra}{T1}{calligra}{m}{n}
\DeclareFontShape{T1}{calligra}{m}{n}{<->s*[2.2]callig15}{}

\newcommand{\scripty}[1]{\ensuremath{\mathcalligra{#1}}}

\pagestyle{fancy}
\author{Jeremiah Givens}
\newcommand{\subject}{Real Analysis II}
\newcommand{\Date}{9/2/2021} 
\makeatletter
\rhead{{\small\@author}}
\lhead{{\small\subject}}
\chead{{\large Homework 4}}
\cfoot{}
\rfoot{\thepage}
\lfoot{\today}

\renewcommand{\theequation}{\arabic{equation}}

\begin{document}
\section*{Problem 1}
\begin{theorem}
Let $f:(a,b) \to \mathbb{R}$ be such that for all $x_0 \in (a, b)$, there is a support line
\begin{align*}
l_{x_0}(x) = f(x_0) + m(x - x_0)
\end{align*}
for some $m \in \mathbb{R}$ such that 
\begin{align*}
f(x) \geq l_{x_0}(x)
\end{align*}
for all $x \in (a, b)$. Then $f$ is convex on $(a, b)$.
\end{theorem}

\subsection*{Solution}
\begin{proof}
Suppose that $f$ is not convex on $(a, b)$. Then, for some $[x_1, x_2] \subset (a, b)$, there exists an $x_0 \in [x_1, x_2]$ such that $f(x_0) > L(x_0)$, where $L$ is the straight line such that $L(x_1) = f(x_1)$ and $L(x_2) = f(x_2)$. We have $l_{x_0}(x_1) \leq f(x_1) = L(x_1)$ and $l_{x_0}(x_2) \leq f(x_2) = L(x_2)$. Since $L$ and $l_{x_0}$ are both straight lines, we have that $l_{x_0}(x) \leq L(x)$ for all $x \in [x_1, x_2]$. However, we have
\begin{align*}
l_{x_0}(x_0) = f(x_0) + m(x_0 - x_0) = f(x_0) > L(x_0),
\end{align*}
and we have reached a contradiction. Thus, we have proven that $f$ is convex.
\end{proof}

\section*{Problem 2}
\begin{theorem}
Let $f:[a, b] \to \mathbb{R}$ be absolutely continuous and $f'(x)$ be increasing except on a zero measure subset of $[a, b]$. Then $f$ is convex on $[a, b]$.
\end{theorem}

\subsection*{Solution}
We will start by proving a lemma which will aid in the proof of this theorem:
\begin{lemma}
Let $f:[a, b] \to \mathbb{R}$ be continuous and satisfy the midpoint convexity
\begin{align*}
f \left(\frac{x_1 + x_2}{2} \right) \leq \frac{f(x_1) + f(x_2)}{2}.
\end{align*}
for any $x_1, x_2 \in [a, b]$. Then $f$ is convex on $[a, b]$.
\end{lemma}
\begin{proof}
We will first prove, that for any $n \in \mathbb{N}$, if $x_1,...,x_{2^n} \in [a, b]$, then
\begin{align*}
f\left( \frac{x_1 + ... + x_{2^n}}{2^n} \right) \leq \frac{f(x_1) + ... + f(x_{2^n})}{2^n}.
\end{align*}
To prove this, we note that the base case ($n=1$) is true by midpoint convexity of $f$. Now, suppose for some $n \in \mathbb{N}$ that if $x_1,...,x_{2^n} \in [a, b]$, then
\begin{align*}
f\left( \frac{x_1 + ... + x_{2^n}}{2^n} \right) \leq \frac{f(x_1) + ... + f(x_{2^n})}{2^n}.
\end{align*}
Let $x_1,...,x_{2^{n+1}} \in [a, b]$. Then, we have
\begin{align*}
f\left( \frac{x_1 + ... + x_{2^{n+1}}}{2^{n+1}} \right) &= f\left( \frac{\frac{x_1 + ... + x_{2^n}}{2^n} + \frac{x_{2^n + 1} + ... + x_{2^{n+1}}}{2^n}}{2} \right)\\
&\leq \frac{f\left( \frac{x_1 + ... + x_{2^n}}{2^n}\right) + f \left( \frac{x_{2^n + 1} + ... + x_{2^{n+1}}}{2^n} \right)}{2}\\
&= \frac{f(x_1) + ... + f(x_{2^{n+1}})}{2^{n+1}},
\end{align*}
and our induction is complete.

From elementary analysis, we have that rational numbers of the form $\frac{m}{2^n}$ with $1 \leq m \leq 2^n$ are dense in $[0, 1]$. Let $[a_1, b_2] \subseteq [a, b]$. Fix some $n \in \mathbb{N}$ and some $1 \leq m \leq 2^n$. Setting $x_i = a_i$ for $1 \leq i \leq m$, and $x_i = b_i$ for $m+1 \leq i \leq 2^n$, we see from the above argument that
\begin{align*}
f\left( \frac{x_1 + ... + x_{2^n}}{2^n} \right) &\leq \frac{f(x_1) + ... + f(x_{2^n})}{2^n}\\
&= \frac{mf(a_i) + (2^n - m)f(b_i)}{2^n}\\
&= \frac{m}{2^n}f(a_i) + (1 - \frac{m}{2^n})f(b_i).
\end{align*}
Finally, let $\theta \in [0, 1]$ be a real number. For each $n \in \mathbb{N}$, define $1 \leq m_n \leq 2^n$ to be the largest number such that $\frac{m_n}{2^n} \geq \theta$. Then, $\frac{m_n}{2^n} \to \theta$, and it follows from the continuity of $f$ that 
\begin{align*}
f(\theta a_i + (1 - \theta)b_i) \leq \theta f(a_i) + (1 - \theta)f(b_i),
\end{align*}
and we have shown that $f$ is convex.
\end{proof}
Now we are ready to prove the main theorem:
\begin{proof}
Let $[x_1, x_2] \subseteq [a, b]$. By the above lemma, it will suffice to show that 
\begin{align*}
f \left(\frac{x_1 + x_2}{2} \right) \leq \frac{f(x_1) + f(x_2)}{2}.
\end{align*} 
Since $f$ is absolutely continuous, we have
\begin{align*}
f \left(\frac{x_1 + x_2}{2} \right) - f(x_1) = \int_{x_1}^{\frac{x_1 + x_2}{2}} f'(x)dx,
\end{align*}
and 
\begin{align*}
f(x_2) - f \left(\frac{x_1 + x_2}{2} \right) = \int_{\frac{x_1 + x_2}{2}}^{x_2} f'(x)dx.
\end{align*}
Now, since $f'$ is increasing almost everywhere, we have
\begin{align*}
\int_{x_1}^{\frac{x_1 + x_2}{2}} f'(x)dx &\leq \int_{\frac{x_1 + x_2}{2}}^{x_2} f'(x)dx.
\end{align*}
With this, we have
\begin{align*}
f \left(\frac{x_1 + x_2}{2} \right) - f(x_1) &\leq f(x_2) - f \left(\frac{x_1 + x_2}{2} \right)\\
f \left(\frac{x_1 + x_2}{2} \right) &\leq \frac{f(x_1) + f(x_2)}{2}
\end{align*}
and we have proven that $f$ is midpoint-convex, and therefore convex.
\end{proof}

\section*{Problem 3}
\begin{theorem}
Let $f:[a, b] \to \mathbb{R}$ and let $E \subseteq [a, b]$. Assume that $f'(x)$ exists with a finite value for any $x \in E$. Then,
\begin{align*}
|f(E)|_e \leq \int_E |f'(x)|dx.
\end{align*}
\end{theorem}

\begin{proof}
We start by defining the set of values of $f$ on $E$ as $f(E) = \{f(x) : x \in E\}$. We then define the outer measure of this set as $|f(E)|_e = \inf{\sum_{i=1}^\infty |I_i| : f(E) \subseteq \bigcup_{i=1}^\infty I_i}$, where the $I_i$ are intervals.

Now, let $A \subseteq E$ be a subset of $E$. By the mean value theorem, for any $x, y \in A$, there exists a point $c \in A$ such that $f(x) - f(y) = f'(c)(x - y)$. Therefore, taking absolute values and using the triangle inequality, we have $|f(x) - f(y)| = |f'(c)||x - y| \leq \sup_{x \in A} |f'(x)||x - y|$.

We can now cover $f(A)$ with intervals of length at most $\sup_{x \in A} |f'(x)|$ as follows: for each $f(x) \in f(A)$, choose an interval $I_x$ centered at $f(x)$ with length $\epsilon_x$, where $\epsilon_x$ is such that $2\epsilon_x \sup_{x \in A} |f'(x)| \geq \epsilon_x |A|e$ (such an $\epsilon_x$ exists since $\sup{x \in A} |f'(x)|$ is finite and $|A|e$ is also finite). Then, we have $f(A) \subseteq \bigcup{x \in A} I_x$ and $\sum_{x \in A} l(I_x) \leq 2\sup_{x \in A} |f'(x)|\sum_{x \in A} \epsilon_x \leq 2\sup_{x \in A} |f'(x)|\cdot |A|_e$.

By the definition of the outer measure of $f(E)$, we have $|f(E)|e \leq \sum{i=1}^\infty l(I_i)$ for any covering of $f(E)$ by intervals $I_i$, including the one above. Therefore, we obtain $|f(A)|e \leq \sum{x \in A} l(I_x) \leq 2\sup_{x \in A} |f'(x)|\cdot |A|_e$. Taking the infimum over all coverings of $f(E)$ by intervals, we have $|f(A)|e \leq \inf{2\sup{x \in A} |f'(x)|\cdot |A|_e}$.
\end{proof}

\end{document}