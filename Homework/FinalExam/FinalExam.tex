\documentclass[10pt,a4paper]{article}
\usepackage[utf8]{inputenc}
\usepackage[a4paper,%
            left=.75in,right=.75in,top=1in,bottom=1in]{geometry}
\setlength{\headsep}{0.25in}

\usepackage{amsthm}
\usepackage{amsmath}

\usepackage{graphicx}
\usepackage{pgfplots}
            
\usepackage[english]{babel}

\theoremstyle{theorem}
\newtheorem{theorem}{Theorem}
\newtheorem{lemma}{Lemma}
\newtheorem{corollary}{Corollary}
\newtheorem{case}{Case}

\usepackage{amsthm}
\usepackage{lipsum}

\makeatletter
\newcommand{\proofpart}[2]{%
  \par
  \addvspace{\medskipamount}%
  \noindent\emph{Part #1: #2}\par\nobreak
  \addvspace{\smallskipamount}%
  \@afterheading
}
\makeatother

\newcommand\restr[2]{{% we make the whole thing an ordinary symbol
  \left.\kern-\nulldelimiterspace % automatically resize the bar with \right
  #1 % the function
  \vphantom{\big|} % pretend it's a little taller at normal size
  \right|_{#2} % this is the delimiter
  }}

\theoremstyle{definition}
\newtheorem{definition}{Definition}
\newtheorem{remark}{Remark}

\usepackage{mathtools}
\DeclarePairedDelimiter\bra{\langle}{\rvert}
\DeclarePairedDelimiter\ket{\lvert}{\rangle}
\DeclarePairedDelimiterX\braket[2]{\langle}{\rangle}{#1 \delimsize\vert #2}

\usepackage{amsmath}
\usepackage{amsfonts}
\usepackage{amssymb}
\usepackage{fancyhdr}

\DeclareMathOperator{\interior}{int}

\newcommand{\Tau}{\mathcal{T}}

\newenvironment{amatrix}[1]{%
  \left(\begin{array}{@{}*{#1}{c}|c@{}}
}{%
  \end{array}\right)
}

\usepackage{calligra}
\DeclareMathAlphabet{\mathcalligra}{T1}{calligra}{m}{n}
\DeclareFontShape{T1}{calligra}{m}{n}{<->s*[2.2]callig15}{}

\newcommand{\scripty}[1]{\ensuremath{\mathcalligra{#1}}}

\pagestyle{fancy}
\author{Jeremiah Givens}
\newcommand{\subject}{Real Analysis I}
\newcommand{\Date}{9/2/2021} 
\makeatletter
\rhead{{\small\@author}}
\lhead{{\small\subject}}
\chead{{\large Final Exam}}
\cfoot{}
\rfoot{\thepage}
\lfoot{\today}

\renewcommand{\theequation}{\arabic{equation}}

\begin{document}
\section*{Problem 1}
\begin{theorem}
Let $f$ be measurable on a measurable set $E$ in $\mathbb{R}^n$. Define an extension of $f$ on $\mathbb{R}$ by 
\begin{align*}
\bar{f}(x) = \begin{cases} 
      f(x) & x \in E \\
     0 & \text{if } x\in \mathbb{R}^n - E \\
\end{cases}
\end{align*}
Then
\proofpart{(i)}{} $\bar{f}$ is measurable on $\mathbb{R}^n$, and

\proofpart{(i)}{} If $f \in L(E)$, then $\bar{f} \in L(\mathbb{R}^n)$ and
\begin{align*}
\int_{\mathbb{R}^n} \bar{f} = \int_E f.
\end{align*}
\end{theorem}

\subsection*{Solution}
\begin{proof}
\proofpart{(i)}{} Let $a \in \mathbb{R}$.  We have
\begin{align*}
\{\bar{f} > a \} &= \{x \in \mathbb{R}^n| \bar{f}(x) > a \}\\
&= \{x \in E| f(x) > a \} \cup \{x \in E^c| 0 > a \}.
\end{align*}
The fist set in the union above is measurable, since $f$ is measurable. The second set is either the empty set (if $a \geq 0$), or is $E^c$ if $a < 0$.  Since the empty set and the complement of a measurable set are both measurable, and the union of two measurable sets is measurable, we have shown that $\{\bar{f} > a \}$ is measurable.

\proofpart{(ii)}{} Suppose $f \in L(E)$.  By theorem 5.24, we have
\begin{align*}
\int_{\mathbb{R}^n} \bar{f} &= \int_{\mathbb{R}^n - E} \bar{f} + \int_E \bar{f}\\
&= \int_{\mathbb{R}^n - E} (0) + \int_E f\\
&=  \int_E f.
\end{align*}
Thus, since $f \in L(E)$, $ \int_E f$ is finite, and we can conclude that $\int_{\mathbb{R}^n} \bar{f}$ is finite. Thus, we have shown that $f \in L(\mathbb{R}^n)$.
\end{proof}

\section*{Problem 2}
\begin{theorem}
Let $f_k$ ($k = 1,2,...$) and $f$ be measurable and finite a.e. on $\mathbb{R}^n$ such that $\int_{\mathbb{R}^n} |f_k - f| dx \to 0$ as $k \to \infty$. Then $f_k$ converges to $f$ in measure.
\end{theorem}

\subsection*{Solution}
\begin{proof}
Suppose that $f_k$ does not converge to $f$ in measure. Then, there exists some $\epsilon > 0$ such that 
\begin{align*}
\lim_{k \to \infty} |\{x \in \mathbb{R}^n : |f(x) - f_k(x)| > \epsilon\}| \not= 0.
\end{align*}
Thus, there exists some $\delta > 0$, such that for all $K \in \mathbb{N}$, there exists a $k \geq K$ such that 
\begin{align*}
\lim_{k \to \infty} |\{x \in \mathbb{R}^n : |f(x) - f_k(x)| > \epsilon\}| > \delta.
\end{align*}
Define $E = \{x \in \mathbb{R}^n : |f(x) - f_k(x)| > \epsilon\}$. Then, we have
\begin{align*}
\int_{\mathbb{R}^n} |f_k - f| dx &= \int_{\mathbb{R}^n - E} |f_k - f| dx + \int_{E} |f_k - f| dx && \text{By theorem 5.7}\\
&\geq \int_{E} |f_k - f| dx &&\text{Since } |f_k - f| \geq 0\\
&\geq \int_{E} \epsilon dx && \text{By theorem 5.5}\\
&\geq |E| \epsilon &&\text{By corollary 5.5}\\
&\geq \delta \epsilon
\end{align*}
Thus, for every $K \in \mathbb{N}$, there exists a $k \geq K$ such that $\int_{\mathbb{R}^n} |f_k - f| dx \geq \delta \epsilon$, and we have shown that $\int_{\mathbb{R}^n} |f_k - f| dx$ does not go to $0$. With this, we have proven the contrapositive, and our proof is complete.
\end{proof}
\end{document}