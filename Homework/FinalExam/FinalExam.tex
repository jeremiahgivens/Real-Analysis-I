\documentclass[10pt,a4paper]{article}
\usepackage[utf8]{inputenc}
\usepackage[a4paper,%
            left=.75in,right=.75in,top=1in,bottom=1in]{geometry}
\setlength{\headsep}{0.25in}

\usepackage{amsthm}
\usepackage{amsmath}

\usepackage{graphicx}
\usepackage{pgfplots}
            
\usepackage[english]{babel}

\theoremstyle{theorem}
\newtheorem{theorem}{Theorem}
\newtheorem{lemma}{Lemma}
\newtheorem{corollary}{Corollary}
\newtheorem{case}{Case}

\usepackage{amsthm}
\usepackage{lipsum}

\makeatletter
\newcommand{\proofpart}[2]{%
  \par
  \addvspace{\medskipamount}%
  \noindent\emph{Part #1: #2}\par\nobreak
  \addvspace{\smallskipamount}%
  \@afterheading
}
\makeatother

\newcommand\restr[2]{{% we make the whole thing an ordinary symbol
  \left.\kern-\nulldelimiterspace % automatically resize the bar with \right
  #1 % the function
  \vphantom{\big|} % pretend it's a little taller at normal size
  \right|_{#2} % this is the delimiter
  }}

\theoremstyle{definition}
\newtheorem{definition}{Definition}
\newtheorem{remark}{Remark}

\usepackage{mathtools}
\DeclarePairedDelimiter\bra{\langle}{\rvert}
\DeclarePairedDelimiter\ket{\lvert}{\rangle}
\DeclarePairedDelimiterX\braket[2]{\langle}{\rangle}{#1 \delimsize\vert #2}

\usepackage{amsmath}
\usepackage{amsfonts}
\usepackage{amssymb}
\usepackage{fancyhdr}

\DeclareMathOperator{\interior}{int}

\newcommand{\Tau}{\mathcal{T}}

\newenvironment{amatrix}[1]{%
  \left(\begin{array}{@{}*{#1}{c}|c@{}}
}{%
  \end{array}\right)
}

\usepackage{calligra}
\DeclareMathAlphabet{\mathcalligra}{T1}{calligra}{m}{n}
\DeclareFontShape{T1}{calligra}{m}{n}{<->s*[2.2]callig15}{}

\newcommand{\scripty}[1]{\ensuremath{\mathcalligra{#1}}}

\pagestyle{fancy}
\author{Jeremiah Givens}
\newcommand{\subject}{Real Analysis I}
\newcommand{\Date}{9/2/2021} 
\makeatletter
\rhead{{\small\@author}}
\lhead{{\small\subject}}
\chead{{\large Final Exam}}
\cfoot{}
\rfoot{\thepage}
\lfoot{\today}

\renewcommand{\theequation}{\arabic{equation}}

\begin{document}
\section*{Problem 1}
\begin{theorem}
Let $f$ be measurable on a measurable set $E$ in $\mathbb{R}^n$. Define an extension of $f$ on $\mathbb{R}$ by 
\begin{align*}
\bar{f}(x) = \begin{cases} 
      f(x) & x \in E \\
     0 & \text{if } x\in \mathbb{R}^n - E \\
\end{cases}
\end{align*}
Then
\proofpart{(i)}{} $\bar{f}$ is measurable on $\mathbb{R}^n$, and

\proofpart{(i)}{} If $f \in L(E)$, then $\bar{f} \in L(\mathbb{R}^n)$ and
\begin{align*}
\int_{\mathbb{R}^n} \bar{f} = \int_E f.
\end{align*}
\end{theorem}

\subsection*{Solution}
\begin{proof}
\proofpart{(i)}{} Let $a \in \mathbb{R}$.  We have
\begin{align*}
\{\bar{f} > a \} &= \{x \in \mathbb{R}^n| \bar{f}(x) > a \}\\
&= \{x \in E| f(x) > a \} \cup \{x \in E^c| 0 > a \}.
\end{align*}
The fist set in the union above is measurable, since $f$ is measurable. The second set is either the empty set (if $a \geq 0$), or is $E^c$ if $a < 0$.  Since the empty set and the complement of a measurable set are both measurable, and the union of two measurable sets is measurable, we have shown that $\{\bar{f} > a \}$ is measurable.

\proofpart{(ii)}{} Suppose $f \in L(E)$.  By theorem 5.24, we have
\begin{align*}
\int_{\mathbb{R}^n} \bar{f} &= \int_{\mathbb{R}^n - E} \bar{f} + \int_E \bar{f}\\
&= \int_{\mathbb{R}^n - E} (0) + \int_E f\\
&=  \int_E f.
\end{align*}
Thus, since $f \in L(E)$, $ \int_E f$ is finite, and we can conclude that $\int_{\mathbb{R}^n} \bar{f}$ is finite. Thus, we have shown that $f \in L(\mathbb{R}^n)$.
\end{proof}

\section*{Problem 2}
\begin{theorem}
Let $f_k$ ($k = 1,2,...$) and $f$ be measurable and finite a.e. on $\mathbb{R}^n$ such that $\int_{\mathbb{R}^n} |f_k - f| dx \to 0$ as $k \to \infty$. Then $f_k$ converges to $f$ in measure.
\end{theorem}

\subsection*{Solution}
\begin{proof}
Suppose that $f_k$ does not converge to $f$ in measure. Then, there exists some $\epsilon > 0$ such that 
\begin{align*}
\lim_{k \to \infty} |\{x \in \mathbb{R}^n : |f(x) - f_k(x)| > \epsilon\}| \not= 0.
\end{align*}
Thus, there exists some $\delta > 0$, such that for all $K \in \mathbb{N}$, there exists a $k \geq K$ such that 
\begin{align*}
\lim_{k \to \infty} |\{x \in \mathbb{R}^n : |f(x) - f_k(x)| > \epsilon\}| > \delta.
\end{align*}
Define $E = \{x \in \mathbb{R}^n : |f(x) - f_k(x)| > \epsilon\}$. Then, we have
\begin{align*}
\int_{\mathbb{R}^n} |f_k - f| dx &= \int_{\mathbb{R}^n - E} |f_k - f| dx + \int_{E} |f_k - f| dx && \text{By theorem 5.7}\\
&\geq \int_{E} |f_k - f| dx &&\text{Since } |f_k - f| \geq 0\\
&\geq \int_{E} \epsilon dx && \text{By theorem 5.5}\\
&\geq |E| \epsilon &&\text{By corollary 5.5}\\
&\geq \delta \epsilon
\end{align*}
Thus, for every $K \in \mathbb{N}$, there exists a $k \geq K$ such that $\int_{\mathbb{R}^n} |f_k - f| dx \geq \delta \epsilon$, and we have shown that $\int_{\mathbb{R}^n} |f_k - f| dx$ does not go to $0$. With this, we have proven the contrapositive, and our proof is complete.
\end{proof}

\section*{Problem 3}
\begin{theorem}
Let $f \in L(\mathbb{R}^n)$. Then
\begin{align*}
\lim_{k \to \infty} \int_{|x| > k} |f(x)|dx = 0, &&\text{and} && \lim_{k \to \infty} \int_{|f(x)| > k} |f(x)|dx = 0.
\end{align*}
\end{theorem}

\subsection*{Solution}
\begin{proof}
Let $\epsilon > 0$. By theorem 5.21, we have that $|f| \in L(\mathbb{R}^n)$.  By theorem 5.24, we have
\begin{align*}
\int_{\mathbb{R}^n} |f(x)|dx = \sum_{k=1}^\infty \int_{k - 1 \leq |x| < k} |f(x)|dx.
\end{align*}
Since the integral on the left is finite, and all of the terms in the summation are nonnegative, we have that there exists some $K \in \mathbb{N}$ such that 
\begin{align*}
\int_{\mathbb{R}^n} |f(x)|dx - \sum_{k=1}^K \int_{k - 1 \leq |x| < k} |f(x)|dx < \epsilon.
\end{align*}
Using theorem 5.24 once more, we have 
\begin{align*}
\int_{\mathbb{R}^n} |f(x)|dx - \sum_{k=1}^K \int_{k - 1 \leq |x| < k} |f(x)|dx = \int_{ |x| \geq K} |f(x)|dx.
\end{align*}
Using theorem 5.5 part (iii), we have
\begin{align*}
\int_{ |x| < K} |f(x)|dx &< \int_{ |x| \geq K} |f(x)|dx\\
&< \epsilon,
\end{align*}
and we have shown that 
\begin{align*}
\lim_{k \to \infty} \int_{|x| > k} |f(x)|dx = 0.
\end{align*}

For the next part, we will proceed in a nearly identical fashion. Using theorem 5.24, we have
\begin{align*}
\int_{\mathbb{R}^n} |f(x)|dx &= \sum_{k=1}^\infty \int_{k - 1 \leq |f(x)| < k} |f(x)|dx.
\end{align*}
To see that each of the sets we are integrating over in the summation, recall that $|f|$ is measurable by theorem 5.1. Thus, since $f$ is measurable,  we have that each of these sets are measurable by corollary 4.2.

Using the same argument as before, we have that there exists a $K \in \mathbb{N}$ such that 
\begin{align*}
\int_{\mathbb{R}^n} |f(x)|dx - \sum_{k=1}^K \int_{k - 1 \leq |f(x)| < k} |f(x)|dx < \epsilon.
\end{align*}
From theorem 5.24, we have
\begin{align*}
\int_{\mathbb{R}^n} |f(x)|dx - \sum_{k=1}^K \int_{k - 1 \leq |f(x)| < k} |f(x)|dx = \int_{|f(x)| \geq K} |f(x)|dx.
\end{align*}
Finally, we have
\begin{align*}
\int_{|f(x)| > K} |f(x)|dx &\leq \int_{|f(x)| \geq K} |f(x)|dx\\
&< \epsilon.
\end{align*}
With this, we have shown that
\begin{align*}
\lim_{k \to \infty} \int_{|f(x)| > k} |f(x)|dx = 0,
\end{align*}
completing the proof.
\end{proof}

\section*{Problem 4}
Give an example of a bounded continuous function $f$ on $(0, \infty)$ such that $\lim_{x \to \infty} f(x) = 0$ but $\int_{(0, \infty)} |f|^p dx = \infty$ for any $p > 0$.

\subsection*{Solution}
Consider the function $f$ on $(0, \infty)$ defined by
\begin{align*}
f(x) = \begin{cases} 
     \sum_{n=1}^\infty 2^{-n} x^{-1/n} & \text{for } x \geq 1 \\
     1 & \text{otherwise} \\
\end{cases}
\end{align*}
Since $x^{-1/n} \leq 1$ for all $n$ and all $x \in (1, \infty)$, we have for all $x\in (1, \infty)$,
\begin{align*}
\sum_{n=1}^\infty 2^{-n} x^{-1/n} &\leq  \sum_{n=1}^\infty 2^{-n}\\
&= 1.
\end{align*}
Therefore, this function is bounded. Furthermore, since this is a sum of continuous functions,  f is also continuous.  Finally, since each term goes to zero as $x \to \infty$, we have that $f \to 0$ as $x \to \infty$.

Now it just remains to show that the integral of $|f|^p$ diverges. We have
\begin{align*}
\int_{(0, \infty)} |f|^p dx &\leq \int_{(1, \infty)} |f|^p dx && \text{By theorem 5.5}\\
&= \int_{(1, \infty)} \left| \sum_{n=1}^\infty 2^{-n} x^{-1/n}  \right|^p dx\\
&\leq \int_{(1, \infty)} \sum_{n=1}^\infty \left| 2^{-n} x^{-1/n}  \right|^p dx\\
&=  \int_{(1, \infty)} \sum_{n=1}^\infty 2^{-np} x^{-p/n} dx\\
&=\sum_{n=1}^\infty  \int_{(1, \infty)} 2^{-np} x^{-p/n} dx &&\text{By 5.16}\\
\end{align*}
Now, lets examine the integral of a single term, and compute it as the limit of a Reimann integral:
\begin{align*}
\int_{(1, \infty)} 2^{-np} x^{-p/n} dx =\left. \frac{nx^\frac{n-p}{n}}{n - p} \right|_1^\infty.
\end{align*}
Thus, once $n > p$, this integral diverges, and we have shown that $\int_{(0, \infty)} |f|^p dx = \infty$ for any $p > 0$.

\section*{Problem 5}
Evaluate the limit
\begin{align*}
\lim_{n \to \infty} \int_0^\infty \frac{n x^{n - 3}}{1 + x^n}\sin \frac{x}{n} dx,
\end{align*}
and justify your answer.

\subsection*{Solution}
To the solve this problem, we will use the Lebesgue Dominated Convergence Theorem. We will break this integral up over two sets, as the integrand has different behavior on $(0, 1)$ than it does on $[1, \infty)$. Thus, let us first consider
\begin{align*}
\lim_{n \to \infty} \int_0^1 \frac{n x^{n - 3}}{1 + x^n}\sin \frac{x}{n} dx.
\end{align*}
We need to show that there exists a function $\phi \in L(0, 1)$ such that for some $K \in \mathbb{N}$,  $|f_n| \leq \phi$ for all $n \geq K$.  By simple evaluation, we see
\begin{align*}
\left|\frac{n x^{n - 3}}{1 + x^n}\sin \frac{x}{n} \right| &= \frac{n x^{n - 3}}{1 + x^n} \left| \sin \frac{x}{n} \right|\\
&\leq \frac{n x^{n - 3}}{1 + x^n} \frac{x}{n} &&\text{A result from the mean value theorem}\\
&\leq  \frac{x^{n - 2}}{1 + x^n}\\
&\leq  \frac{x^{n - 2}}{1}\\
&\leq 1 && \text{For } n > 2.
\end{align*}
With this, we are free to use the LDCT on this integral. Evalutating the limit,
\begin{align*}
\lim_{n \to \infty} \frac{n x^{n - 3}}{1 + x^n}\sin \frac{x}{n} &= \lim_{n \to \infty} n x^{n - 3} \sin \frac{x}{n}\\
&= x^{-3} \lim_{n \to \infty} n x^n \sin \frac{x}{n}\\
&= x^{-3} (\lim_{n \to \infty} n x^n )(\lim_{n \to \infty} \sin \frac{x}{n})\\
\end{align*}
Since $-1 \leq \sin \frac{x}{n} \leq 1$, it will suffice to show that $\lim_{n \to \infty} n x^n = 0$. We have
\begin{align*}
\lim_{n \to \infty} n x^n &= \lim_{n \to \infty} \frac{n}{1/x^n}\\
&= \lim_{n \to \infty} \frac{n}{x^{-n}}\\
&= \lim_{n \to \infty} \frac{1}{-x^{-n} \ln(x)} &&\text{L'Hopital's Rule}\\
&= 0.
\end{align*}
Thus, we have shown that 
\begin{align*}
\lim_{n \to \infty} \frac{n x^{n - 3}}{1 + x^n}\sin \frac{x}{n} = 0,
\end{align*}
and the LDCT tells us that 
\begin{align*}
\lim_{n \to \infty} \int_0^1 \frac{n x^{n - 3}}{1 + x^n}\sin \frac{x}{n} dx = 0.
\end{align*}

Now we consider the integral 
\begin{align*}
\lim_{n \to \infty} \int_1^\infty \frac{n x^{n - 3}}{1 + x^n}\sin \frac{x}{n} dx.
\end{align*}
Again, we must first verify that this is a candidate for the LDCT,:
\begin{align*}
\left|\frac{n x^{n - 3}}{1 + x^n}\sin \frac{x}{n} \right| &= \frac{n x^{n - 3}}{1 + x^n} \left| \sin \frac{x}{n} \right|\\
&\leq \frac{n x^{n - 3}}{1 + x^n} \frac{x}{n} &&\text{A result from the mean value theorem}\\
&= \frac{x^{n - 2}}{1 + x^n}\\
&\leq \frac{x^{n - 2}}{x^n}\\
&= x^{-2}.
\end{align*}
Thus, since $\int_1^\infty x^-2 dx = 1$,  we have found our $\phi \in L(1, \infty)$ such that $|f_n| \leq \phi$. Therefore, we are free to use the LDCT.  Evaluating the limit, we see
\begin{align*}
\lim_{n \to \infty} \frac{n x^{n - 3}}{1 + x^n}\sin \frac{x}{n} &= \lim_{n \to \infty} \frac{n x^{n - 3}}{x^n}\sin \frac{x}{n} && x^n \text{ dominates 1 in the limit}\\
&=\lim_{n \to \infty} \frac{n}{x^3}\sin \frac{x}{n}\\
&= x^{-3} \lim_{n \to \infty} \frac{\sin \frac{x}{n}}{n^{-1}}\\
&=x^{-3} \lim_{n \to \infty} \frac{x\ln(n) \cos \frac{x}{n}}{\ln(n)}&&\text{L'Hopital's Rule}\\
&= x^{-2} \lim_{n \to \infty} \cos\frac{x}{n}\\
&= x^{-2}.
\end{align*}
Putting it all together, we have
\begin{align*}
\lim_{n \to \infty} \int_0^\infty \frac{n x^{n - 3}}{1 + x^n}\sin \frac{x}{n} dx &= \lim_{n \to \infty} \int_0^1 \frac{n x^{n - 3}}{1 + x^n}\sin \frac{x}{n} dx + \lim_{n \to \infty} \int_1^\infty \frac{n x^{n - 3}}{1 + x^n}\sin \frac{x}{n} dx\\
&= \int_0^1  \lim_{n \to \infty} \frac{n x^{n - 3}}{1 + x^n}\sin \frac{x}{n} dx + \int_1^\infty  \lim_{n \to \infty} \frac{n x^{n - 3}}{1 + x^n}\sin \frac{x}{n} dx\\
&= \int_0^1 0 dx + \int_1^\infty x^-2 dx\\
&= 0 + 1\\
&= 1.
\end{align*}

\section*{Problem 6}
Let $f: \mathbb{R}^n \to [0, \infty)$ be measurable and let $0 < \alpha < 1$. Evaluate the limit 
\begin{align*}
\lim_{n \to \infty} \int_{\mathbb{R}^n} n \left( \left( 1 + \frac{f(x)}{n} \right)^ \alpha - 1 \right) dx,
\end{align*}
and justify your answer.

\subsection*{Solution}
For this problem, we will use the Monotone Convergence Theorem for Nonnegative functions. Define the sequence $\{a_n \}$ by
\begin{align*}
a_n = n \left( \left( 1 + \frac{f(x)}{n} \right)^ \alpha - 1 \right).
\end{align*}
In order to use the monotone convergence theorem, we will need to prove that this sequence is increasing. To do this, we will treat $n$ as as a continuous variable, and take the derivative of $a_n$:
\begin{align*}
\frac{d}{dn}a_n &= \left( 1 + \frac{f(x)}{n} \right)^ \alpha - 1 + n\left( 1 + \frac{f(x)}{n} \right)^ {\alpha - 1} f(x) \ln(n)\\
&\geq n\left( 1 + \frac{f(x)}{n} \right)^ {\alpha - 1} f(x) \ln(n) && \text{Since } \left( 1 + \frac{f(x)}{n} \right)^ \alpha \geq 1\\
&\geq 0 &&\text{Since all terms are nonnegative}
\end{align*}
Thus, $a_n$ is an increasing sequence. Now, we just need to find what this sequence converges to, and we can compute the integral. We have
\begin{align*}
\lim_{n \to \infty} n \left( \left( 1 + \frac{f(x)}{n} \right)^ \alpha - 1 \right) &= \lim_{n \to \infty} \frac{\left( 1 + \frac{f(x)}{n} \right)^ \alpha - 1}{n^{-1}}\\
&= \lim_{n \to \infty} \frac{\left( 1 + \frac{f(x)}{n} \right)^{ \alpha - 1} \alpha f(x) \ln(n)}{\ln(n)} &&\text{L'Hopital's Rule}\\
&=  \lim_{n \to \infty} \left( 1 + \frac{f(x)}{n} \right)^{ \alpha - 1} \alpha f(x)\\
&= \alpha f(x).
\end{align*}
Thus, the Monotone Convergence Theorem for Nonnegative Functions leads us to conclude that
\begin{align*}
\lim_{n \to \infty} \int_{\mathbb{R}^n} n \left( \left( 1 + \frac{f(x)}{n} \right)^ \alpha - 1 \right) dx = \alpha \int_{\mathbb{R}^n} f(x)dx
\end{align*}
\end{document}