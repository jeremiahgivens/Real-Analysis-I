\documentclass[10pt,a4paper]{article}
\usepackage[utf8]{inputenc}
\usepackage[a4paper,%
            left=.75in,right=.75in,top=1in,bottom=1in]{geometry}
\setlength{\headsep}{0.25in}

\usepackage{amsthm}

\usepackage{graphicx}
\usepackage{pgfplots}
            
\usepackage[english]{babel}

\theoremstyle{theorem}
\newtheorem{theorem}{Theorem}
\newtheorem{lemma}{Lemma}
\newtheorem{corollary}{Corollary}
\newtheorem{case}{Case}

\usepackage{amsthm}
\usepackage{lipsum}

\makeatletter
\newcommand{\proofpart}[2]{%
  \par
  \addvspace{\medskipamount}%
  \noindent\emph{Part #1: #2}\par\nobreak
  \addvspace{\smallskipamount}%
  \@afterheading
}
\makeatother

\newcommand\restr[2]{{% we make the whole thing an ordinary symbol
  \left.\kern-\nulldelimiterspace % automatically resize the bar with \right
  #1 % the function
  \vphantom{\big|} % pretend it's a little taller at normal size
  \right|_{#2} % this is the delimiter
  }}

\theoremstyle{definition}
\newtheorem{definition}{Definition}
\newtheorem{remark}{Remark}

\usepackage{mathtools}
\DeclarePairedDelimiter\bra{\langle}{\rvert}
\DeclarePairedDelimiter\ket{\lvert}{\rangle}
\DeclarePairedDelimiterX\braket[2]{\langle}{\rangle}{#1 \delimsize\vert #2}

\usepackage{amsmath}
\usepackage{amsfonts}
\usepackage{amssymb}
\usepackage{fancyhdr}

\DeclareMathOperator{\interior}{int}

\newcommand{\Tau}{\mathcal{T}}

\newenvironment{amatrix}[1]{%
  \left(\begin{array}{@{}*{#1}{c}|c@{}}
}{%
  \end{array}\right)
}

\usepackage{calligra}
\DeclareMathAlphabet{\mathcalligra}{T1}{calligra}{m}{n}
\DeclareFontShape{T1}{calligra}{m}{n}{<->s*[2.2]callig15}{}

\newcommand{\scripty}[1]{\ensuremath{\mathcalligra{#1}}}

\pagestyle{fancy}
\author{Jeremiah Givens}
\newcommand{\subject}{Real Analysis I}
\newcommand{\Date}{9/2/2021} 
\makeatletter
\rhead{{\small\@author}}
\lhead{{\small\subject}}
\chead{{\large Homework 3}}
\cfoot{}
\rfoot{\thepage}
\lfoot{\today}

\renewcommand{\theequation}{\arabic{equation}}

\begin{document}
\section*{Problem 1}
We know that if $f$ is measurable, then for every $a \in \mathbb{R}^1$, the set $f^{-1}(\{a\})$ is measurable. Use the following function $f$ to show that the converse of this result is not true, where $f: \mathbb{R}^1 \to \mathbb{R}^1$ is defined by
\[   f = \left\{
\begin{array}{ll}
      e^x & x \in E \\
      -e^x & x \in E^c \\
\end{array} 
\right. \]
where $E \in \mathbb{R}^1$ is a nonmeasurable set.

\subsection*{Solution}
\begin{proof}
We have that $e^x$ is a positive function. Thus, $-e^x$ is a negative function. Thus, it follows that
\begin{align*}
\{f > 0\} &= E.
\end{align*}
Since $E$ is nonmeasurable, this tells us that $\{f > 0\}$ is nonmeasurable. With this, we have shown that $f$ is not measurable.
\end{proof}

\section*{Problem 2}
\begin{theorem}
Let $E \in \mathbb{R}^1$ be measurable. Show that if $f: E \to [-\infty, \infty ]$ is increasing, then $f$ is measurable.
\end{theorem}
\begin{proof}
Let $a \in \mathbb{R}$. We have
\begin{align*}
\{f > a \} &= \{x: E| f(x) > a \}.
\end{align*}
Now, suppose there is no lower bound to $\{f > a \}$. Then,
\begin{align*}
(\forall x \in E)(\exists x_a \in \{f > a \})(x_a < x) &\implies (\forall x \in E)(\exists x_a \in \{f > a \})(f(x_a) \leq  f(x)) && \text{Since } f \text{ is increasing}\\
&\implies (\forall x \in E)(f(x) > a)\\
&\implies \{f > a \} = E\\
&\implies \{f > a \} \text{ is measurable.}
\end{align*}
Suppose then, that $\{f > a \}$ has a lower bound. Then, by a fundamental properties of the real numbers, $\{f > a \}$ has a greatest lower bound.
\end{proof}
\end{document}