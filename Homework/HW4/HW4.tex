\documentclass[10pt,a4paper]{article}
\usepackage[utf8]{inputenc}
\usepackage[a4paper,%
            left=.75in,right=.75in,top=1in,bottom=1in]{geometry}
\setlength{\headsep}{0.25in}

\usepackage{amsthm}

\usepackage{graphicx}
\usepackage{pgfplots}
            
\usepackage[english]{babel}

\theoremstyle{theorem}
\newtheorem{theorem}{Theorem}
\newtheorem{lemma}{Lemma}
\newtheorem{corollary}{Corollary}
\newtheorem{case}{Case}

\usepackage{amsthm}
\usepackage{lipsum}

\makeatletter
\newcommand{\proofpart}[2]{%
  \par
  \addvspace{\medskipamount}%
  \noindent\emph{Part #1: #2}\par\nobreak
  \addvspace{\smallskipamount}%
  \@afterheading
}
\makeatother

\newcommand\restr[2]{{% we make the whole thing an ordinary symbol
  \left.\kern-\nulldelimiterspace % automatically resize the bar with \right
  #1 % the function
  \vphantom{\big|} % pretend it's a little taller at normal size
  \right|_{#2} % this is the delimiter
  }}

\theoremstyle{definition}
\newtheorem{definition}{Definition}
\newtheorem{remark}{Remark}

\usepackage{mathtools}
\DeclarePairedDelimiter\bra{\langle}{\rvert}
\DeclarePairedDelimiter\ket{\lvert}{\rangle}
\DeclarePairedDelimiterX\braket[2]{\langle}{\rangle}{#1 \delimsize\vert #2}

\usepackage{amsmath}
\usepackage{amsfonts}
\usepackage{amssymb}
\usepackage{fancyhdr}

\DeclareMathOperator{\interior}{int}

\newcommand{\Tau}{\mathcal{T}}

\newenvironment{amatrix}[1]{%
  \left(\begin{array}{@{}*{#1}{c}|c@{}}
}{%
  \end{array}\right)
}

\usepackage{calligra}
\DeclareMathAlphabet{\mathcalligra}{T1}{calligra}{m}{n}
\DeclareFontShape{T1}{calligra}{m}{n}{<->s*[2.2]callig15}{}

\newcommand{\scripty}[1]{\ensuremath{\mathcalligra{#1}}}

\pagestyle{fancy}
\author{Jeremiah Givens}
\newcommand{\subject}{Real Analysis I}
\newcommand{\Date}{9/2/2021} 
\makeatletter
\rhead{{\small\@author}}
\lhead{{\small\subject}}
\chead{{\large Homework 4}}
\cfoot{}
\rfoot{\thepage}
\lfoot{\today}

\renewcommand{\theequation}{\arabic{equation}}

\begin{document}
\section*{Problem 1}
\begin{theorem}
Suppose that $f$ is a continuous function a.e. in a measurable set $E \in \mathbb{R}^n$. Show that $f$ is measurable on $E$.
\end{theorem}

\subsection*{Solution}
\begin{proof}
By Theorem 4.5 in our textbook, we can assume, without loss of generality, that $f$ is continuous on all of $E$. Let $a \in \mathbb{R}$. We have
\begin{align*}
x \in \{f > a\} &\iff f(x) > a\\
&\iff (f(x) - a > 0).
\end{align*}
Define $\epsilon_x = f(x) - a$. Since$f$ is continuous, there exists some $\delta_x$ such that $f(B(x; \delta_x)) \subseteq B(f(x); \epsilon_x)$.  Let $y \in B(x; \delta_x)$. Then
\begin{align*}
f(y) \in B(f(x); \epsilon_x) &\implies |f(x) - f(y)| < \epsilon_x\\
&\implies f(y) > f(x) - \epsilon_x\\
&\implies f(y) > f(x) - (f(x) - a)\\
&\implies f(y) > a\\
&\implies y \in \{f > a\}.
\end{align*} 
Thus, $B(x; \delta_x) \subseteq \{f > a\}$, and we have shown that $\{f > a\}$ is an open set. Since every open set is measurable, we have shown that $\{f > a\}$ is measurable. Since $a$ was arbitrary, we have shown that $f$ is measurable, and our proof is complete.
\end{proof}

\section*{Problem 2}
\begin{theorem}
Let $f_k$ ($k=1,2,...$) be measurable functions and finite $a.e.$ on $E$, with $|E| < \infty$. Let $f$ be a function on $E$. Assume that for any $\delta > 0$, there is a measurable set $F \subseteq E$ such that $|E - F| < \delta$ and $f_k \to f$. Then $f_k \to f$ a.e. on $E$ and $f$ is measurable on $E$.
\end{theorem}

\subsection*{Solution}
\begin{proof}
Construct a family of measurable sets $\{F_j\}$ such that for each $j \in \mathbb{N}$, $F_j \subseteq E$ with $|E - F_j| < \frac{1}{j}$, and $f_k \to f$ on $F_j$.  Define a new family of measurable sets $\{E_j \}$ by $E_j = \bigcup_{i = 1}^j F_j$. Then we have that $E_j \nearrow E$ and $E_j \nearrow  \bigcup_{j=1}^\infty F_j$ . By theorem 3.26 in our textbook
\begin{align*}
\left| \bigcup_{j=1}^\infty F_j \right| &= \lim_{k \to \infty} |E_k| = |E|.
\end{align*}
Thus, since $|E| < \infty$, we have that
\begin{align*}
\left| E - \bigcup_{j=1}^\infty F_j \right| = 0.
\end{align*}
Since $f_k \to f$ on $F_j$ for all $j$, we have that $f_k \to f$ on $ \bigcup_{j=1}^\infty F_j $, and our proof is complete.
\end{proof}

\section*{Problem 3}
\begin{theorem}
Let $f_k = \chi_{|x| < k}$ for all $k \in \mathbb{N}$. Then,
\proofpart{(i)}{} $f_k \to 1$ for all $x \in \mathbb{R}^n$.
\proofpart{(ii)}{} $f_k$ does not converge uniformly outside any ball $|x| > m$ for $m > 0$.
\proofpart{(iii)}{} The conclusion in Egorov's theorem does not hold for $\{f_k\}$.
\end{theorem}

\subsection*{Solution}
\begin{proof}
\proofpart{(i)}{} Let $x \in \mathbb{R}^n$. Define $K \in \mathbb{N}$ to be the smallest natural number such that $K \geq 	|x|$. Then, we have that $f_k(x) = 1$ for all $k \geq K$. Since we can do this for any $x$, we have that $f_k \to 1$.

\proofpart{(ii)}{} Suppose, for sake of contradiction, that $f_k$ does converge uniformly to $f$ on some unbounded subset of $\mathbb{R}^n$. Then, there exists some natural $K$ such that $|f_k(x) - 1| < 1$ for all $k \geq K$. Now let $x \in \mathbb{R}^n$ be such that $|x| > K$. Then, $f_K(x) = 0$, and we have a contradiction. Thus, $f_k$ does not converge uniformly on any unbounded set.

\proofpart{(iii)}{} Let $f \subseteq \mathbb{R}^n$ be closed, with $|E - F| < \epsilon$ for some finite $\epsilon > 0$. Now, $F$ must be unbounded, for any bounded subset of $\mathbb{R}^n$ would have finite measure, and thus we would have $|E - F| = \infty$. Therefore, by part (ii), $f_k$ cannot converge uniformly to 1 on $F$, and the conclusion of Ergorov's theorem does not hold.
\end{proof}

\section*{Problem 4}
\begin{theorem}
\proofpart{(i)}{} Assume that $f_k \overset{\text{m}}{\to} f$ on $E$ and $f_k \leq f_{k + 1}$ for natural $k$.  Then, $f_k \to f$ a.e. on $E$.

\proofpart{(ii)}{} Assume that $f_k \overset{\text{m}}{\to} f$ on $E$ and $f_k \leq M$ a.e. on $E$ for each natural $k$. Then $f \leq M$ a.e. on $E$.
\end{theorem}

\subsection*{Solution}
\begin{proof}
\proofpart{(i)}{} 
By theorem 4.22 in our textbook, there exists a subsequence $f_{k_j}$ such that $f_{k_j} \to f$ a.e. on $E$.  Let $\epsilon > 0$, and let $x \in \mathbb{R}^n$ be such that $f_{k_j}(x) \to f(x)$. Then,  there exists some $J \in \mathbb{N}$ such that for all $j \geq J$
\begin{align*}
|f_{k_j}(x) - f(x)| < \frac{\epsilon}{3}.
\end{align*}
Let $k \geq k_J$. Then, there exist $j_1, j_2 \geq J$ such that $k_{j_1} \leq k \leq k_{j_2}$. Then, we have
\begin{align*}
|f_k(x) - f(x)| &\leq |f_k(x) - f_{k_{j_1}}(x)| + |f_{k_{j_1}}(x) - f(x)| &&\text{By the triangle inequality}\\
&= f_k(x) - f_{k_{j_1}}(x) + |f_{k_{j_1}}(x) - f(x)| &&\text{Since } k \geq k_1\\
&\leq f_{k_{j_2}}(x) - f_{k_{j_1}}(x) + |f_{k_{j_1}}(x) - f(x)| &&\text{Since } k \leq k_2\\
&= |f_{k_{j_2}}(x) - f_{k_{j_1}}(x)| + |f_{k_{j_1}}(x) - f(x)| \\
&= 2|f_{k_{j_1}}(x) - f(x)| + |f_{k_{j_2}}(x) - f(x)|&&\text{By the triangle inequality}\\
&< 2\frac{\epsilon}{3} + \frac{\epsilon}{3}\\
&= \epsilon.
\end{align*}

\proofpart{(ii)}{}
Suppose that $f > M$ on a set $F \subseteq E$ with nonzero measure. We can write $F$ as a countable union of disjoint measurable sets
\begin{align*}
F =\bigcup_{j = 1}^\infty \{f > M + \frac{1}{j} \}.
\end{align*}
One of these sets must have nonzero measure. Thus, there exists some $\epsilon > 0$ such that $\{f > M + \epsilon\}$ has non zero measure. Since $f_k \leq M$ a.e. on $E$, we have that for all $k$
\begin{align*}
\{f > M + \epsilon\} - Z \subseteq \{x \in E: |f(x) - f_k(x)| > \epsilon \},
\end{align*}
where $Z$ is a set of measure zero, where each of the $f_k > M$. With this, we have that
\begin{align*}
\lim_{k \to \infty} |\{x \in E: |f(x) - f_k(x)| > \epsilon \}|\geq  |\{f > M + \epsilon\}| > 0,
\end{align*}
showing that $f_k \not \overset{\text{m}}{\to} f$, and proving the contrapositive.
\end{proof}

\section*{Problem 5}
\begin{theorem}
Let $f_k(x) = \frac{x}{k}$ for $x \in \mathbb{R}$ and $k \in \mathbb{N}$. Then $f_x$ does not converge in measure on $\mathbb{R}$.
\end{theorem}

\subsection*{Solution}
\begin{proof}
Clearly, $f_k \to 0$. Let $\epsilon > 0$, and let $k \in \mathbb{N}$. Then, there exists and $X \in \mathbb{R}$ such that for all $x \geq X$,  we have $\frac{x}{k} > \epsilon$. Therefore, 
\begin{align*}
\{x \in E: |f_k(x) - 0| > \epsilon \} \subseteq \{x \geq X \}
\end{align*}
has infinite measure. Since this is true for all $k$, we can conclude that 
\begin{align*}
\lim_{k \to \infty} |\{x \in E: |f_k(x) - 0| > \epsilon \}| = \infty,
\end{align*}
and we have shown that $f_k$ does not converge in measure.
\end{proof}

\section*{Problem 6}
\begin{theorem}
$f_k \overset{\text{m}}{\to} f$ on $E$ if and only if for any $\epsilon > 0$, there exists $N \in \mathbb{N}$ such that for any $k \geq N$,
\begin{align*}
|\{x \in E: |f_k(x) - 0| > \epsilon \}| < \epsilon.
\end{align*}
\end{theorem}

\subsection*{Solution}
\begin{proof}
Suppose first that $f_k \overset{\text{m}}{\to} f$. Then, by definition, we have that for every $\epsilon > 0$,
\begin{align*}
\lim_{k \to \infty} |\{x \in E: |f_k(x) - 0| > \epsilon \}| = 0.
\end{align*}
By directly applying the definition of a limit,  there exists some $N \in \mathbb{N}$ such that for all $k \geq N$,
\begin{align*}
|\{x \in E: |f_k(x) - 0| > \epsilon \}| < \epsilon.
\end{align*}
With this, we have proven the forward direction.

Now suppose that for any $\epsilon > 0$, there exists $N \in \mathbb{N}$ such that for any $k \geq N$,
\begin{align*}
|\{x \in E: |f_k(x) - 0| > \epsilon \}| < \epsilon.
\end{align*}
Fix some $\epsilon_0 > 0$, and let $\epsilon > 0$ be such that $\epsilon < \epsilon_0$. Then, 
\begin{align*}
\{x \in E: |f_k(x) - 0| > \epsilon_0 \} \subseteq \{x \in E: |f_k(x) - 0| > \epsilon \} &\implies |\{x \in E: |f_k(x) - 0| > \epsilon_0 \}| \leq |\{x \in E: |f_k(x) - 0| > \epsilon \}|.
\end{align*}
By initial assumption, there exists $N \in \mathbb{N}$ such that for any $k \geq N$,
\begin{align*}
|\{x \in E: |f_k(x) - 0| > \epsilon \}| < \epsilon.
\end{align*}
Thus, for any $k \geq N$,
\begin{align*}
|\{x \in E: |f_k(x) - 0| > \epsilon_0 \}| < \epsilon,
\end{align*}
and we have shown that 
\begin{align*}
\lim_{k \to \infty} |\{x \in E: |f_k(x) - 0| > \epsilon_0 \}| = 0.
\end{align*}
Thus, we have shown that $f_k \overset{\text{m}}{\to} f$, and our proof is complete.
\end{proof}

\section*{Problem 7}
\begin{theorem}
Let $E$ be measurable and $|E| < \infty$. Then $f_k \overset{\text{m}}{\to} f$ if and only if any subsequence $\{f_{k_n}\}$ has a subsequence $\{ f_{k_{n_j}} \}$ that is convergent to $f$ a.e. on $E$.
\end{theorem}

\subsection*{Solution}
\begin{proof}
Suppose first that $f_k \overset{\text{m}}{\to} f$, and let $\{f_{k_n}\}$ be a subsequence of $f_k$. Let $\epsilon > 0$. Since $f_k \overset{\text{m}}{\to} f$, there exists a $K \in \mathbb{N}$ such that for all $k \geq K$, 
\begin{align*}
|\{x \in E: |f_k(x) - f(x)| > \epsilon \}| < \epsilon.
\end{align*}
Thus, since $k_n \geq K$ for all $n \geq K$, we have that 
\begin{align*}
|\{x \in E: |f_{n_k}(x) - f(x)| > \epsilon \}| < \epsilon
\end{align*}
for all $n \geq K$. Thus, $\{f_{k_n}\}$ converges in measure to $f$. By theorem 4.22 in our textbook,  we can conclude that $\{f_{k_n}\}$ has a subsequence $\{ f_{k_{n_j}} \}$ that is convergent to $f$ a.e. on $E$.

For the reverse direction, we will prove the contrapositive. Suppose then, that  $f_k \not \overset{\text{m}}{\to} f$. Then, there exists some $\epsilon > 0$ such that 
\begin{align*}
\lim_{k \to \infty} |\{x \in E: |f_k(x) - f(x)| > \epsilon \}| \not = 0.
\end{align*}
Then, for some $\delta > 0$, we have that for all $K \in \mathbb{N}$, there exists a $k \geq K$ such that 
\begin{align*}
|\{x \in E: |f_k(x) - f(x)| > \epsilon \}| > \delta.
\end{align*}
Thus, we can define a subsequence $\{f_{k_n}\}$ of $\{f_k\}$ such that 
\begin{align*}
|\{x \in E: |f_{k_n}(x) - f(x)| > \epsilon \}| > \delta
\end{align*}
for all $n \in \mathbb{N}$. Define $E_n \subseteq E$ by 
\begin{align*}
E_n = \{x \in E: |f_{k_n}(x) - f(x)| > \epsilon \}.
\end{align*}
Let $\{ f_{k_{n_j}} \}$ be a subsequence of $\{f_{k_n}\}$, and suppose for sake of contradiction that $f_{k_{n_j}} \to f$ a.e.. Then, there exists some set $Z \subseteq E$ such that $|Z| = 0$ and $f_{k_{n_j}} \to f$ on $E - Z$. Thus, there exists some $J \in \mathbb{N}$ such that for all $j \geq J$, and all $x \in E - Z$
\begin{align*}
|f_{k_j}(x) - f(x)| < \epsilon.
\end{align*}
However, since $|E_{n_j}| > 0$, we must have that $E_{n_j} \cap (E - Z) \not = \emptyset$. Thus, there exists an $x \in E - Z$ such that
\begin{align*}
|f_{k_j}(x) - f(x)| > \epsilon,
\end{align*}
and we have reached a contradiction. Thus, $f_{k_{n_j}} \not \to f$ a.e.. Since this is true for any subsequence of $\{f_{k_n}\}$, our proof is complete.
\end{proof}

\section*{Problem 8}
\begin{theorem}
Assume that $f_k \overset{\text{m}}{\to} f$ on $E$ with $|E| < \infty$. Then, $f^2_k \overset{\text{m}}{\to} f^2$ on $E$.
\end{theorem}

\subsection*{Solution}
Before we jump into this proof, we will need to prove a useful lemma.
\begin{lemma}
Let $\{a_k\}$ and $\{b_k\}$ be two sequences in $R$ with limits $A$ and $B$ respectively. Then,
\begin{align*}
a_k b_k \to AB.
\end{align*}
\end{lemma}

\begin{proof}
Let $\epsilon > 0$.  Assume first that $A$ and $B$ are nonzero, as otherwise the proof would be trivial. Using the handy equality
\begin{align*}
a_kb_k - AB = (a_k - A)B + (b_k - B)A + (a_k - A)(b_k - B),
\end{align*}
we have that for all $k$,
\begin{align*}
|a_kb_k - AB| \leq |a_k - A||B| + |b_k - B||A| + |a_k - A||b_k - B|.
\end{align*}
Define $N_1 \in \mathbb{N}$ to be such that for all $n \geq N_1$,
\begin{align*}
|a_k - A| < \min \left\{ \frac{\epsilon}{3 |B|}, \sqrt{\frac{\epsilon}{3}}\right\}.
\end{align*}
Furthermore, define $N_2 \in \mathbb{N}$ to be such that for all $n \geq N_2$,
\begin{align*}
|b_k - B| < \min \left\{ \frac{\epsilon}{3 |A|}, \sqrt{\frac{\epsilon}{3}}\right\}.
\end{align*}
Finally, define $N \in \mathbb{N}$ to be the larger of $N_1$ and $N_2$, and we can see that for all $k \geq N$, we have
\begin{align*}
|a_kb_k - AB| &< \frac{\epsilon}{3} +  \frac{\epsilon}{3} +  \frac{\epsilon}{3}\\
&= \epsilon.
\end{align*}
With this, we have completed the proof.
\end{proof}
Now we can jump into the main proof.
\begin{proof}
Let $\{f_{k_n}^2 \}$ be a subsequence of $\{f_k^2 \}$. By the result of problem 7, and the fact that $f_k \overset{\text{m}}{\to} f$, we have that there exists a subsequence $\{f_{k_{n_j}}\}$ of $\{f_{k_n}\}$ that converges almost everywhere to $f$. Thus, for almost every $x \in E$, we have that 
\begin{align*}
f_{k_{n_j}}(x) \to f(x).
\end{align*}
Thus, by Lemma 1,  we can see that for almost every $x \in E$, 
\begin{align*}
f_{k_{n_j}}(x)^2 \to f(x)^2.
\end{align*}
Thus,  $\{f_{k_{n_j}}^2\}$ converges almost every where to $f^2$. Utilizing the other direction of what we proved in problem 7, we have shown that $f^2_k \overset{\text{m}}{\to} f^2$ on $E$, and our proof is complete.
\end{proof}
\end{document}