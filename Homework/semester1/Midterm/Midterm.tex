\documentclass[10pt,a4paper]{article}
\usepackage[utf8]{inputenc}
\usepackage[a4paper,%
            left=.75in,right=.75in,top=1in,bottom=1in]{geometry}
\setlength{\headsep}{0.25in}

\usepackage{amsthm}

\usepackage{graphicx}
\usepackage{pgfplots}
            
\usepackage[english]{babel}

\theoremstyle{theorem}
\newtheorem{theorem}{Theorem}
\newtheorem{lemma}{Lemma}
\newtheorem{corollary}{Corollary}
\newtheorem{case}{Case}

\usepackage{amsthm}
\usepackage{lipsum}

\makeatletter
\newcommand{\proofpart}[2]{%
  \par
  \addvspace{\medskipamount}%
  \noindent\emph{Part #1: #2}\par\nobreak
  \addvspace{\smallskipamount}%
  \@afterheading
}
\makeatother

\newcommand\restr[2]{{% we make the whole thing an ordinary symbol
  \left.\kern-\nulldelimiterspace % automatically resize the bar with \right
  #1 % the function
  \vphantom{\big|} % pretend it's a little taller at normal size
  \right|_{#2} % this is the delimiter
  }}

\theoremstyle{definition}
\newtheorem{definition}{Definition}
\newtheorem{remark}{Remark}

\usepackage{mathtools}
\DeclarePairedDelimiter\bra{\langle}{\rvert}
\DeclarePairedDelimiter\ket{\lvert}{\rangle}
\DeclarePairedDelimiterX\braket[2]{\langle}{\rangle}{#1 \delimsize\vert #2}

\usepackage{amsmath}
\usepackage{amsfonts}
\usepackage{amssymb}
\usepackage{fancyhdr}

\DeclareMathOperator{\interior}{int}

\newcommand{\Tau}{\mathcal{T}}

\newenvironment{amatrix}[1]{%
  \left(\begin{array}{@{}*{#1}{c}|c@{}}
}{%
  \end{array}\right)
}

\usepackage{calligra}
\DeclareMathAlphabet{\mathcalligra}{T1}{calligra}{m}{n}
\DeclareFontShape{T1}{calligra}{m}{n}{<->s*[2.2]callig15}{}

\newcommand{\scripty}[1]{\ensuremath{\mathcalligra{#1}}}

\pagestyle{fancy}
\author{Jeremiah Givens}
\newcommand{\subject}{Real Analysis I}
\newcommand{\Date}{9/2/2021} 
\makeatletter
\rhead{{\small\@author}}
\lhead{{\small\subject}}
\chead{{\large Midterm}}
\cfoot{}
\rfoot{\thepage}
\lfoot{\today}

\renewcommand{\theequation}{\arabic{equation}}

\begin{document}
\section*{Problem 1}
Before we jump into this problem, we will first prove an easy lemma.
\begin{lemma}
Let $F, B $ and $E$ be sets with $F \subseteq B \subseteq E$. Then, 
\begin{align*}
(E - F) \cap B &= B - F.
\end{align*}
\end{lemma}

\begin{proof}
We have
\begin{align*}
x \in (E - F) \cap B &\iff (x \in E - F) \land x \in B\\
&\iff (x \in E) \land (x \not \in F) \land (x \in B)\\
&\iff (x \in E \cap B) \land (x \not \in F)\\
&\iff (x \in B) \land (x \not \in F) &&\text{Since } B \subseteq E\\
&\iff x \in B - F,
\end{align*}
and our proof is complete.
\end{proof}

\begin{theorem}
$E \subseteq \mathbb{R}^n$ is measurable if and only if for every $\epsilon >0$ there exists a measurable set $B \subseteq	E$ such that $|E - B|_e < \epsilon$.
\end{theorem}

\subsection*{Solution}
\begin{proof}
Suppose $E$ is measurable, and let $\epsilon >0$. By Lemma 3.22 in our textbook,  there exists a closed set $B \subseteq E$ such that $|E - B|_e < \epsilon$.  Since all closed sets are measurable, $B$ is measurable, and we have proven one direction of the theorem.

Now suppose that for each $\epsilon > 0$,  there exists a measurable set $B \subseteq	E$ such that $|E - B|_e < \epsilon$.  Thus, let $\epsilon > 0$, and fix a measurable $B \subseteq	E$ such that $|E - B|_e < \frac{\epsilon}{2}.$ Since $B$ is measurable, Lemma 3.22 tells us that there exists a closed set $F \subseteq B$ such that $|B - F|_e < \frac{\epsilon}{2}$. Now, by Caratheory's theorem, we have
\begin{align*}
|E - F|_e &= |(E - F) \cap B|_e + |(E - F) - B|_e\\
&= |(E - F) \cap B|_e + |E - B|_e && \text{Since } F \subseteq B\\
&= |B - F|_e + |E - B|_e && \text{By Lemma 1}\\
&< \frac{\epsilon}{2} + \frac{\epsilon}{2}\\
&= \epsilon.
\end{align*}
Thus, utilizing Lemma 3.22 one last time, it follows that $E$ is measurable, and our proof is complete.
\end{proof}

\section*{Problem 2}
\begin{theorem}
Let $\{I_1, I_2, ...,I_N \}$ be a finite family of closed intervals in $\mathbb{R}^1$ such that $\mathbb{Q} \cap [0, 1] \subseteq \bigcup_{j = 1}^N I_j$. Then $\sum_{j = 1}^N |I_j| \geq 1$. Furthermore, if this family of intervals is infinite, this may not be true.
\end{theorem}

\subsection*{Solution}

\begin{proof}
Suppose there exists an $x \in [0, 1] \backslash \mathbb{Q}$ such that $x \not\in \bigcup_{j = 1}^N I_j$. Then, since each $I_j$ is closed, there exist $\delta_j > 0$ such that 
\begin{align*}
(\forall y \in I_j)(|x - y| \geq \delta_j).
\end{align*}
Define $\delta = \min \{\delta_j | j = 1,2,...,N\}$, which exists because there are only finitely many $I_j$. By construction, we have that $\delta > 0$. By density of the rational numbers, there exists a rational number $q \in [0, 1]$ such that $|x - q| < \delta$. Thus,  $q \not \in \bigcup_{j = 1}^N I_j$, and we have a contradiction. 

Therefore, we have that $[0, 1] \subseteq \bigcup_{j = 1}^N I_j$. Now, by definition of Lebesgue measure, we have
\begin{align*}
1 &= |[0, 1]|\\
&\leq \sum_{j = 1}^N |I_j| ,
\end{align*}
and our proof is complete.

For the last remark, let $\{q_1, q_2,...\}$ be an enumeration of $\mathbb{Q} \cap [0, 1] $. Define an infinite family of closed intervals $\{I_1, I_2,...\}$ such that 
\begin{align*}
I_k = [q_k, q_k + \frac{\epsilon}{2}]
\end{align*}
for some $0 < \epsilon <1$.  Clearly, we have $\mathbb{Q} \cap [0, 1] \subseteq \bigcup_{j = 1}^\infty I_j$, but we also have
\begin{align*}
\sum_{j = 1}^\infty |I_j| &= \sum_{j = 1}^\infty |[q_k, q_k + \frac{\epsilon}{2^k}]|\\
&= \sum_{j = 1}^\infty \frac{\epsilon}{2^k}\\
&= \epsilon\\
&<1,
\end{align*}
and we have shown that the statement does not hold for an infinite family of closed intervals.
\end{proof}

\section*{Problem 3}
\begin{theorem}
\proofpart{(i)}{} 
Let $A, B \subseteq \mathbb{R}^n$, with $A$ measurable.  If $A \cap B = \emptyset$, then $|A \cup B|_e = |A| + |B|_e$

\proofpart{(ii)}{} 
If $F \subseteq E$ is closed such that $|E|_e - |F| < \epsilon$ and $|F| < \infty$, then $|E - F|_e < \epsilon$.
\end{theorem}

\subsection*{Solution}
\begin{proof}
\proofpart{(i)}{} 
Since $A$ is measurable, we have
\begin{align*}
|A \cup B|_e &= |(A \cup B) \cap A|_e + |(A \cup B) - A|_e &&\text{By Caratheory's theorem}\\
&= |A| + |(A \cup B) - A|_e &&\text{Since } A \cap B = \emptyset\\
&= |A| + |B|_e &&\text{Since } (A \cup B) - A = B\\
\end{align*}
and we have completed the proof of this part.

\proofpart{(ii)}{}
Since $F$ is closed (and therefore measurable), we can use Caratheodory's theorem once more, to see
\begin{align*}
|E|_e &= |E \cap F|_e + |E - F|_e\\
&= |F| + |E - F|_e. &&\text{Since } F \subseteq E\\ 
\end{align*} 
Since $|F|$ is finite, we can subtract it from both sides to yield
\begin{align*}
|E - F|_e &= |E|_e - |F|\\
&< \epsilon,
\end{align*}
as desired.
\end{proof}

\section*{Problem 4}
\begin{theorem}
Let $A$ and $B$ be subsets of $\mathbb{R}^n$. Then 
\begin{align*}
|A \cup B|_e + |A \cap B|_e \leq |A|_e + |B|_e.
\end{align*}
\end{theorem}

\subsection*{Solution}
\begin{proof}
Let $\epsilon > 0$. By theorem 3.6 in our textbooks, there exist open sets $G_A , G_B$ such that $A \subseteq G_A$, $B \subseteq G_B$, $|G_A| \leq |A|_e + \frac{\epsilon}{2}$, and $|G_B| \leq |B|_e + \frac{\epsilon}{2}$. With this, we have
\begin{align*}
|A \cup B|_e + |A \cap B|_e &\leq |G_A \cup G_B| + |G_A \cap G_B| &&\text{Since } A \cup B \subseteq G_A \cup G_B \text{ and } A \cap B \subseteq G_A \cap G_B\\
&= |G_A| + |G_B| &&\text{From HW2 Problem 4}\\
&\leq |A|_e + \frac{\epsilon}{2} + |B|_e + \frac{\epsilon}{2}\\
&= |A|_e +|B|_e + \epsilon.
\end{align*}
Since this is true for any $\epsilon > 0$, we have
\begin{align*}
|A \cup B|_e + |A \cap B|_e \leq |A|_e + |B|_e,
\end{align*}
and our proof is complete.
\end{proof}

\section*{Problem 5}
\begin{theorem}
If $E \subseteq \mathbb{R}$ is measurable and $|E| > 0$, then there are $x, y \in E$, with $x \not = y$,  such that $x - y$ is a rational number.
\end{theorem}

\subsection*{Solution}

\begin{proof}
Since $E$ has nonzero measure, there exists a bounded subset $F \subseteq E$ such that $|F| > 0$.  Since $F$ is bounded, there exists integers $a < b$ such that $F \subseteq [a, b]$. Thus, it will suffice to show that there are $x,y \in F$ such that $x - y$ is a rational number.

Let $\{q_1, q_2,...\}$ be an enumeration of the rational numbers contained in $[0, 1]$. Define a family of sets $\{F_k\}$ by 
\begin{align*}
F_k &= F + q_1,
\end{align*}
where we have just translated $F$ some rational number in $[0, 1]$. Now, suppose that $F_j \cap F_k = \emptyset$ if $j \not = k$.  Using the fact that $\bigcup_{k=1}^\infty F_k \subseteq [a, b + 1]$, we have
\begin{align*}
|[a, b+1]| &\geq | \bigcup_{k=1}^\infty F_k|\\
&= \sum_{k=1}^\infty |F_k| && \text{Since the sets are disjoint, by assumption}\\
&= \sum_{k=1}^\infty |F| &&\text{By translation invariance of Lebesgue measure}\\
&= \infty, 
\end{align*}
which is a contradiction, since $|[a, b+1]|$ is finite. Thus,  there exist numbers $j, k \in \mathbb{N}$, with $j \not = k$,  such that $F_j \cap F_k \not = \emptyset$. Thus, there exist $f_j,f_k \in F$ such that 
\begin{align*}
f_j + q_j = f_k + q_k.
\end{align*}
Then, we have
\begin{align*}
f_j - f_k = q_k - q_j.
\end{align*}
Since the rational numbers are closed under subtraction, we have that $f_j - f_k$ is rational, and our proof is complete.
\end{proof}

\section*{Problem 6}
\begin{theorem}
Let $E$ be the nonmeasurable set in $I = [0, 1]$ constructed in the lecture. Then
\proofpart{(i)}{}
$|E|_i = 0$.

\proofpart{(ii)}{}
$|I| < |E|_e + |I - E|_e$.

\proofpart{(iii)}{}
$|I| > |E|_i + |I - E|_i$.
\end{theorem}

\subsection*{Solution}
\begin{proof}
\proofpart{(i)}{}
Let $F \subseteq E$ be closed. Then, we have that $F$ is measurable, and that $|F| \leq 1$. Let $\{1_1, q_2,...\}$ be an enumeration of the rational numbers in $[0, 1]$. Define the translated sets $\{F_k\}$ by 
\begin{align*}
F_k = F + q_k.
\end{align*}
Suppose that there exist $j,k \in \mathbb{N}$ such that $F_j \cap F_k \not= \emptyset$. Then, there exists some $x,y \in F$ such that 
\begin{align*}
x + q_j = y + q_k.
\end{align*}
This implies that $x - y = q_k - q_j$, which is a rational number. From this, we have that $y$ is in the equivalence class of $x$. However, since each element of $E$ (and therefore $F$) belong to distinct equivalent classes, we have that $x = y$, which implies that $j = k$. Thus, we can conclude that the sets in $\{F_k\}$ are pairwise disjoint. Furthermore, by translation invariance of the Lebesgue measure, each of these sets is measurable, with measure $|F|$. Since the union of these sets is contained in $[0, 2]$, we have
\begin{align*}
2 &= |[0, 2]|\\
&\geq \left| \bigcup_{k = 1}^\infty F_k \right|\\
&= \sum_{k=1}^\infty |F_k| &&\text{Since these sets are disjoint}\\
&= \sum_{k=1}^\infty |F|\\
&= \infty \cdot |F|.
\end{align*}
Thus, we can conclude that $|F|$ has measure zero. Since this is true for any closed subset of $E$, we have shown that 
\begin{align*}
|E|_i = 0.
\end{align*}
\proofpart{(ii)}{}
As we proved on Problem 7 of Homework 2, we have that
\begin{align*}
|I| = |E|_i + |I - E|_e.
\end{align*}
By part (i), we have that $|I| = |I - E|_e$. Furthermore,  we have that $|E|_e > 0$, since sets of zero outer measure are measurable, and we have already proven that $E$ is not measurable. Thus, since $|I|$ is finite, it follows that 
\begin{align*}
|I| < |E|_e + |I - E|_e.
\end{align*}

\proofpart{(iii)}{}
As we showed above, $|I| = |I - E|_e$. As we also showed on Homework 2, $|I - E|_e \geq |I - E|_i$.  Suppose, for sake of contradiction, that $|I - E|_e = |I - E|_i$. Then, since $|I - E|_e$ is finite, this would imply that $I - E$ is measurable (which we also proved on Homework 2). However, we have
\begin{align*}
I - (I - E) = E,
\end{align*}
and since the set difference of two measurable sets is measurable, we have reached a contradiction. Thus, $|I - E|_e > |I - E|_i$, and we have reached the desired conclusion.
\end{proof}

\section*{Problem 7}
\begin{theorem}
Let $f(x) = x^3$.  If $E \subseteq \mathbb{R}$ is a zero measure set, then $|f(E)| = 0$.
\end{theorem}

\subsection*{Solution}
\begin{proof}
Suppose first that $E$ is bounded. Then, as I proved on homework 1, $f$ is uniformly continuous. Thus, there exists some $M > 0$ such that for all $x,y \in E$, we have
\begin{align*}
|f(x) - f(y)| \leq M|x - y|.
\end{align*}
Furthermore, it follows that for any interval $I$, we have
\begin{align*}
|f(I)| \leq M|I|.
\end{align*}

Now let $\epsilon > 0$. Since $E$ is of measure zero, there exists a cover of $\{I_1, I_2,...\}$ of $E$ consisting of closed intervals such that 
\begin{align*}
\sum_{j=1}^\infty |I_j| < \frac{\epsilon}{M}.
\end{align*} 
We have
\begin{align*}
f(E) = \bigcup_{j=1}^\infty f(I_j),
\end{align*}
therefore
\begin{align*}
|f(E)| &\leq \sum_{j=1}^\infty |f(I_j)| \\
&\leq \sum_{j=1}^\infty M|I_j| \\
&< M \frac{\epsilon}{M}\\
&= \epsilon.
\end{align*}
Since this is true for any $\epsilon$, we can conclude that $|f(E)| = 0$. 

Now, suppose that $E$ is not bounded. We have that
\begin{align*}
E = \bigcup_{k = 1} E \cap B(0; k),
\end{align*}
where $B(0; k)$ is the open ball centered at the origin with radius $k$.  Furthermore, we have
\begin{align*}
f(E) = \bigcup_{k = 1} f(E \cap B(0; k)).
\end{align*}
Now, since $E \cap B(0; k)$ is bounded for each $k$ and has measure zero, the previous result allows us to conclude that $|f(E \cap B(0; k))| = 0$, and therefore that $|f(E)| = 0$, as desired.
\end{proof}

\section*{Problem 8}
\begin{theorem}
If $f$ and $g$ are continuous functions on $\mathbb{R}^n$ and are equal a.e. in $\mathbb{R}^n$, then $f \equiv g$ on $\mathbb{R}^n$.
\end{theorem}

\subsection*{Solution}
\begin{proof}
Suppose that $g$ is continuous, and that $f \not \equiv g$. Then, there exists some point $x_0$ such that $f(x_0) \not = g(x_0)$. Since $f$ and $g$ are equal almost everywhere, we can construct a sequence $\{x_k\}$ such that $f(x_k) = g(x_k)$ and $|x_0 - x_k| < \frac{1}{k}$ for all $k \in \mathbb{N}$. By construction, we have that $x_k \to x_0$. By continuity of $g$, we have that $g(x_k) \to g(x_0)$.  Furthermore, since $f(x_k) = g(x_k)$ for all $k$, we have that $f(x_k) \to g(x_0)$,  implying that $f(x_k) \not \to f(x_0)$. Thus, we have shown that $f$ is not continuous, and we have proven the contrapositive. 
\end{proof}

\section*{Problem 9}
\begin{theorem}
Let $\{f_k \}$ be a sequence of measurable functions defined on a measurable set $E$ with $|E| < \infty$. If $|f_k(x)| \leq M_x < \infty$ for all $k$ for each $x \in E$, then given $\epsilon > 0$, there is a closed $f \subseteq E$ and a finite $M > 0$ such that $|E - F| < \epsilon$ and $|f_k(x)| \leq M$ for all $k$ and all $x \in F$.
\end{theorem}

\subsection*{Solution}
\begin{proof}
By the remark after theorem 4.6 in our textbook, we have that $|f_k|$ is measurable for each $k$. By theorem 4.11 in our text book, $\sup_k |f_k(x)|$ is measurable.  Furthermore, we have by assumption that $\sup_k |f_k(x)|$ is finite everywhere. With this, we can construct a sequence of measurable sets $\{E_k\}$ such that $E_k \nearrow E$ defined by
\begin{align*}
E_k = \{x \in E | \sup_k |f_k(x)| \leq k \}.
\end{align*}
By theorem 3.26 in the book, we have
\begin{align*}
\lim_{k \to \infty} |E_k| = |E|.
\end{align*}
Thus, there exists an $M > 0$ such that 
$|E| - |E_M| < \frac{\epsilon}{2}$. Using Lemma 3.22, we can find a closed $F \subseteq E_M$ such that $|E_M - F| < \frac{\epsilon}{2}$. Using Caratheory's theorem, we have
\begin{align*}
|E_n| &= |E_n \cap F| + |E_n - F|\\
&= |F| + |E_n - F| && \text{Since } F \subseteq E_n\\
|F| &= |E_n| - |E_n - F| &&\text{Since } |E| < \infty \implies |E_n| < \infty \implies |F| < \infty,
\end{align*}
and, using the exact same approach,
\begin{align*}
|E - F| &= |E| - |E \cap F|\\
&= |E| - |F|\\
&= |E| - (|E_n| - |E_n - F| )\\
&= (|E| - |E_n|) + |E_n - F|\\
&< \frac{\epsilon}{2} + \frac{\epsilon}{2}\\
&= \epsilon.
\end{align*}
Finally, since $F \subseteq E_M$, we have that 
\begin{align*}
(\forall x \in F)(\sup_k |f_k(x)| \leq M) \implies (\forall x \in F)(\forall k)(|f_k(x)| \leq M),
\end{align*}
and our proof is complete.
\end{proof}

\section*{Problem 10}
\begin{theorem}
Let $f$ be a measurable function and finite a.e. on $E \subseteq \mathbb{R}^n$. Show that there is a sequence of continuous functions $g_k$ on $\mathbb{R}^n$ such that $g_k \to f$ a.e. on $E$.
\end{theorem}

\subsection*{Solution}
\begin{proof}
By the alternative version of Lusin's theorem, we can construct a sequence of closed sets $\{F^*_k\}$ such that $F^*_k \subseteq E$ and $|E - F^*_k| < \frac{1}{k}$, and a sequence of continuous functions $\{g^*_k\}$ such that $g^*_k:\mathbb{R}^n \to \mathbb{R}$ and $g^*_k(x) = f(x)$ for all $x \in F_k$.

Now, this sequence of continuous functions does not necessarily converge, as  $\{F^*_k\}$ is not necessarily an increasing sequence. One way to remedy this, is to define a new sequence, $\{F_k\}$ such that $F_1 = F^*_1$, and $F_k = F^*_k \cup F_{k-1}$ for $k > 1$. By design, we have that $F_k \nearrow E$, and each $F_k$ is closed.  Using this, we will define a new sequence of functions $\{g_k\}$, where $g_1 = g^*_1$, and $g_{k + 1}(x) = g^*_{k+1}(x)$ when $x \in F^*_{k+1}$,  and $g_{k + 1}(x) = g_{k}(x)$ when $x \not \in F^*_{k+1}$. All that remains is to show that each $g_k$ is continuous, and our proof will be complete.

Since $g^*_1$ is continuous, it follows that $g_1$ is continuous. Now suppose that for some $k \geq 1$ that $g_k$ is continuous.  Let $\{x_j\}$ be any convergent sequence in $\mathbb{R}^n$, and call it's limit $x$. Now, either $x \in F^*_{k+1}$, or $x \not \in F^*_{k+1}$.  Suppose first that $x \in F^*_{k+1}$. Since $F^*_{k+1}$ is closed,  there exists some $J \in \mathbb{N}$ such that $x_j \in F^*_{k+1}$ for all $j \geq J$. Thus,  $g_{k+1}(x_j) = g^*_{k+1}(x_j)$ for all $j \geq J$. Then, by continuity of $g^*_{k+1}$,  $g_{k+1}(x_j) \to g_{k+1}(x)$, and we have that $g_{k+1}$ is continuous in $F^*_{k+1}$.

Now, suppose that $x \not \in F^*_{k+1}$.  Then, since $F^*_{k+1}$ is closed,  there exists some $J \in \mathbb{N}$ such that $x_j \not \in F^*_{k+1}$ for all $j \geq J$. Thus,  $g_{k+1}(x_j) = g_{k}(x_j)$ for all $j \geq J$.  By continuity of $g_k$, we have that $g_{k+1}(x_j) \to g_{k+1}(x)$, and we have shown that $g_{k+1}$ is continuous everywhere. Since we have already shown the base case of $k = 1$, the principle of mathematical induction allows us to conclude that $g_k$ is continuous for all $k$, and our proof is complete.
\end{proof}

\section*{Problem 11}
\begin{theorem}
Let $f$ be a measurable function and finite a.e.  on $[a, b]$. Then, there is a sequence of polynomials $p_k$ such that $p_k \to f$ a.e. on $[a, b]$.
\end{theorem}

\subsection*{Solution}
\begin{proof}
As we proved in the previous problem,  there exists a sequence of continuous functions $g_k$ on $\mathbb{R}^n$ such that $g_k \to f$ a.e. on $[a, b]$. By the Stone-Weierstrass theorem, we can construct a sequence of polynomials $p_k$ such that for all $x \in [a, b]$, $|p_k(x) - g_k(x)| < \frac{1}{k}$. Now, let $x \in [a, b]$ be such that $g_k(x) \to f(x)$, and let $\epsilon > 0$.  Choose $K_1 \in \mathbb{N}$ to be such that $\frac{1}{K_1} < \frac{\epsilon}{2}$. Additionally, choose $K_2 \in \mathbb{N}$ large enough that for all $k \geq K_2$, we have $|g_k(x) - f(x)| < \frac{\epsilon}{2}$. Then, if we define $K$ as the maximum of $K_1$ and $K_2$, we have that for all $k \geq K$,
\begin{align*}
|p_k(x) - f(x)| &\leq |p_k(x) + g_k(x)| + |g_k(x) - f(x)| &&\text{By the triangle inequality}\\
&< \frac{\epsilon}{2} + \frac{\epsilon}{2}\\
&= \epsilon.
\end{align*}
Thus, since $\epsilon$ was arbitrary, we have shown that $p_k \to f$ a.e. on $[a, b]$, and our proof is complete.
\end{proof}

\end{document}