\documentclass[10pt,a4paper]{article}
\usepackage[utf8]{inputenc}
\usepackage[a4paper,%
            left=.75in,right=.75in,top=1in,bottom=1in]{geometry}
\setlength{\headsep}{0.25in}

\usepackage{amsthm}

\usepackage{graphicx}
\usepackage{pgfplots}
            
\usepackage[english]{babel}

\theoremstyle{theorem}
\newtheorem{theorem}{Theorem}
\newtheorem{lemma}{Lemma}
\newtheorem{corollary}{Corollary}
\newtheorem{case}{Case}

\usepackage{amsthm}
\usepackage{lipsum}

\makeatletter
\newcommand{\proofpart}[2]{%
  \par
  \addvspace{\medskipamount}%
  \noindent\emph{Part #1: #2}\par\nobreak
  \addvspace{\smallskipamount}%
  \@afterheading
}
\makeatother

\newcommand\restr[2]{{% we make the whole thing an ordinary symbol
  \left.\kern-\nulldelimiterspace % automatically resize the bar with \right
  #1 % the function
  \vphantom{\big|} % pretend it's a little taller at normal size
  \right|_{#2} % this is the delimiter
  }}

\theoremstyle{definition}
\newtheorem{definition}{Definition}
\newtheorem{remark}{Remark}

\usepackage{mathtools}
\DeclarePairedDelimiter\bra{\langle}{\rvert}
\DeclarePairedDelimiter\ket{\lvert}{\rangle}
\DeclarePairedDelimiterX\braket[2]{\langle}{\rangle}{#1 \delimsize\vert #2}

\usepackage{amsmath}
\usepackage{amsfonts}
\usepackage{amssymb}
\usepackage{fancyhdr}

\DeclareMathOperator{\interior}{int}

\newcommand{\Tau}{\mathcal{T}}

\newenvironment{amatrix}[1]{%
  \left(\begin{array}{@{}*{#1}{c}|c@{}}
}{%
  \end{array}\right)
}

\usepackage{calligra}
\DeclareMathAlphabet{\mathcalligra}{T1}{calligra}{m}{n}
\DeclareFontShape{T1}{calligra}{m}{n}{<->s*[2.2]callig15}{}

\newcommand{\scripty}[1]{\ensuremath{\mathcalligra{#1}}}

\pagestyle{fancy}
\author{Jeremiah Givens}
\newcommand{\subject}{Real Analysis I}
\newcommand{\Date}{9/2/2021} 
\makeatletter
\rhead{{\small\@author}}
\lhead{{\small\subject}}
\chead{{\large Homework 5}}
\cfoot{}
\rfoot{\thepage}
\lfoot{\today}

\renewcommand{\theequation}{\arabic{equation}}

\begin{document}
\section*{Problem 1}
\begin{theorem}
Let $|E| < \infty$ and $E_i$ $(i = 1,2,...,m)$ be measurable subsets of $E$. Let $k \in \{1,2,...,m\}$. Show that if every point of $E$ belongs to at least $k$ of $E_i$, then there is $i$ such that $|E_i| \geq \frac{k}{m}|E|$.
\end{theorem}

\subsection*{Solution}
\begin{proof}
Since each $k \leq \sum_{i = 1}^m \chi_{E_i}$ for each $k \in E$, theorem 5.5 tells us that 
\begin{align*}
\int_E k &\leq \int_E \sum_{i = 1}^m \chi_{E_i}.
\end{align*}
The left hand side of this inequality is easily evaluated to be $k|E|$ by Corollary 5.4. For the right hand side, we can use theorem 5.14 to see
\begin{align*}
\int_E \sum_{i = 1}^m \chi_{E_i} &= \sum_{i = 1}^m \int_E \chi_{E_i}\\
&= \sum_{i = 1}^m |E_i|\\
&\leq \sum_{j = 1}^m \max \{|E_i|: i = 1, 2, ..., m\}\\
&= m \cdot \max \{|E_i|: i = 1, 2, ..., m\}.
\end{align*}
Thus, we have that 
\begin{align*}
k|E| \leq m \cdot \max \{|E_i|: i = 1, 2, ..., m\} &\implies \frac{k}{m} = \max \{|E_i|: i = 1, 2, ..., m\},
\end{align*}
and our proof is complete.
\end{proof}

\section*{Problem 2}
Let $f$ be continuous and nonnegative on $[a, b]$ where $-\infty < a < b < \infty$. Define a nondecreasing sequence of step functions $\{\phi_k \}$ on $[a, b]$ such that $\phi_k \to f$ on $[a, b]$. Then,  use the monotone convergence theorem to show that
\begin{align*}
(L) \int_{[a, b]} f(x) dx = (R) \int_a^b f(x)dx.
\end{align*}

\subsection*{Solution}
Since $f$ is continuous, we have that it is Reimann integrable and Lebesgue integrable. For each $k \in \mathbb{N}$, lets create a partition of $[a, b]$, $\Gamma_k = \{I^k_1, I^k_2,...,I^k_{N_k} \}$ with norm $|\Gamma_k| <\frac{1}{k}$. For each partition, define a function $\phi_k: [a, b] \to \mathbb{R}$ such that for $x \in I^k_n$, we have
\begin{align*}
\phi_k(x) = \inf \{f(x_n): x_n \in I^k_n\}.
\end{align*}
Then, we clearly have that $\phi_k \nearrow f$. By the monotone convergence theorem, we have
\begin{align*}
(L) \int_{[a, b]} \phi_k(x)dx \to (L) \int_a^b f(x)dx.
\end{align*}
Also, we have by corollary 5.4 that
\begin{align*}
(L) \int_{[a, b]} \phi_k(x)dx = \sum_{n = 1}^{N_k} \inf \{f(x_n): x_n \in I^k_n\} \cdot |k|
\end{align*}
which is nothing more than a lower Reimann sum of $f$. Thus, we can conclude that 
\begin{align*}
(L) \int_{[a, b]} \phi_k(x)dx \to (R) \int_a^b f(x)dx,
\end{align*}
and ultimately that
\begin{align*}
(L) \int_{[a, b]} f(x) dx = (R) \int_a^b f(x)dx.
\end{align*}

\section*{Problem 3}
\begin{theorem}
Let $f \geq 0$ be measurable in $\mathbb{R}^n$. For $k=1,2,...,$, define the cut-off functions 
\[   f = \left\{
\begin{array}{ll}
      f(x) & \text{if } f(x)<k \\
      0 & \text{if } f(x) \geq k\\
\end{array} 
\right. \]

Then, (i) each $f_k$ is measurable on $\mathbb{R}^n$ and (ii)
\begin{align*}
\int_{\mathbb{R}^n} f_k(x)dx \to \int_{\mathbb{R}^n} f(x)dx
\end{align*}
as $k \to \infty$.
\end{theorem}

\subsection*{Solution}
\begin{proof}
\proofpart{(i)}{} Let $a \in \mathbb{R}$. Suppose first that $a \geq k$. Then
\begin{align*}
\{f_k > a \} &= \{x \in  \mathbb{R}: f_k(x) > a \}\\
&\subseteq \{x \in  \mathbb{R}: f_k(x) > k \}\\
&= \emptyset,
\end{align*}
and we can conclude that $\{f_k > a \} = \emptyset$, which is measurable.

Now suppose that $0 < a < k$. Then, we have
\begin{align*}
\{f_k > a \} &= \{a < f_k <k \}\\
&= \{a < f < k \},
\end{align*}
which is measurable, since $f$ is measurable. 

Now suppose that $a = 0$. Then,
\begin{align*}
\{f_k > a \} &= \{f_k > 0 \}\\
&= \{0 < f < k \} 
\end{align*}
which is measurable, since $f$ is measurable. 

Finally,  suppose that $a < 0$. Then,
\begin{align*}
\{f_k > a \} &= \{f_k \geq 0 \}\\
&= \mathbb{R}^n &&\text{Since } f_k \text{ is a nonnegative function.}
\end{align*}
Thus, in every case, $\{f_k > a \}$ is measurable, and we have shown that each $f_k$ is measurable.

\proofpart{(ii)}{} 
Since $f$ is finite a.e., theorem 5.10 in our text tells us that it will suffice to show that
\begin{align*}
\int_{E} f_k(x)dx \to \int_{E} f(x)dx,
\end{align*}
where $E \subseteq \mathbb{R}^n$ is the set of all $x \in \mathbb{R}^n$ such that $f(x)$ is finite. Since each $f_k$ is measurable and nonnegative, the monotone convergence theorem tells us that if $f_k \nearrow f$ on $E$, then we will have the desired result. Thus, we will show that $f_k \nearrow f$ on $E$. 

Let $x \in E$. Since $f$ is finite on $E$, there exists some least $K \in \mathbb{N}$ such that $f(x) < K$. Thus, for all $k \geq K$, we have that $f_k(x) = f(x)$. Now suppose that $k < K$. Then, we have $f_k(x) = 0 < f(x)$. Since this is true for any $x$, we have shown that $f_k \leq f$ for all $k$ and that $f_k \to f$. Thus, $f_k \nearrow f$, and our proof is complete.
\end{proof}
\end{document}