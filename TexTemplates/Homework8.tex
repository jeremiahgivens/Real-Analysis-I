\documentclass[10pt,a4paper]{article}
\usepackage[utf8]{inputenc}
\usepackage[a4paper,%
            left=.75in,right=.75in,top=1in,bottom=1in]{geometry}
\setlength{\headsep}{0.25in}

\usepackage{amsthm}

\usepackage{graphicx}
\usepackage{pgfplots}
            
\usepackage[english]{babel}

\theoremstyle{theorem}
\newtheorem{theorem}{Theorem}
\newtheorem{lemma}{Lemma}
\newtheorem{corollary}{Corollary}
\newtheorem{case}{Case}

\newcommand\restr[2]{{% we make the whole thing an ordinary symbol
  \left.\kern-\nulldelimiterspace % automatically resize the bar with \right
  #1 % the function
  \vphantom{\big|} % pretend it's a little taller at normal size
  \right|_{#2} % this is the delimiter
  }}

\theoremstyle{definition}
\newtheorem{definition}{Definition}
\newtheorem{remark}{Remark}

\usepackage{mathtools}
\DeclarePairedDelimiter\bra{\langle}{\rvert}
\DeclarePairedDelimiter\ket{\lvert}{\rangle}
\DeclarePairedDelimiterX\braket[2]{\langle}{\rangle}{#1 \delimsize\vert #2}

\usepackage{amsmath}
\usepackage{amsfonts}
\usepackage{amssymb}
\usepackage{fancyhdr}

\DeclareMathOperator{\interior}{int}

\newcommand{\Tau}{\mathcal{T}}

\newenvironment{amatrix}[1]{%
  \left(\begin{array}{@{}*{#1}{c}|c@{}}
}{%
  \end{array}\right)
}

\usepackage{calligra}
\DeclareMathAlphabet{\mathcalligra}{T1}{calligra}{m}{n}
\DeclareFontShape{T1}{calligra}{m}{n}{<->s*[2.2]callig15}{}

\newcommand{\scripty}[1]{\ensuremath{\mathcalligra{#1}}}

\pagestyle{fancy}
\author{Jeremiah Givens}
\newcommand{\subject}{Metric Spaces}
\newcommand{\Date}{9/2/2021} 
\makeatletter
\rhead{{\small\@author}}
\lhead{{\small\subject}}
\chead{{\large Homework 8}}
\cfoot{}
\rfoot{\thepage}
\lfoot{\today}

\renewcommand{\theequation}{\arabic{equation}}

\begin{document}
\section*{Problem 24}
We will start this problem by proving a useful lemma.
\begin{lemma}
Let $(X, d)$ be a metric space. Then, if every sequence in $X$ has a Cauchy subsequence, $X$ is totally bounded.
\end{lemma}

\begin{proof}
We will prove the contrapositive. Suppose $X$ is not totally bounded. Let $\epsilon > 0$, and let $x_0 \in X$. Since $X$ is not totally bounded, $B(x_0; \epsilon)$ does not cover $X$. Thus there exists an $x_1 \in X$ such that $x_1 \not \in B(x_0; \epsilon)$. Continuing in this manner, we can choose $x_n \in X$ by 
\begin{align*}
x_n \not \in \bigcup_{i = 0}^{n-1} B(x_i; \epsilon)
\end{align*}
for all $n \in \mathbb{N}$. If for any $n$ we fail to find such an $x_n$, this means we have created a finite ($n-1$ element) $\epsilon$-net, which contradicts our assumption that $X$ is not totally bounded.

With this, we have created a sequence $\{x_n \}_{n=0}^\infty$ in $A$ such that $d(x_i, x_j) \geq \epsilon$ for all $i,j \in \mathbb{N}$, from which it follows that $\{x_n \}_{n=0}^\infty$ has no Cauchy subsequence. We have now proven the contrapositive, and our proof is complete.
\end{proof}

\begin{definition}[Closed Box] 
A closed box $A \subseteq \mathbb{R}^n$ is a set that can be written as $\Pi_{i = 1}^n [a_i, b_i]$ where for all $i \in \{1,...,n\}$ and for some $l >0$, we have $b_i - a_i = l$. We say this box has width $l$.
\end{definition}

\begin{lemma}
Any closed box $A$ in $\mathbb{R}^n$ can be written as the union of $2^n$ closed boxes of width $l/2$.
\end{lemma}

\begin{proof}
Consider first the one dimensional case, $n = 1$. Then $A = [a, b]$ for some $a, b \in \mathbb{R}$ such that $b - a = l$. Then, $A = [a, a+l/2] \cup [a + l/2, b]$, and we have written $A$ as the union of two closed boxes of width $l/2$.

Now,  utilizing what we just showed for the one dimensional case, we can see that for any $n \in \mathbb{N}$,
\begin{align*}
A &= \Pi_{i=1}^n ([a_i, b_i])\\
&= \Pi_{i=1}^n ([a_i, a_i + l/2] \cup [a_i+ l/2, b_i]).
\end{align*}
Let $X = \{\Pi_{i=1}^n x_i : x_i \in \{[a_i, a_i + l/2], [a_i+ l/2, b_i]\} \}$.  Clearly, every element of $X$ is a closed box of width $l/2$. By elementary combinatorics, we can also see that $X$ has $2^n$ elements. We propose that $A =  \bigcup_{x \in X} x$. We have
\begin{align*}
a \in \Pi_{i=1}^n ([a_i, a_i + l/2] \cup [a_i+ l/2, b_i]) &\iff (\forall i \in \{1,...,n\})(p_i(a) \in ([a_i, a_i + l/2] \cup [a_i+ l/2, b_i]))\\
&\iff (\forall i \in \{1,...,n\})(p_i(a) \in [a_i, a_i + l/2] \lor p_i(a) \in [a_i+ l/2, b_i])\\
&\iff (\forall i \in \{1,...,n\})(\exists x_i \in \{[a_i, a_i + l/2], [a_i+ l/2, b_i]\})(p_i(a) \in x_i)\\
&\iff a \in \bigcup_{x \in X},
\end{align*}
and our proof is complete.
\end{proof}

\begin{theorem}
Let $A \subset \mathbb{R}^n$ be bounded. Then $A$ is totally bounded.
\end{theorem}

\begin{proof}
Let $x \in A$, and let $\epsilon > 0$.  For two points $x, y$ in a box of width $l$, we have 
\begin{align*}
d(x, y) &= \sqrt{\sum_{i = 1}^n (p_i(x) - p_i(y))^2}\\
&\leq \sqrt{\sum_{i = 1}^n \max\{(p_j(x) - p_j(y))^2 :j \in \{1,...,n\} \}} \\
&= \sqrt{n} \max\{|(p_j(x) - p_j(y)| :j \in \{1,...,n\} \} \}\\
&\leq l \sqrt{n}.
\end{align*}
Thus, for any points $x, y$ in an $n$ dimensional box of width $l = \frac{\epsilon}{\sqrt{n}}$, we have 
\begin{equation}
d(x, y) \leq \epsilon.
\end{equation}

Let $\{x_n \}_{n=1}^\infty$ be a sequence in $A$.  By Lemma 1, it will suffice to show that $\{x_n \}_{n=1}^\infty$ has a Cauchy Subsequence.  Since $A$ is bounded, there exists an $r > 0$ such that $d(x, x_1) \leq r$ for all $x \in A$.  We have, for all $x \in A$, 
\begin{align*}
r &\geq d(x, x_1)\\
&= \sqrt{\sum_{i = 1}^n (p_i(x) - p_i(x_1))^2}\\
&\geq  |p_i(x) - p_i(x_1)| \text{ for all } i \in \{1,...,n\}.
\end{align*}
Thus, we have $A \subseteq b_1$, where $b_1 \subseteq \mathbb{N}^n$ is a box centered at $x_1$ with width $l = r/2$. From Lemma 2, we can divide split $b_1$ into $2^n$ boxes of width $l/2 = r/4$. Since $\{x_n \}_{n=1}^\infty$ is infinite, one of these boxes must have an infinite subsequence of $\{x_n \}_{n=1}^\infty$ inside of it; denote this box $b_2$. Following this procedure, we can define a set of boxes $\{b_n: n \in \mathbb{N} \}$ where $b_n$ is a closed box of width $l = \frac{r}{2^n}$ that results from splitting the previous box into $2^n$ boxes and choosing one that contains an infinite number of terms in the sequence $\{x_n \}_{n=1}^\infty$. By design, $b_{n+1} \subseteq b_{n}$ for all $n \in \mathbb{N}$. 

Let $\{x_{n_j}\}_{j=1}^\infty$ be a subsequence of $\{x_n \}_{n=1}^\infty$ such that $x_{n_j} \in b_j$ for all $j \in \mathbb{N}$. Picking  an $N \in \mathbb{N}$ large enough that $\frac{r}{2^n} \leq \frac{\epsilon}{\sqrt{n}}$, we have from (1) that $d(x_{n_j},  x_{n_k}) \leq \epsilon$ for all $j, k \geq N$. Thus, since $\epsilon$ was arbitrary,  $\{x_{n_j}\}_{j=1}^\infty$ is a Cauchy subsequence of $\{x_n \}_{n=1}^\infty$, and we can conclude that $A$ is totally bounded.
\end{proof}
\section*{Problem 23}
\begin{theorem}
Every sequentially compact metric space is separable. 
\end{theorem}

\begin{proof}
Let $A$ be a sequentially compact metric space. Define $B_{x}^n = B(x; \frac{1}{n})$ for $x \in A$ and $n \in \mathbb{N}$. Let $x_{0}^n \in A$ be arbitrary for all $n \in \mathbb{N}$. Choose $x_{i}^n \in A$ such that $x_{i}^n \in A \backslash(\bigcup_{j = 0}^{i - 1} B_{x_{j}^n}^n)$, for $i \in \mathbb{N}$. Suppose we can choose these for all $i \in \mathbb{N}$. Then,  we'd have a sequence $\{x_{i}^n\}_{i = 0}^\infty$. By design, we have $d(x_{i}^n, x_{j}^n) \geq \frac{1}{n}$ for all $i,j \in \mathbb{N}$. However, since $A$ is sequentially compact, this must have a Cauchy subsequence, which is a contradiction. Therefore, there exists some largest $m_n \in \mathbb{N}$ such that $x_{m_n}^n$ exists.

Define $D \subseteq A$ by 
\begin{align*}
D = \{x_{i}^n : n \in \mathbb{N} \land i \in \{1,..., m_n \}\}.
\end{align*}
Clearly, since $D$ is a countable union of finite sets, $D$ is countable. We propose that $\bar{D} = A$. Since $\bar{D} \subseteq A$, it will suffice to show that $A \subseteq \bar{D}$.

Let $x \in A$, and let $n \in \mathbb{N}$. Suppose, for the sake of contradiction, that for all $i \in \{1, ..., m_n\}$ that $x_{i}^n \not \in B_{x}^n$. Then,
\begin{align*}
d(x_{i}^n, x) \geq \frac{1}{n} &\implies x \not \in B_{x_{i}^n}^n\\
&\implies x \not \in \bigcup_{j = 0}^{m_n} B_{x_{j}^n}^n\\
&\implies x \in A \backslash \left(\bigcup_{j = 0}^{m_n} B_{x_{j}^n}^n\right)\\
&\implies x^{n}_{m_n + 1} \text{ is well defined,}
\end{align*}
which is a contradiction. Therefore, there exists some $x_i^n \in B_x^n$. Since this is true for all $n \in \mathbb{N}$, we have that $x \in \bar{D}$, and we have proven that $A$ is separable.
\end{proof}

\section*{Problem 24}
\begin{theorem}
Let $X$ be compact and let $f:X \to \mathbb{R}$ be upper semi-continuous. Then $f$ attains its supremum.
\end{theorem}

\begin{proof}
Since $f$ is upper semi-continuous, $f^{-1}((-\infty, \alpha)) \in \Tau$, for all $\alpha \in \mathbb{R}$.  Let $x \in X$. Then,  it follows that 
\begin{align*}
f(x) \in (-\infty,f(x) + 1) &\implies x \in f^{-1}((-\infty,f(x) + 1))\\
&\implies (\forall x \in X)(\exists \alpha \in \mathbb{R})(x \in f^{-1}((\infty, \alpha)))\\
&\implies \{f^{-1}((\infty, \alpha)): \alpha \in \mathbb{R} \} \text{ forms an open cover of } X.
\end{align*}
Since $X$ is compact, there exists a finite subcover $\{f^{-1}((-\infty, \alpha_i)): i \in \{1,...,n\} \}$ of $X$. Thus, we have that $\max \{\alpha_i: i \in \{1,...,n\} \}$ is an upper bound of $f(X)$. By the least upper bound property of the real numbers, the supremum
\begin{align*}
M = \sup\{f(x): x \in X \}
\end{align*}
exists.

Now lets suppose, for sake of contradiction, that $f$ never attains its supremum. Then, we have that 
\begin{align*}
\{f^{-1}((-\infty, M - \frac{1}{n})): n \in \mathbb{N}\}
\end{align*}
forms an open covering of $X$. By compactness of $X$, there exists a finite open subcovering
\begin{align*}
\{f^{-1}((-\infty, M - \frac{1}{n_i})): i \in \{1, ..., n\} \}
\end{align*}
of X. Define $m = \max\{n_i:  i \in \{1, ..., n\}\}$. Then, we have that for all $x \in X$, $f(x) < M - \frac{1}{m}$. However, $M - \frac{1}{m} < M$, so we have just contradicted the fact that $M$ is a supremum. Therefore, $f$ must attain it's supremum.
\end{proof}

\end{document}